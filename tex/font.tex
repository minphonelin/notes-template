
\subsubsection{font set}

在 linux 中 latex 字体如果想使用 win 的字体,需要先把 win 的字体拷贝到 linux 中,具体操作可以参考 
\ref{sec:use-win-font-in-ubuntu}

在导言区设置如下,就可以使用 win 的字体:
\begin{lstlisting}[language={matlab}]
\usepackage{fontspec}
%\setmainfont{Times New Roman}   % 这些字体可以用 fc`-`list 中查看名字
\setmainfont{Calibri}
\setsansfont{Verdana}
%\setmonofont{Courier New}
%\setmonofont{Consolas}
\end{lstlisting}

\begin{lstlisting}[language={sh}]
minphone@mcp:~$fc`-`list
/usr/share/fonts/windows_fonts/BROADW.TTF: `\textcolor{red}{Broadway}`:style=Regular,obyčejné
/usr/share/fonts/windows_fonts/consola.ttf: `\textcolor{red}{Consolas}`:style=Regular
/usr/share/fonts/windows_fonts/constanb.ttf: `\textcolor{red}{Constantia}`:style=Bold
/usr/share/fonts/windows_fonts/ONYX.TTF: `\textcolor{red}{Onyx}`:style=Regular,obyčejné
\end{lstlisting}

{\color{red}按以下方式来使用字体:}
\begin{enumerate}[topsep=0pt,itemsep=0pt,parsep=0pt,leftmargin=3.6em,label=\arabic*>]
    \item \{\textbackslash fontspec\{Microsoft YaHei Light\} 微软雅黑 Light\}:
        {\fontspec{Microsoft YaHei Light}微软雅黑 Light}  % for win-font
    \item \{\textbackslash fontspec\{YouYuan\} 幼圆\}:{\fontspec{YouYuan}幼圆}  % for win-font
    \item \{\textbackslash fontspec\{Comic Sans MS\} Comic Sans MS\}:{\fontspec{Comic Sans MS}Comic Sans MS}  % for win-font
    \item \{\textbackslash small\textbackslash fontspec\{Comic Sans MS\} Comic Sans MS\}
        {\small\fontspec{Comic Sans MS}Comic Sans MS}  % for win-font

    \item \{\textbackslash fontsize\{8pt\}\{0\}\textbackslash fontspec\{Comic Sans MS\} Comic Sans MS\}
        {\fontsize{8pt}{0}\fontspec{Comic Sans MS}Comic Sans MS}  % for win-font

    \item \{\textbackslash small\textbackslash fontspec\{Consolas\} Consolas\}:
        {\small\fontspec{Consolas}Consolas}  % for win-font
        | small {\color{teal}就是小五号}
    \item \verb=\textup{upright shape}= : \textup{upright shape}
    \item \verb=\textit{italic shape}= : \textit{italic shape}
    \item \verb=\textsl{slanted shape}= : \textsl{slanted shape}
    \item \verb=\textsc{small capitals shape}= : \textsc{small capticals shape}
    \item \verb=\textbf{apple}= : \textbf{apple}
\end{enumerate}

\paragraph{set font size}~{}

\begin{enumerate}[topsep=0pt,itemsep=0pt,parsep=0pt,leftmargin=3.6em,label=\arabic*>]
    \item \verb=\tiny{Text}= : {\tiny{Text}}
    \item \verb=\small{Text}= : {\small{Text}}
    \item \verb=\normalsize{Text}= : {\normalsize{Text}}
    \item \verb=\large{Text}= : {\large{Text}}
    \item \verb=\huge{Text}= : {\huge{Text}}
    \item \verb={\fontsize{8pt}{0}Text}= : {\fontsize{8pt}{0}Text}
\end{enumerate}

各个字号的榜值 : \url{https://blog.csdn.net/weixin_39679367/article/details/115794548}

\paragraph{font style of math}~{}

\begin{enumerate}[topsep=0pt,itemsep=0pt,parsep=0pt,leftmargin=3.6em,label=\arabic*>]
    \item \verb=$\mathnormal{ABCDEF, abcdef, 123456}$= : {$\mathnormal{ABCDEF, abcdef, 123456}$}
    \item \verb=$\mathnormal{123456}$= : {$\mathnormal{123456}$}
    \item \verb=$\mathrm{ABCDEF, abcdef, 123456}$= : {$\mathrm{ABCDEF, abcdef, 123456}$}
    \item \verb=$\mathit{ABCDEF, abcdef, 123456}$= : {$\mathit{ABCDEF, abcdef, 123456}$}
    \item \verb=$\mathbf{ABCDEF, abcdef, 123456}$= : {$\mathbf{ABCDEF, abcdef, 123456}$}
    \item \verb=$\mathsf{ABCDEF, abcdef, 123456}$= : {$\mathsf{ABCDEF, abcdef, 123456}$}
    \item \verb=$\mathtt{ABCDEF, abcdef, 123456}$= : {$\mathtt{ABCDEF, abcdef, 123456}$}
    \item \verb=$\mathcal{ABCDEF}$= : {$\mathcal{ABCDEF}$}
\end{enumerate}

