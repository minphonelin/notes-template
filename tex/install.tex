
\subsection{install}

\subsubsection{separate installation}

xelatex 可以编译出带目录的 pdf 文档 : (这两个安装包都很大,每个都几百M)

\begin{enumerate}[topsep=0pt,itemsep=0pt,parsep=0pt,leftmargin=3.6em,label=\arabic*>]
    \item sudo apt-get install texlive-xetex
    \item sudo apt-get install latex-cjk-all | chinese support
    \item sudo apt-get install texmaker | texmaker is tex editor
\end{enumerate}

\subsubsection{integrated installation}

推荐使用这种安装方式,且安装也比较简单

TeX Live Home Page : \url{http://tug.org/texlive/acquire-netinstall.html}

Texmaker Home Page : \url{http://www.xm1math.net/texmaker/}

\begin{enumerate}[topsep=0pt,itemsep=0pt,parsep=0pt,leftmargin=3.6em,label=\arabic*>]
    \item download texlive from TexLive HomePage : eg. texlive2019-20190410.iso
    \item sudo mount -o loop texlive2019-20190410.iso /media/iso
    \item cd /media/iso
    \item install texlive \par
        {\color{DefinedColorGreen}./install-tl}
    \item set .bashrc \par
        {\color{violet}export PATH=\verb!"!/usr/local/texlive/2019/bin/x86\_64-linux:\$PATH\verb!"!}
    \item set windows fonts, ref section \ref{sec:use-win-font-in-ubuntu}
\end{enumerate}

