
\subsection{other}

%%\subsubsection{chen shuo config}
%%这个是陈硕的 latex 常用设置,其中的反斜杠使用正斜杠替代,这样方便显示
%%
%%https://blog.csdn.net/solstice/article/details/638
%%
%%\paragraph{段首缩进}~{}
%%
%%中文习惯在段首缩进两格,在LaTeX中,/parindent 表示段首缩进的长度,我们将它设置为当前字号的两个
%%大写字母M的宽度,大约正好是两个汉字的宽度:{\color{DefinedColorRed}/setlength\{/parindent\}\{2em\}}
%%
%%LaTeX 默认每节的第一段的段首不缩进,这不符合中文排版习惯。我们希望正文的每一段都要缩进,使用
%%indentfirst宏包就可办到:{\color{DefinedColorRed}/usepackage\{indentfirst\}}
%%
%%\paragraph{段距行距}~{}
%%
%%LaTeX 用/baselineskip表示当前的行距,其默认值大约是当前字号的1.2倍,如果当前字号是10pt,
%%那么/baselineskip是12pt。这对英文排版是合适的,对中文就显得太拥挤了,因为英文正文多为小
%%写字母,字高与小写的x差不多(即1ex)。如果字号为10pt,那么1ex =4.3pt。我通常把行距设为字
%%号的1.8倍:
%%
%%{\color{DefinedColorRed}/setlength\{/baselineskip\}\{1.8em\}}
%%
%%这个值随时可以改,对更改点以后的文字有效
%%
%%LaTeX 用/parskip表示段距,我一般把它设为1ex:{\color{DefinedColorRed}/setlength\{/parskip\}\{1ex\}}
%%
%%注意这些修改长度的命令最好都放在正文区(即/begin\{document\}之后)
%%
%%\paragraph{页眉页脚}~{}
%%
%%/documentclass[10pt, a4paper]{book}
%%/usepackage{fancyhdr}
%%
%%在 LaTeX 中先把 page style 设为fancy,再设置这个style中的页眉和页脚。但是它默认每章的
%%第一页的page style是plain,需要单独处理。
%%
%%\begin{lstlisting}[language={matlab}]
%%% 设置 plain style 的属性
%%/fancypagestyle{plain}{%
%%/fancyhf{}                    % 清空当前设置
%%
%%% 设置页眉 (head)
%%/fancyhead[RE]{/leftmark}     % 在偶数页的右侧显示章名
%%/fancyhead[LO]{/rightmark}    % 在奇数页的左侧显示小节名
%%/fancyhead[LE,RO]{~/thepage~} % 在偶数页的左侧,奇数页的右侧显示页码
%%
%%% 设置页脚:在每页的右下脚以斜体显示书名
%%/fancyfoot[RO,RE]{/it Typesetting with /LaTeX}
%%
%%/renewcommand{/headrulewidth}{0.7pt} % 页眉与正文之间的水平线粗细
%%/renewcommand{/footrulewidth}{0pt}
%%}
%%
%%/pagestyle{fancy}             % 选用 fancy style
%%% 其余同 plain style
%%/fancyhf{}                   
%%/fancyhead[RE]{/leftmark}
%%/fancyhead[LO]{/rightmark}
%%/fancyhead[LE,RO]{~/thepage~}
%%/fancyfoot[RO,RE]{/it Typesetting with /LaTeX}
%%/renewcommand{/headrulewidth}{0.7pt}
%%/renewcommand{/footrulewidth}{0pt}
%%
%%% 设置章名和节名的显示方式
%%/renewcommand{/chaptermark}[1]{/markboth{~第~/thechapter~章~~~#1~}{}}
%%/renewcommand{/sectionmark}[1]{/markright{~/thesection~~#1~}{}}
%%\end{lstlisting}
%%
%%\paragraph{章节标题}~{}
%%
%%通常用titlesec宏包来设置正文中出现的章节标题的格式:
%%
%%/usepackage{titlesec}
%%
%%设置章名为右对齐,字号为/Huge,字型为黑体,章号用粗体,并设置间距:
%%
%%/titleformat{/chapter}{/flushright/Huge/hei}{{/bf /thechapter}}{0pt}{}
%%
%%/titlespacing{/chapter}{0pt}{-20pt}{25pt}
%%
%%设置节名的字号为/Large,字型为黑体,节号用粗体,并设置间距:
%%
%%/titleformat{/section}{/Large /hei}{{/bf /thesection/space}}{0pt}{}
%%
%%/titlespacing*{/section}{0pt}{1ex plus .3ex minus .2ex}{-.2ex plus .2ex}
%%
%%其中/hei的定义为:
%%
%%/newcommand{/hei}{/CJKfamily{hei}}
%%
%%\paragraph{纸张大小}~{}
%%
%%\begin{lstlisting}[language={matlab}]
%%/documentclass[10pt, b5paper]{report}
%%/usepackage[body={12.6cm, 20cm}, centering, dvipdfm]{geometry}
%%% 以上将版心宽度设为 12.6cm,高度 20cm,版心居中,且自动设置PDF文件的纸张大小
%%\end{lstlisting}
%%
%%\subsubsection{documentclass}
%%
%%latex 有三种文档类型(CTE 环境):ctexart、ctexrep、ctexbook
%%
%%优先选用 ctexrep 文档类型,因为这个支持到 chapter,且\textcolor{red}{目录}后面不会添加一个空白页
%%
%%而 book 类型是要考虑印刷时的奇偶页问题,所以会在\textcolor{red}{目录}后面在不满足 chapter 所在页是奇数页时会添加一个空白页,
%%这个是 book 规范设置
%%
