
%
% practice.tex
%
% 0>
%     '\' : 开始的 string 表示一个命令
%     {}  : 其中的内容表示命令的参数
%     []  : 表示命令的选项,是可以忽略的, 省略了则表示使用的默认值
%
%     \documentclass : 表示文档类
%
%     \usepackage : 用于载入 Latex 宏包
%
% 1>  \begin{document} ... \end{document} : 是正文区
% 2>  \documentclass ... \begin{} : 是导言区, 用于载入宏包,定义命令和调整格式 等
% 3>  \textbackslash 是输入 '\'
% 4>  paragraph 是段 | 不编目录,且默认不编号
% 5>  \vspace{-0.5em} 设置垂直方向间距,作用于当前行
%

%
% [Note] 对于 graph 中的 scale 的设置推荐值,具体实际差异再稍微调整 scale
%  1> draw.io 145% 截图的,使用 scale=0.75
%     draw.io 120% 截图的,使用 scale=0.70
%  2> staruml 100% 截图的,使用 scale=0.70
%


%\documentclass[a4paper]{article}
%\documentclass[UTF8]{ctexart} % support : section 及以下
\documentclass[UTF8]{ctexrep} % support : chapter 及以下

%\documentclass[UTF8]{ctexbook} % support : part / chapter 及以下

\usepackage[bookmarks=true]{hyperref}
\hypersetup{
    CJKbookmarks=true,
    colorlinks=true,
    bookmarksnumbered=true
}

\usepackage{geometry}
\geometry{papersize={21cm, 29.7cm}}
\geometry{left=1.27cm, right=1.27cm, top=2.00cm, bottom=2.00cm}

%
% math 绘图
%
\usepackage[all,pdf]{xy}

%
% 使用 AMS-Latex 来提供数学表达式输入
%
\usepackage{amsmath}

\usepackage{graphicx}

\usepackage{enumitem}

\setlength{\lineskiplimit}{10bp}
\setlength{\lineskip}{10bp}
\setlength{\parskip}{0pt}

\usepackage{latexsym}
\usepackage{amssymb}

%
% 现在确定比较好看的字体:
% 1> Microsoft YaHei UI
% 2> Calibri
%

%
% book
%
%\CTEXsetup[format+={\raggedright},beforeskip=0pt,afterskip=0pt]{part}
\CTEXsetup[name={,},
           number={\thechapter},
           numberformat+={\upshape},
           format+={\raggedright\slshape\fontspec{Calibri}},
           beforeskip=0pt,
           afterskip=1em]{chapter}

\CTEXsetup[format+={\raggedright\fontspec{Calibri}},beforeskip=0pt,afterskip=0pt]{section}
\CTEXsetup[format+={\fontspec{Calibri}},beforeskip=0pt,afterskip=0pt]{subsection}
\CTEXsetup[format+={\fontspec{Calibri}},beforeskip=0pt,afterskip=0pt]{subsubsection}
\CTEXsetup[format+={\fontspec{Calibri}},beforeskip=0pt,afterskip=0pt]{paragraph}
\CTEXsetup[format+={\fontspec{Calibri}},beforeskip=0pt,afterskip=0pt,indent=0pt]{subparagraph}

\usepackage{ltxtable}

\usepackage[font={small,sl},labelfont=md]{caption}

\usepackage{color}

\definecolor{LightGreen}{RGB}{0,190,2}
\definecolor{DefinedColorGreen}{RGB}{0,120,2}
\definecolor{DefinedColorRed}{RGB}{255,0,0}
\definecolor{DefinedColorDarkRed}{RGB}{120,8,13}
\definecolor{DefinedColorYellow}{RGB}{230,90,7}

\usepackage[usernames,dvipsnames]{xcolor}
\usepackage{framed}
\colorlet{shadecolor}{gray!20}

%
% font 1
%
%\usepackage{txfonts}

%
% font 2
%
%\usepackage[T1]{fontenc}
%\usepackage{ccfonts,eulervm}
%\usepackage{lmodern,mathrsfs}

%
% font 3 | 下面设置的几种字体需要 win 来提供
%
% fontspec 所能使用的字体就是 fontconfig 库所能够找到的所有字体,也就是 Tex 发行版与操作系统
% 中所安装的 字体
% 字体列表可以使用命令 fc-list 来列出
%
\usepackage{fontspec}
%\setmainfont{Times New Roman}
\setmainfont{Calibri}
\setsansfont{Verdana}
%\setmonofont{Courier New}
%\setmonofont{Consolas}

\usepackage{listings}
\lstset{numbers=left,
        %numberstyle=\tiny,
        numberstyle=\small,
        keywordstyle=\color{blue!70}, % !70 表示比例是 0.7
        %commentstyle=\color{red!50!green!50!blue!50},
        commentstyle=\color[RGB]{50,185,50},
        frame=shadowbox,
        rulesepcolor=\color{red!20!green!20!blue!20},
        escapeinside=``,
        xleftmargin=2em,
        xrightmargin=2em,
        aboveskip=1em,
        belowskip=0em,
        backgroundcolor=\color[RGB]{245,245,244},
        breaklines,                 % 自动折行
        extendedchars=false,        % 解决代码跨页时,章节标题、页眉等汉字不显示的问题
        tabsize=4,                  % 设置 tab 为 4 个空格
        %basicstyle=\small          % 这么设置只会对代码起作用,对注释不起作用
        %basicstyle=\small\ttfamily  % 对注释和代码都起作用
        basicstyle=\small\ttfamily\fontspec{Consolas}  % 对注释和代码都起作用
}

\setcounter{secnumdepth}{7}
\setcounter{tocdepth}{7}

%
% 这个是用来设置 目录 的缩进
%
\usepackage{tocloft}
\setlength{\cftsecindent}{1em}
\setlength{\cftsubsecindent}{2em}
\setlength{\cftsubsubsecindent}{3em}
\setlength{\cftparaindent}{4em}
\setlength{\cftsubparaindent}{5em}

%
% 这个是设置页眉
%
\usepackage{fancyhdr}
\pagestyle{fancy}
\rhead{\textsl{xxx@xxx.com}}
\setlength{\headsep}{10pt}
\setlength{\skip\footins}{0.5cm}

\usepackage{picinpar}

%
% 设置了这个之后就不会在最后一行发生 溢出
%
\usepackage{blindtext}

%
% 19/06/21 : 已经实验验证过,这后面两个设置可以不需要也不会再发生溢出
%
% \usepackage[textwidth=14.5cm] % 这个数值需要重新计算
% \parindent=0pt
%

\usepackage{array}
\usepackage{multirow}
\usepackage{tabularx}
\usepackage{tabu}
\usepackage{colortbl}
\setlength{\arrayrulewidth}{0.19mm}

\usepackage{threeparttable}
\usepackage{ulem}

%
% 设置 longtable 的表格的下间距
%
%\setlength{\LTpre}{0em}
\setlength{\LTpost}{0em}

\renewcommand\thefootnote{\textcircled{\arabic{footnote}}}

%
% 这个设置是让 \verb 中的字体和全文保持一致
%
%\makeatletter
%\renewcommand*{\verbatim@font}{}
%\makeatother

%
% 这个是让 \url{} 中的 url 地址能够折行显示在文档中
%
\makeatletter
\def\UrlAlphabet{%
        \do\a\do\b\do\c\do\d\do\e\do\f\do\g\do\h\do\i\do\j%
        \do\k\do\l\do\m\do\n\do\o\do\p\do\q\do\r\do\s\do\t%
        \do\u\do\v\do\w\do\x\do\y\do\z\do\A\do\B\do\C\do\D%
        \do\E\do\F\do\G\do\H\do\I\do\J\do\K\do\L\do\M\do\N%
        \do\O\do\P\do\Q\do\R\do\S\do\T\do\U\do\V\do\W\do\X%
        \do\Y\do\Z}
\def\UrlDigits{\do\1\do\2\do\3\do\4\do\5\do\6\do\7\do\8\do\9\do\0}
\g@addto@macro{\UrlBreaks}{\UrlOrds}
\g@addto@macro{\UrlBreaks}{\UrlAlphabet}
\g@addto@macro{\UrlBreaks}{\UrlDigits}
\makeatother

%
% 这个命令是能够 让需要着色的文字的前面有一个小间隙
%
% eg. \ColorStr{DefinedColorGreen}{xxx}
%
\newcommand\ColorStr[2]{{\color{#1}{#2}}}

\newcommand\FontNewRoman[1]{{\fontspec{Times New Roman}{#1}}}
%\newcommand\CmdFont[1]{{\small\ttfamily\fontspec{Consolas}{#1}}}
\newcommand\CmdFont[1]{{\fontsize{8.8pt}{0}\ttfamily\fontspec{Consolas}{#1}}}

%
% 这几个不能工作,原因是 \verb 不能使用 \Verb 方式来使用
%
%\newcommand\Verb[1]{{\small\ttfamily\fontspec{Consolas}\verb|{#1}|}}
%\newcommand\Verb[1]{\verb!{#1}!}
%\newcommand\VerbEnv[2][!]{{\small\ttfamily\fontspec{Consolas}\verb{#1}{#2}{#1}}}


%
%-----------------------------------------------------------------------------------------
%

\begin{document}
\begin{sloppypar}


%
% body begin ...
%

%
% set title page
%

\begin{titlepage}
    \vspace*{\fill}
    \begin{center}
        {\Huge\textsl{\textbf{积累笔记-Tex}}}

        \bigskip
        {\zihao{3}{V 0.1.0.0402}}\footnote{Total-commit-count:0} % version

        \bigskip
        {\small\textsl{\textbf{Minphone Lin}}}

    \end{center}
    \vspace{\stretch{3}}
\end{titlepage}




\newpage
\tableofcontents

%\chapter{preface}


\section{latex}

\textsl{\url{http://www.ctex.org/OnlineDocuments}}

\textsl{\url{http://www.ctex.org/documents/packages/layout/fancyhdr.htm}}

\url{http://cremeronline.com/LaTeX/minimaltikz.pdf} : Tikz 制图

\url{http://www.texample.net/tikz/}

\url{https://www.latexlive.com/} : 在线 tex formula 验证

\url{https://blog.csdn.net/lishoubox/article/details/6783726} : 这个基础知识写的不错

\url{https://mathpretty.com/10864.html} : 各种小技巧


\subsection{install}

\subsubsection{separate installation}

xelatex 可以编译出带目录的 pdf 文档 : (这两个安装包都很大,每个都几百M)

\begin{enumerate}[topsep=0pt,itemsep=0pt,parsep=0pt,leftmargin=3.6em,label=\arabic*>]
    \item sudo apt-get install texlive-xetex
    \item sudo apt-get install latex-cjk-all | chinese support
    \item sudo apt-get install texmaker | texmaker is tex editor
\end{enumerate}

\subsubsection{integrated installation}

推荐使用这种安装方式,且安装也比较简单

TeX Live Home Page : \url{http://tug.org/texlive/acquire-netinstall.html}

Texmaker Home Page : \url{http://www.xm1math.net/texmaker/}

\begin{enumerate}[topsep=0pt,itemsep=0pt,parsep=0pt,leftmargin=3.6em,label=\arabic*>]
    \item download texlive from TexLive HomePage : eg. texlive2019-20190410.iso
    \item sudo mount -o loop texlive2019-20190410.iso /media/iso
    \item cd /media/iso
    \item install texlive \par
        {\color{DefinedColorGreen}./install-tl}
    \item set .bashrc \par
        {\color{violet}export PATH=\verb!"!/usr/local/texlive/2019/bin/x86\_64-linux:\$PATH\verb!"!}
    \item set windows fonts, ref section \ref{sec:use-win-font-in-ubuntu}
\end{enumerate}



\subsection{basic}

\begin{enumerate}[topsep=0pt,itemsep=0pt,parsep=0pt,leftmargin=3.6em,label=\arabic*>]
    \item 输入一个反斜杠 : {\color{red}\ \verb|\textbackslash|} => ``\textbackslash"
    \item 输入大括号 : {\color{red}\ \verb|\{|} => ``\{"
    \item 输入百分号 : {\color{red}\ \verb|\%|} => ``\%"
    \item 输入一个空格 : ``\textbackslash\ " 或者 ``{ }"
    \item 输入\CmdFont{--} : \verb!\CmdFont{!\CmdFont{--}\verb!}! 会比较好看
    \item 中文标点(zh) ‘ | ; | ’ | , | 。| “ | ” \par
        \textcolor{teal}{就正常中文输入法中输入的标点符号,就可以直接显示为中文标点}
        {\color{red}(注意是 Calibri 字体的情况下才可以)}

    \item {\color{blue}Times New Roman 字体下的标点(en)}:\par
        \verb={\fontspec{Times New Roman}' | ' | " | " | , | ; | `}= 
        {\color{DefinedColorRed}$\Rightarrow$} % =>
        {\fontspec{Times New Roman}' | ' | " | " | , | ; | `} \par
        \verb={\fontspec{Times New Roman} ' "A" and "B" '}=
        {\color{DefinedColorRed}$\Rightarrow$} % =>
        {\fontspec{Times New Roman}' "A" and "B" '} \par
    \item {\color{blue}Calibri 字体下的标点(en)}{\color{red}(这种输入方式的字体会比较多)}:\par
        \verb={\fontspec{Calibri} ` | ' | `` | " | , | ; | \textasciigrave }=
        {\color{DefinedColorRed}$\Rightarrow$} % =>
        ` | ' | `` | " | , | ; | \textasciigrave \par
        \verb={\fontspec{Calibri} `\,``A" and ``B"\,' }=
        {\color{DefinedColorRed}$\Rightarrow$} % =>
        `\,``A" and ``B"\,'

    \item using symbol:\textbackslash symbol\{96\} {\color{DefinedColorRed}$\to$} \symbol{96} 、
        \textbackslash symbol\{39\} {\color{DefinedColorRed}$\to$} \symbol{39} \par
        通过使用 symbol 来输入字符,可以输入任意有编码的字符
    \item \^{}:input(\textbackslash \^{}\{\})
    \item \~{}:input(\textbackslash \~{}\{\}),在 \textbackslash CmdFont 中也可以使用这个方法来输入 \~{}
    \item \{\textbackslash songti 宋体\}:{\songti 宋体}
    \item \{\textbackslash kaishu 楷书\}:{\kaishu 楷书}
    \item \{\textbackslash heiti 黑体\}:{\heiti 黑体}
    \item \{\textbackslash fangsong 仿宋\}:{\fangsong 仿宋}
    \item \{\textbackslash fontspec\{YouYuan\} 幼圆\}:{\fontspec{YouYuan}幼圆}  % for win-font
    \item \{\textbackslash fontspec\{Consolas\} Consolas\}:{\fontspec{Consolas}Consolas}  % for win-font
        \textcolor{teal}{| 常规五号}
    \item \{\textbackslash zihao\{-5\}\textbackslash fontspec\{Consolas\} Consolas\}:
        {\zihao{-5}\fontspec{Consolas}Consolas}  % for win-font
        \textcolor{teal}{| 小五号}
    \item \{\textbackslash small\textbackslash fontspec\{Consolas\} Consolas\}:
        {\small\fontspec{Consolas}Consolas}  % for win-font
        \textcolor{teal}{| small 就是小五号}
    \item \textbackslash verb:eg. \verb'\verb!\songti!' 
        $\Rightarrow$ % =>
        \verb!\songti! \par
        \textcolor{red}{`!'} 表示环境符号,也可以用其他的符号替代 \par
        \verb={\color{teal}\verb!$\max_n$!}= => {\color{teal}\verb!$\max_n$!} : 给 \textbackslash verb 包围的上色
    \item 数字的下划线分割(使用公式环境):eg. \verb=0xf000$\underline\ $0000= 
        $\Rightarrow$ 
        0xf000$\underline\ $0000 \par
        {\color{teal}在 Calibri 字体下则可以直接使用\_,而不需要使用\textbackslash underline}
    \item url link:\par
        eg. \textbackslash url\{https://xxxx\} 
        $\Rightarrow$ % =>
        \url{https://xxxx} \par
        eg. \verb=\textsl{\url{/https://xxxx}}=
        $\Rightarrow$ % =>
        \textsl{\url{https://xxxx}} \par
        eg. \verb=\emph{\url{/https://xxxx}}=
        $\Rightarrow$ % =>
        \emph{\url{https://xxxx}} | \textcolor{teal}{在 Calibri 字体下会比较好看}
    \item \textbackslash vspace\{-0.8em\}:设置前后行的间距
    \item \textbackslash dotfill:绘制虚线分割线,可以查看 section \ref{sec:book-list} 中的使用效果
    \item \textbackslash hrulefill:绘制实线分割线
\end{enumerate}

\subsubsection{lstlisting / verbatim}

\paragraph{enumerate}~{}

\begin{lstlisting}[language={python}]
\begin{enumerate}

\begin{enumerate}[topsep=0pt,itemsep=0pt,parsep=0pt,leftmargin=3.6em,label=\arabic*>]
\begin{enumerate}[topsep=0pt,itemsep=0pt,parsep=0pt,leftmargin=3.6em,label=\arabic*.]
\begin{enumerate}[topsep=0pt,itemsep=0pt,parsep=0pt,leftmargin=3.6em,label=\arabic*)]
\begin{enumerate}[topsep=0pt,itemsep=0pt,parsep=0pt,leftmargin=3.6em,label=(\arabic*)]

# 大小写罗马数字
\begin{enumerate}[topsep=0pt,itemsep=0pt,parsep=0pt,leftmargin=3.6em,label=\roman*>]
\begin{enumerate}[topsep=0pt,itemsep=0pt,parsep=0pt,leftmargin=3.6em,label=\Roman*>]

# 大小写字母
\begin{enumerate}[topsep=0pt,itemsep=0pt,parsep=0pt,leftmargin=3.6em,label=\alph*>]
\begin{enumerate}[topsep=0pt,itemsep=0pt,parsep=0pt,leftmargin=3.6em,label=\Alph*>]

# 特殊符号
\begin{enumerate}[topsep=0pt,itemsep=0pt,parsep=0pt,leftmargin=3.6em,label=\fnsymbol*>]
\end{lstlisting}

\begin{lstlisting}[language={python}]
\begin{enumerate}[topsep=0pt,itemsep=0pt,parsep=0pt,leftmargin=3.6em,label=\arabic*>]
    \item 编号
        \arabic{enumi}, \roman{enumi}, \Roman{enumi}, 
        \alph{enumi}, \Alph{enumi}, \fnsymbol{enumi}
    \item 编号
        \arabic{enumi}, \roman{enumi}, \Roman{enumi}, 
        \alph{enumi}, \Alph{enumi}, \fnsymbol{enumi}
    \item 编号
        \arabic{enumi}, \roman{enumi}, \Roman{enumi}, 
        \alph{enumi}, \Alph{enumi}, \fnsymbol{enumi}
\end{enumerate}
\end{lstlisting}

\begin{enumerate}[topsep=0pt,itemsep=0pt,parsep=0pt,leftmargin=3.6em,label=\arabic*>]
    \item 编号
        \arabic{enumi}, \roman{enumi}, \Roman{enumi}, \alph{enumi}, \Alph{enumi}, \fnsymbol{enumi}
    \item 编号
        \arabic{enumi}, \roman{enumi}, \Roman{enumi}, \alph{enumi}, \Alph{enumi}, \fnsymbol{enumi}
    \item 编号
        \arabic{enumi}, \roman{enumi}, \Roman{enumi}, \alph{enumi}, \Alph{enumi}, \fnsymbol{enumi}
\end{enumerate}


\paragraph{lstlisting}~{}

\begin{lstlisting}[language={python}]
`\verb!\begin{lstlisting}[language={python}]!`
xxx
`\verb=\end{lstlisting}=`

`\verb!\begin{lstlisting}[language={c++}]!`
`\verb!\begin{lstlisting}[language={sh}]!`
\end{lstlisting}

可以在 lstlisting 环境中使用 \textbackslash verb 来输入各种特殊的字符,比如 {\color{red}\ \verb=`$CMD_A`=}

\begin{lstlisting}[language={python}]
`\verb!\verb=`$CMD_A`=!` `{\color{red}->}` `\verb=`$CMD_A`=`
\end{lstlisting}

lstlisting 在序列中的项时这么使用:

\begin{lstlisting}[language={python}]
    ...
    `\verb!\item!` a item
`\verb!\begin!`{lstlisting}[language={sh}, `{\color{red}xleftmargin=0em}`]
...
`\verb!\end!`{lstlisting}
    `\verb!\item!` other item
    ...
\end{lstlisting}



\paragraph{verbatim}~{}

缩略写法:\textbackslash verb=\textcolor{red}{xxx}= : \textcolor{red}{xxx} 是要原格式表示的内容

%\begin{verbatim}
%    \usepackage[T1]{fontenc}            % font styling
%    \usepackage{lmodern, mathrsfs}
%\end{verbatim}

\begin{lstlisting}[language={python}]
\makeatletter
\renewcommand*{\verbatim@font}{}
\makeatother
\end{lstlisting}

如果需要设置 \textbackslash verb 环境中的颜色,可以直接在外部包上 \textbackslash color\{\} 来进行设置即可

如果还需要在  \textbackslash verb 环境中设置字体,可以以如下方式来使用:

\begin{lstlisting}[language={c}]
{\small\ttfamily\fontspec{Consolas}\verb!grep `-`in "\`-`\`-`" ./ `-`R!}
\end{lstlisting}

%\begin{lstlisting}[language={python}]
%\verb={\fontspec{Times New Roman} abcdefgxxxx}= 
%\end{lstlisting}


\subsubsection{tabular、tabularx、array}

table to latex code : \url{https://www.tablesgenerator.com/}

\url{https://github.com/krlmlr/Excel2LaTeX}

文本或数学模式都可以使用 tabular,数学模式还可以使用 array 环境(即包含数学符号的公式)

table 浮动表格中的几个命令设置如下:

\begin{lstlisting}[language={tex}]
...
\setlength{\abovecaptionskip}{1.0em} %
\setlength{\belowcaptionskip}{`-`1.3em}% 这两个必须得放在 caption 前面,否则不起作用
\caption{mxnet 的基本概念}
...
\end{lstlisting}

\paragraph{tabular}~{}

tabular 中的参数设置: 
\begin{enumerate}[topsep=0pt,itemsep=0pt,parsep=0pt,leftmargin=3.6em,label=\arabic*>]
    \item \textbackslash raggedleft:表示左边不对齐,\textcolor{red}{即右对齐}
    \item \textbackslash raggedright:表示右边不对齐,\textcolor{red}{即左对齐}
    \item \textbackslash centering:表示居中
    \item p\{width value\}:设置列框,使用了列宽后,需在列内容处使用上述的 对其方式
    \item \textbackslash arraystretch\{val\}:设置行间距,默认值为 1
        eg. \textbackslash renewcommand\textbackslash arraystretch\{2\} | 默认值的 2 倍
    \item \textbackslash multicolumn:合并不同列 \par
        \textbackslash multicolumn\{列数\}\{格式\}\{内容\} \par
        \textbackslash multicolumn\{2\}\{c|\}\{内容\} : {\color{red}c 表示居中,| 表示边框} \par
        \textbackslash multicolumn\{2\}\{l|\}\{内容\} : left \par
        \textbackslash multicolumn\{2\}\{r|\}\{内容\} : right \par
        \textcolor{DefinedColorGreen}{如果只是针对 1 列操作时,相当于是修改对齐、竖线格式}
    \item \textbackslash multirow:跨不同行,类似于合并不同行 | \textbackslash usepackage\{multirow\}\par
        \textbackslash multirow\{行数\}\{宽度\}\{内容\} \par
        \textbackslash multirow\{行数\}*\{内容\} | 宽度由输入内容来决定 \par
        需要结合 \textbackslash cline{start - end} 来使用 eg. \verb!\cline{2-5}!
\end{enumerate}

\begin{lstlisting}[language={tex}]
\begin{table}[htbp!]
    \centering
    \setlength{\abovecaptionskip}{0.5em}    % 这个设置在 tabular 前面的话的参数值
    \setlength{\belowcaptionskip}{`-`0.5em}
    \caption{xxx}
    \label{tab:xxx}
\begin{tabular}{|l|l|}
...
\end{lstlisting}

\begin{lstlisting}[language={tex}]
...
\end{tabular}
    \centering
    \setlength{\abovecaptionskip}{1.0em}    % 设置在 tabular 后面的话的参数值
    \setlength{\belowcaptionskip}{`-`1.3em}
    \caption{xxx}
    \label{tab:xxx}
\end{table}
\end{lstlisting}


\begin{center}
%\renewcommand\arraystretch{2}
\begin{tabular}{|p{8em}|c|l|p{4em}|p{4em}|}
    \hline
    \raggedleft{\textbackslash raggedleft} & abc & abc & abc & abc \\
    \hline
    \raggedright{\textbackslash raggedright} & abc & abc & \multicolumn{2}{c|}{abc} \\
    \hline
    \centering{\textbackslash centering} & abc & abc & abc & abc \\
    \hline
    default & abc & abc & abc & abc \\
    \hline
    \multirow{2}*{multirow} & abc & abc & abc & abc \\

    %
    % 这里表示只需要对 2-5 列画横线 , \hline 表示从第 1 列到最后 1 列都画横线
    %
    \cline{2-5}
                            & abc & abc & abc & abc \\
    \hline
    default & abc & abc & abc & abc \\
    \hline
    \centering{\multirow{2}*{multirow}} & abc & abc & abc & abc \\
    \cline{2-5}
                            & abc & abc & abc & abc \\
    \hline
    \raggedleft{\multirow{2}*{multirow}} & abc & abc & abc & abc \\
    \cline{2-5}
                            & abc & abc & abc & abc \\
    \hline
\end{tabular}
\end{center}

\paragraph{tabularx}~{}

tabularx 环境是固定宽度的表格,这里的固定宽度指\textcolor{red}{铺满页面宽度}

%
% Y 是重新定义的格式,\arraybackslash 表示命令恢复
%
\newcolumntype{C}{>{\centering\arraybackslash}X}
\newcolumntype{R}{>{\raggedleft\arraybackslash}X}
\begin{tabularx}{46em}{|c|X|X|R|C|}
    \hline
    数字 & 1 & 2 & 3 & 4 \\
    \hline
    英文 & one & two & three & four \\
    \hline
\end{tabularx}

\paragraph{longtable}~{}

\begin{lstlisting}[language={tex}]
\begin{longtable}{|l|m{40em}|}
    \hline
    aaa   & xxx \\
    \hline
    bbb   & xxx \\
    \hline
    ccc   & xxx # 这样可以实现换行

          1> xxx

          2> xxx \\
    \hline
\end{longtable}
\end{lstlisting}

\begin{longtable}{|l|m{40em}|}
    \hline
    aaa   & xxx \\
    \hline
    bbb   & xxx \\
    \hline
    ccc   & xxx

          1> xxx

          2> xxx \\
    \hline
\end{longtable}

\paragraph{array | tabu}~{}

\begin{enumerate}[topsep=0pt,itemsep=0pt,parsep=0pt,leftmargin=3.6em,label=\arabic*>]
    \item 列格式控制:\textbackslash usepackage\{array\}
    \item 行格式控制:\textbackslash usepackage\{tabu\}
\end{enumerate}

\subparagraph{array}~{}

array 新增的基本命令(\verb=\usepackage{array}=):
\begin{enumerate}[topsep=0pt,itemsep=0pt,parsep=0pt,leftmargin=3.6em,label=\arabic*>]
    \item \verb=m{宽}=:类似 p 格式,产生具有固定宽度的列,并可以自动换行,垂直方向居中对齐 \par
        这个 m 格式在 tabular 、longtable 中都是可以使用的
    \item \verb=b{宽}=:类似 p 格式,垂直方向与最后一行对齐
    \item \verb=>{xx}=:把 xx 插入后面一列的开头
    \item \verb=<{xx}=:把 xx 插入前面一列的末尾
    \item \verb=!{xx}=:把 xx 作为表格线处理,相当于使用了 \verb=@{xx}= 但左右两边会有额外的间距 \par
        eg. \verb=c!{$\Rightarrow$}=
\end{enumerate}

格式符 > 与 < 通常用来设置整列的格式,例如改变一列表格的字体或者使用数学模式

\begin{tabular}{>{\bfseries}c|>{\itshape}c>{$}c<{$}}
    \hline
    姓名 & \textnormal{得分} & \multicolumn{1}{c}{额外加分} \\
    \hline
    A & 85 & +7 \\
    B & 82 & 0 \\
    C & 70 & -2 \\
    \hline
\end{tabular}

%
% 这个可以自动居中
%
\begin{tabular}{|>{$}r<{$}|>{\setlength\parindent{2em}}m{35em}|>{\centering\arraybackslash}m{4em}|}
    \hline
    \pi & 希腊字母,希腊字母希腊字母希腊字母希腊字母希腊字母希腊字母希腊字母希腊字母 & 常用 \\
    \hline
    \pi & 希腊字母,希腊字母希腊字母希腊字母希腊字母希腊字母希腊字母希腊字母希腊字母 & 常用 \\
    \hline
\end{tabular}

array 宏包提供的\verb=\newcolumntype= 用于定义新的列格式的命令:
\begin{enumerate}[topsep=0pt,itemsep=0pt,parsep=0pt,leftmargin=3.6em,label=\arabic*>]
    \item \verb=\newcolumntype{M}{>{$}c<{$}}=
    \item \verb=\newcolumntype{P}[1]{>{\setlength\parindent{2em}}p{#1}}=
    \item \verb=\newcolumntype{C}[1]{>{\centering\arraybackslash}m{#1}}= \par
        eg. \verb=\begin{tabular}{|M|P{15em}|C{4em}|}=
\end{enumerate}

\subparagraph{tabu}~{}

tabu 宏包提供 \textbackslash rowfont 命令来设置行的格式

\begin{tabu}{ccc}
    \hline
    \rowfont{\bfseries} 姓名 & 得分 & \multicolumn{1}{c}{额外加分} \\
    \hline
    \rowfont{\color{blue}} A & 85 & +7 \\
    \rowfont{\color{red}} B & 82 & 0 \\
    \rowfont{\color{green}} C & 70 & -2 \\
    \hline
\end{tabu}

\paragraph{threeparttable}~{}

\begin{enumerate}[topsep=0pt,itemsep=0pt,parsep=0pt,leftmargin=3.6em,label=\arabic*>]
    \item \verb=usepackage{threeparttable}=
    \item \verb=\tnote{val}=:使用这个来定义表格中的角标 eg.\par
        \textcolor{red}{\textbackslash tnote\{2\}} \par
        \textcolor{red}{\textbackslash tnote\{a\}} \par
        \textcolor{red}{\textbackslash tnote\{I\}} \par
        \textcolor{red}{\textbackslash tnote\{*\}}
\end{enumerate}

\begin{center}
\begin{threeparttable} 
    \centering 
    \caption{threeparttable example} 
    \label{tbl:threeparttable example} 
    \begin{tabular}{l|cc|cc}
        \hline 
        $F_{1}$ &$3.70e-03$ &$4.20e-03$\tnote{2} &$ \textbf{3.70e-03}$ &$3.90e-03$\\
        $F_{2}$ &$3.60e-03$\tnote{a} &$1.55e-02$ &$\textbf{6.70e-03}$ &$8.50e-03$\\ 
        \hline
    \end{tabular} 
    \begin{tablenotes} 
    \item[1] The bolder ones mean better. 
    \item[2] my note
    \item[a] my note a
    \end{tablenotes} 
\end{threeparttable} 
\end{center}



\subsubsection{set color}

\begin{enumerate}[topsep=0pt,itemsep=0pt,parsep=0pt,leftmargin=3.6em,label=\arabic*>]
    \item \{\textbackslash color\{blue\}\{蓝色字\}\}:{\color{blue}{蓝色字}}
    \item \textbackslash textcolor\{red\}\{红色字体\}:\textcolor{red}{红色字体}
    \item \textbackslash colorbox\{yellow\}\{黄色盒子\}:\colorbox{yellow}{黄色盒子}
    \vspace{-0.4em}
    \item \textbackslash fcolorbox\{black\}\{green\}\{黑框绿色背景盒子\}:\fcolorbox{black}{green}{黑框绿色背景盒子}
\end{enumerate}

\begin{longtable}{cp{5em}|cp{5em}|cp{5em}|cp{5em}|cp{5em}|cp{5em}}
    \hline
    \colorbox{black}{\textcolor{black}{color}} & \multicolumn{1}{c|}{black} &
    \colorbox{gray}{\textcolor{gray}{color}} & \multicolumn{1}{c|}{gray} &
    \colorbox{olive}{\textcolor{olive}{color}} & \multicolumn{1}{c|}{olive} &
    \colorbox{blue}{\textcolor{blue}{color}} & \multicolumn{1}{c|}{blue} &
    \colorbox{green}{\textcolor{green}{color}} & \multicolumn{1}{c|}{green} &
    \colorbox{teal}{\textcolor{teal}{color}} & \multicolumn{1}{c}{teal} \\
    \hline
    \colorbox{violet}{\textcolor{violet}{color}} & \multicolumn{1}{c|}{violet} &
    \colorbox{brown}{\textcolor{brown}{color}} & \multicolumn{1}{c|}{brown} &
    \colorbox{lightgray}{\textcolor{lightgray}{color}} & \multicolumn{1}{c|}{lightgray} &
    \colorbox{pink}{\textcolor{pink}{color}} & \multicolumn{1}{c|}{pink} &
    \colorbox{white}{\textcolor{white}{color}} & \multicolumn{1}{c|}{white} &
    \colorbox{lime}{\textcolor{lime}{color}} & \multicolumn{1}{c}{lime} \\
    \hline
    \colorbox{purple}{\textcolor{purple}{color}} & \multicolumn{1}{c|}{purple} &
    \colorbox{yellow}{\textcolor{yellow}{color}} & \multicolumn{1}{c|}{yellow} &
    \colorbox{darkgray}{\textcolor{darkgray}{color}} & \multicolumn{1}{c|}{darkgray} &
    \colorbox{magenta}{\textcolor{magenta}{color}} & \multicolumn{1}{c|}{magenta} &
    \colorbox{red}{\textcolor{red}{color}} & \multicolumn{1}{c|}{red} &
    \colorbox{orange}{\textcolor{orange}{color}} & \multicolumn{1}{c}{orange} \\
    \hline
    \colorbox{cyan}{\textcolor{cyan}{color}} & \multicolumn{1}{c|}{cyan} \\
    \hline
\end{longtable}

\begin{longtable}{cp{6em}|cp{6em}|cp{6em}|cp{6em}|cp{6em}}
    \hline
    \colorbox{Yellow}{\textcolor{Yellow}{color}} & \multicolumn{1}{c|}{Yellow} &
    \colorbox{GreenYellow}{\textcolor{GreenYellow}{color}} & \multicolumn{1}{c|}{GreenYellow} &
    \colorbox{RubineRed}{\textcolor{RubineRed}{color}} & \multicolumn{1}{c|}{RubineRed} &
    \colorbox{RoyalPurple}{\textcolor{RoyalPurple}{color}} & \multicolumn{1}{c|}{RoyalPurple} &
    \colorbox{Emerald}{\textcolor{Emerald}{color}} & \multicolumn{1}{c}{Emerald} \\
    \hline
    \colorbox{WildStrawberry}{\textcolor{WildStrawberry}{color}} & \multicolumn{1}{c|}{WildStrawberry} &
    \colorbox{BlueViolet}{\textcolor{BlueViolet}{color}} & \multicolumn{1}{c|}{BlueViolet} &
    \colorbox{Emerald}{\textcolor{Emerald}{color}} & \multicolumn{1}{c}{Emerald} \\
    \hline
\end{longtable}

\begin{enumerate}[topsep=0pt,itemsep=0pt,parsep=0pt,leftmargin=3.6em,label=\arabic*>]
    \item \textbackslash columncolor:eg. \verb!\begin{tabular}{>{\columncolor{gray}}c >{\columncolor{red}}c}!
    \item \textbackslash rowcolor:eg. \verb!\rowcolor{red} A & B & C \\!
    \item \verb=\setlength{\arrayrulewidth}{0.19mm}=:设置表格横线与竖线的粗细,实验下来设置 0.19 mm 最好,
        设置颜色时不会把线给遮挡了,而且线条显示也均匀 \par
        可以把这个参数设置到\textcolor{red}{导言区}
    \item \verb=\arrayrulecolor{green}= 用来设置表格线条为\textcolor{green}{绿色} \par
        后面的表格如果想要恢复成黑色,需要再重新设置一下
\end{enumerate}


\arrayrulecolor{green}
\begin{tabular}{|l|c|r|}
    \arrayrulecolor{black}\hline
    United Kingdom & London & Thames\\
    \arrayrulecolor{blue}\hline
    France & Paris & Seine \\
    \arrayrulecolor{black}\cline{1-1}
    \arrayrulecolor{red}\cline{2-3}
    Russia & Moscow & Moskva \\ \hline
\end{tabular}


\arrayrulecolor{black}
\begin{tabular}{|>{\columncolor{lime}[6pt][6pt]}c|l|l|r|c|}
    \hline
    数字 & 1 & 2 & 3 & \cellcolor{red}4 \\
    \hline
    英文 & one & two & three & four \\
    \hline
    \rowcolor{lightgray}[\tabcolsep][\tabcolsep] 汉文 & 一 & 二 & 三 & 四 \\
    \hline
\end{tabular}




\subsubsection{define command and environment}

\paragraph{define command}~{}

使用 \verb=\newcommand<命令>[<参数个数>][<首参数默认值>]{<具体定义>}= 来定义一个新的命令
\begin{enumerate}[topsep=0pt,itemsep=0pt,parsep=0pt,leftmargin=3.6em,label=\arabic*>]
    \item 无参数定义:\par
        \verb=\newcommand<命令>{<具体定义>}= 
    \item 有参数定义:\par
        参数编号使用\verb=#1、#2、#3、...、#9=
\end{enumerate}

以下是一些定义实例:

\newcommand\PRCa{\emph{China}}
\newcommand\PRCb{\textcolor{red}{\emph{China}}}
\newcommand\PRCc[1]{\textcolor{blue}{\textcolor{red}{#1} country is \emph{China}}}
\newcommand\PRCd[1][my]{\textcolor{blue}{\textcolor{red}{#1} country is \emph{China}}}

\begin{enumerate}[topsep=0pt,itemsep=0pt,parsep=0pt,leftmargin=3.6em,label=\arabic*>]
    \item \verb=\newcommand\PRCa{\emph{China}}= \par
        \verb=\PRCa= $\Rightarrow$ \PRCa
    \item \verb=\newcommand\PRCb{\textcolor{red}{\emph{China}}}= \par
        \verb=\PRCb= $\Rightarrow$ \PRCb
    \item \verb=\newcommand\PRCc[1]{\textcolor{blue}{\textcolor{red}{#1} country is \emph{China}}}= \par
        \verb=\PRCc{my}= $\Rightarrow$ \PRCc{my}
    \item \verb=\newcommand\PRCd[1][my]{\textcolor{blue}{\textcolor{red}{#1} country is \emph{China}}}= \par
        \verb=\PRCd= $\Rightarrow$ \PRCd \par
        \verb=\PRCd[your]= $\Rightarrow$ \PRCd[your] \par
\end{enumerate}

\paragraph{renew command}~{}

使用 \verb=\renewcommand<命令>[<参数个数>][<首参数默认值>]{<具体定义>}= 来重定义命令

\paragraph{new/renew environment}~{}



\subsubsection{font set}

在 linux 中 latex 字体如果想使用 win 的字体,需要先把 win 的字体拷贝到 linux 中,具体操作可以参考 
\ref{sec:use-win-font-in-ubuntu}

在导言区设置如下,就可以使用 win 的字体:
\begin{lstlisting}[language={matlab}]
\usepackage{fontspec}
%\setmainfont{Times New Roman}   % 这些字体可以用 fc`-`list 中查看名字
\setmainfont{Calibri}
\setsansfont{Verdana}
%\setmonofont{Courier New}
%\setmonofont{Consolas}
\end{lstlisting}

\begin{lstlisting}[language={sh}]
minphone@mcp:~$fc`-`list
/usr/share/fonts/windows_fonts/BROADW.TTF: `\textcolor{red}{Broadway}`:style=Regular,obyčejné
/usr/share/fonts/windows_fonts/consola.ttf: `\textcolor{red}{Consolas}`:style=Regular
/usr/share/fonts/windows_fonts/constanb.ttf: `\textcolor{red}{Constantia}`:style=Bold
/usr/share/fonts/windows_fonts/ONYX.TTF: `\textcolor{red}{Onyx}`:style=Regular,obyčejné
\end{lstlisting}

{\color{red}按以下方式来使用字体:}
\begin{enumerate}[topsep=0pt,itemsep=0pt,parsep=0pt,leftmargin=3.6em,label=\arabic*>]
    \item \{\textbackslash fontspec\{Microsoft YaHei Light\} 微软雅黑 Light\}:
        {\fontspec{Microsoft YaHei Light}微软雅黑 Light}  % for win-font
    \item \{\textbackslash fontspec\{YouYuan\} 幼圆\}:{\fontspec{YouYuan}幼圆}  % for win-font
    \item \{\textbackslash fontspec\{Comic Sans MS\} Comic Sans MS\}:{\fontspec{Comic Sans MS}Comic Sans MS}  % for win-font
    \item \{\textbackslash small\textbackslash fontspec\{Comic Sans MS\} Comic Sans MS\}
        {\small\fontspec{Comic Sans MS}Comic Sans MS}  % for win-font

    \item \{\textbackslash fontsize\{8pt\}\{0\}\textbackslash fontspec\{Comic Sans MS\} Comic Sans MS\}
        {\fontsize{8pt}{0}\fontspec{Comic Sans MS}Comic Sans MS}  % for win-font

    \item \{\textbackslash small\textbackslash fontspec\{Consolas\} Consolas\}:
        {\small\fontspec{Consolas}Consolas}  % for win-font
        | small {\color{teal}就是小五号}
    \item \verb=\textup{upright shape}= : \textup{upright shape}
    \item \verb=\textit{italic shape}= : \textit{italic shape}
    \item \verb=\textsl{slanted shape}= : \textsl{slanted shape}
    \item \verb=\textsc{small capitals shape}= : \textsc{small capticals shape}
    \item \verb=\textbf{apple}= : \textbf{apple}
\end{enumerate}

\paragraph{set font size}~{}

\begin{enumerate}[topsep=0pt,itemsep=0pt,parsep=0pt,leftmargin=3.6em,label=\arabic*>]
    \item \verb=\tiny{Text}= : {\tiny{Text}}
    \item \verb=\small{Text}= : {\small{Text}}
    \item \verb=\normalsize{Text}= : {\normalsize{Text}}
    \item \verb=\large{Text}= : {\large{Text}}
    \item \verb=\huge{Text}= : {\huge{Text}}
    \item \verb={\fontsize{8pt}{0}Text}= : {\fontsize{8pt}{0}Text}
\end{enumerate}

各个字号的榜值 : \url{https://blog.csdn.net/weixin_39679367/article/details/115794548}

\paragraph{font style of math}~{}

\begin{enumerate}[topsep=0pt,itemsep=0pt,parsep=0pt,leftmargin=3.6em,label=\arabic*>]
    \item \verb=$\mathnormal{ABCDEF, abcdef, 123456}$= : {$\mathnormal{ABCDEF, abcdef, 123456}$}
    \item \verb=$\mathnormal{123456}$= : {$\mathnormal{123456}$}
    \item \verb=$\mathrm{ABCDEF, abcdef, 123456}$= : {$\mathrm{ABCDEF, abcdef, 123456}$}
    \item \verb=$\mathit{ABCDEF, abcdef, 123456}$= : {$\mathit{ABCDEF, abcdef, 123456}$}
    \item \verb=$\mathbf{ABCDEF, abcdef, 123456}$= : {$\mathbf{ABCDEF, abcdef, 123456}$}
    \item \verb=$\mathsf{ABCDEF, abcdef, 123456}$= : {$\mathsf{ABCDEF, abcdef, 123456}$}
    \item \verb=$\mathtt{ABCDEF, abcdef, 123456}$= : {$\mathtt{ABCDEF, abcdef, 123456}$}
    \item \verb=$\mathcal{ABCDEF}$= : {$\mathcal{ABCDEF}$}
\end{enumerate}



\newpage

\subsubsection{formula | \$ xxx \$}

\begin{enumerate}[topsep=0pt,itemsep=0pt,parsep=0pt,leftmargin=3.6em,label=\arabic*>]
    \item {\color{teal}\verb!$a \dots b$!} => $a \dots b$
    \item {\color{teal}\verb!$a \vdots b$!} => $a \vdots b$
    \item {\color{teal}\verb!$a \ddots b$!} => $a \ddots b$
    \item {\color{teal}\verb!$\{aaa\}$!} => $\{aaa\}$
    \item {\color{teal}\verb!$[aaa]$!} => $[aaa]$
    %\item {\color{teal}\verb!$a \iddots b$!} => $a \iddots b$
\end{enumerate}

\paragraph{上标/下标}~{}

\begin{enumerate}[topsep=0pt,itemsep=0pt,parsep=0pt,leftmargin=3.6em,label=\arabic*>]
    \item {\color{teal}\verb!$x_{1}$!} => $x_{1}$
    \item {\color{teal}\verb!$x^{2}$!} => $x^{2}$
    \item {\color{teal}\verb!$A_{ij} = 2^{i+j}$!} => $A_{ij} = 2^{i+j}$
    \item {\color{teal}\verb!$A\prime$!} => $A\prime$
    \item {\color{teal}\verb!$A'$!} => $A'$
    \item {\color{teal}\verb!$A''$!} => $A''$
    \item {\color{teal}\verb!$A = 90^\circ$!} => $A = 90^\circ$
    \item {\color{teal}\verb!$\max_n f(n)$!} => $\max_n f(n)$
    \item {\color{teal}\verb!$A_m{}^n$!} => $A_m{}^n$
    \item {\color{teal}\verb!$\int_0^1 f(t)$!} => $\int_0^1 f(t)$
    \item {\color{teal}\verb!$\iint_0^1 f(t)$!} => $\iint_0^1 f(t)$
    \item {\color{teal}\verb!$\int_0^1 f(t)dt$!} => $\int_0^1 f(t)dt$
    \item {\color{teal}\verb!$\iint_0^1 g(x, y)dxdy$!} => $\iint_0^1 g(x, y)dxdy$
    \item {\color{teal}\verb!$\int_0^1 f(t)\mathrm{d}t$!} => $\int_0^1 f(t)\mathrm{d}t$
    \item {\color{teal}\verb!$\vec x = \overrightarrow{AB}$!} => $\vec x = \overrightarrow{AB}$

    \item {\color{teal}\verb!$\overset{*}{X}$!} => $\overset{*}{X}$ {\color{red}|} 
          {\color{teal}\verb!$\underset{*}{X}$!} => $\underset{*}{X}$ {\color{red}|} 
          {\color{teal}\verb!$\underset{m}{X}$!} => $\underset{m}{X}$
    \item {\color{teal}\verb!\[\max_n f(n)\]!} => \[\max_n f(n)\]
\end{enumerate}

\paragraph{分式}~{}

\begin{enumerate}[topsep=0pt,itemsep=0pt,parsep=0pt,leftmargin=3.6em,label=\arabic*>]
    \item {\color{teal}\verb!$\frac 12 + \frac 1a = \frac{2+a}{2a}$!} => $\frac 12 + \frac 1a = \frac{2+a}{2a}$
    \item {\color{teal}\verb!\[\tfrac 12 f(x) = \frac{1}{\dfrac 1a + \dfrac 1b + c}\]!} 
        => \[\tfrac 12 f(x) = \frac{1}{\dfrac 1a + \dfrac 1b + c}\] \par
        {\color{teal}\verb!\dfrac!} : display style, looks smaller \par
        {\color{teal}\verb!\tfrac!} : text style, looks normal
\end{enumerate}

\paragraph{根式}~{}

\begin{enumerate}[topsep=0pt,itemsep=0pt,parsep=0pt,leftmargin=3.6em,label=\arabic*>]
    \item {\color{teal}\verb!$\sqrt 4 = \sqrt[3]{8} = 2$!} => $\sqrt 4 = \sqrt[3]{8} = 2$
\end{enumerate}

\paragraph{inline formula : 行内公式}~{}

行内公式是嵌在文本中的,其输入可以有三种方式:

\begin{enumerate}[topsep=0pt,itemsep=0pt,parsep=0pt,leftmargin=3.6em,label=\arabic*>]
    \item \$ \emph{formula} \$ \par
        {\color{DefinedColorGreen}\$ a + b > c \$} 
        $\Rightarrow$ % =>
        {\color{DefinedColorRed}$a + b > c$}
    \item \textbackslash( \emph{formula} \textbackslash) \par
        {\color{DefinedColorGreen}\textbackslash( a + b > c \textbackslash)} 
        $\Rightarrow$ % =>
        {\color{DefinedColorRed}\(a + b > c\)}
    \item \textbackslash begin\{math\} \emph{formula} \textbackslash end\{math\} \par
        {\color{DefinedColorGreen} \textbackslash begin\{math\} a + b > c \textbackslash end\{math\}} 
        $\Rightarrow$ % =>
        {\color{DefinedColorRed}\begin{math} a + b > c \end{math} }
\end{enumerate}

\paragraph{line formula : 行间公式}~{}

行间公式是单独占据一行居中显示,其输入也有三种:
\begin{enumerate}[topsep=0pt,itemsep=0pt,parsep=0pt,leftmargin=3.6em,label=\arabic*>]
    \item {\color{teal}\verb!\[\max_n f(n)\]!} => \[\max_n f(n)\]
\end{enumerate}

\paragraph{matrix}~{}

matrix 环境共有以下几种:

\begin{enumerate}[topsep=0pt,itemsep=0pt,parsep=0pt,leftmargin=3.6em,label=\arabic*>]
    \item matrix
    \item bmatrix
    \item vmatrix
    \item pmatrix
    \item Bmatrix
    \item Vmatrix
\end{enumerate}

\[ A_{\mathrm{matrix}} = 
\begin{matrix}
    a_{11} & a_{12} & a_{13} \\
    a_{21} & a_{22} & a_{23} \\
    a_{31} & a_{32} & a_{33}
\end{matrix}
\]

\[ A_{\mathrm{bmatrix}} = 
\begin{bmatrix}
    a_{11} & a_{12} & a_{13} \\
    a_{21} & a_{22} & a_{23} \\
    a_{31} & a_{32} & a_{33}
\end{bmatrix}
\]

\[ A_{\mathrm{vmatrix}} = 
\begin{vmatrix}
    a_{11} & a_{12} & a_{13} \\
    a_{21} & a_{22} & a_{23} \\
    a_{31} & a_{32} & a_{33}
\end{vmatrix}
\]

\[ A_{\mathrm{pmatrix}} = 
\begin{pmatrix}
    a_{11} & a_{12} & a_{13} \\
    a_{21} & a_{22} & a_{23} \\
    a_{31} & a_{32} & a_{33}
\end{pmatrix}
\]

\[ A_{\mathrm{Vmatrix}} = 
\begin{Vmatrix}
    a_{11} & a_{12} & a_{13} \\
    a_{21} & a_{22} & a_{23} \\
    a_{31} & a_{32} & a_{33}
\end{Vmatrix}
\]

%%%%%%%%%%%%%%%%%%%%%%%%%%%%%%%%%%%%%%%%%%

\begin{lstlisting}[language={python}]
\[ A_{\mathrm{matrix}} = 
\begin{matrix}
    a_{11} & a_{12} & a_{13} \\
    a_{21} & a_{22} & a_{23} \\
    a_{31} & a_{32} & a_{33}
\end{matrix}
\]
\end{lstlisting}

\begin{lstlisting}[language={python}]
\[ A_{\mathrm{bmatrix}} = 
\begin{bmatrix}
    a_{11} & a_{12} & a_{13} \\
    a_{21} & a_{22} & a_{23} \\
    a_{31} & a_{32} & a_{33}
\end{bmatrix}
\]
\end{lstlisting}

\begin{lstlisting}[language={python}]
\[ A_{\mathrm{vmatrix}} = 
\begin{vmatrix}
    a_{11} & a_{12} & a_{13} \\
    a_{21} & a_{22} & a_{23} \\
    a_{31} & a_{32} & a_{33}
\end{vmatrix}
\]
\end{lstlisting}

\begin{lstlisting}[language={python}]
\[ A_{\mathrm{pmatrix}} = 
\begin{pmatrix}
    a_{11} & a_{12} & a_{13} \\
    a_{21} & a_{22} & a_{23} \\
    a_{31} & a_{32} & a_{33}
\end{pmatrix}
\]
\end{lstlisting}

\begin{lstlisting}[language={python}]
\[ A_{\mathrm{Vmatrix}} = 
\begin{Vmatrix}
    a_{11} & a_{12} & a_{13} \\
    a_{21} & a_{22} & a_{23} \\
    a_{31} & a_{32} & a_{33}
\end{Vmatrix}
\]
\end{lstlisting}



\newpage

\subsubsection{special mark}

\paragraph{mark key point}~{}

\begin{enumerate}[topsep=0pt,itemsep=0pt,parsep=0pt,leftmargin=3.6em,label=\arabic*>]
    \item \verb!\underline{xxx}! \par
        \verb!\underline{abc}! \ColorStr{red}{=>} \underline{abc}
    \item \verb!\emph{xxx}! \par
        \verb!\emph{abc}! \ColorStr{red}{=>} \emph{abc}
    \item \verb!\uuline{xxx}! \par
        \verb!\uuline{abc}! \ColorStr{red}{=>} \uuline{abc}
    \item \verb!\uwave{xxx}! \par
        \verb!\uwave{abc}! \ColorStr{red}{=>} \uwave{abc}
    \item \verb!\sout{xxx}! \par
        \verb!\sout{abc}! \ColorStr{red}{=>} \sout{abc}
    \item \verb!\xout{xxx}! \par
        \verb!\xout{abc}! \ColorStr{red}{=>} \xout{abc}
    \item \verb!\dashuline{xxx}! \par
        \verb!\dashuline{abc}! \ColorStr{red}{=>} \dashuline{abc}
    \item \verb!\dotuline{xxx}! \par
        \verb!\dotuline{abc}! \ColorStr{red}{=>} \dotuline{abc}
\end{enumerate}

\paragraph{todo list}~{}

\begin{enumerate}[topsep=0pt,itemsep=0pt,parsep=0pt,leftmargin=3.6em,label=\arabic*>]
    \item \verb!$\square$! \ColorStr{red}{=>} $\square$
    \item \verb!$\boxtimes$! \ColorStr{red}{=>} $\boxtimes$
    \item \verb!$\surd$! \ColorStr{red}{=>} $\surd$
\end{enumerate}

\begin{enumerate}[topsep=0pt,itemsep=0pt,parsep=0pt,leftmargin=3.6em,label=\arabic*>]
    \item $\square$《笔记本》- 笔记本
    \item $\surd$《\sout{笔记本}》- 笔记本
    \item $\boxtimes$\sout{《笔记本》- 笔记本} $\surd$\ColorStr{red}{(done)}
    \item $\square$《笔记本》- 笔记本
\end{enumerate}

\paragraph{roman numerals}~{}

\begin{lstlisting}[language={python}]
I       `-`> 1                          VI      `-`> 6
II      `-`> 2                          VII     `-`> 7
III     `-`> 3                          VIII    `-`> 8
IV      `-`> 4                          IX      `-`> 9
V       `-`> 5                          X       `-`> 10
\end{lstlisting}





\newpage

\subsubsection{special symbols}

\url{https://www.latexlive.com/help#d13}

\url{http://mohu.org/info/symbols/symbols.htm}

\url{https://www.caam.rice.edu/~heinken/latex/symbols.pdf}

特殊符号都是在 \$ xxx \$ 中输入的,或者在输入公式的其他方式中输入,{\color{DefinedColorRed}eg. \$\textbackslash 
bar\{x\}\$}

\newcommand\PrintCode[2]{\textcolor{DefinedColorGreen}{\textbackslash #1\{#2\}}}
\newcommand\PrintSymbol[1]{\textcolor{red}{$#1$}}
\newcommand\SpecCharPair[3]{\PrintSymbol{#1} & \PrintCode{#2}{#3}}

\paragraph{math related}~{}

\begin{longtable}{|c|m{10em}|c|m{10em}|c|m{10em}|}
    \hline
    \SpecCharPair{\sin\theta}{sin}{\textbackslash theta} &
    \SpecCharPair{\cos{\theta}}{cos}{\textbackslash theta} &
    \SpecCharPair{\cosh{h}}{cosh}{h} \\
    \hline
    \SpecCharPair{\sinh g}{sinh}{g} &
    \SpecCharPair{\min{L}}{min}{L} &
    \SpecCharPair{\max{H}}{max}{H} \\
    \hline
    \SpecCharPair{\sup{t}}{sup}{t} &
    \SpecCharPair{\ker{x}}{ker}{x} &
    \SpecCharPair{\Pr{x}}{Pr}{x} \\
    \hline
    \SpecCharPair{\arctan{\frac{L}{T}}}{arctan}{\textbackslash frac\{L\}\{T\}} &
    \SpecCharPair{\arcsin\frac{L}{r}}{arcsin}{\textbackslash frac\{L\}\{r\}} &
    \SpecCharPair{\arccos{\frac{T}{r}}}{arccos}{\textbackslash frac\{T\}\{r\}} \\
    \hline
    \SpecCharPair{\operatorname{argch}l}{operatorname}{\{argch\}l} &
    \SpecCharPair{\operatorname{argsh}k}{operatorname}{\{argsh\}k} &
    \SpecCharPair{\operatorname{th}i}{operatorname}{\{th\}i} \\
    \hline
    \SpecCharPair{\operatorname{ch}h}{operatorname}{\{ch\}h} &
    \SpecCharPair{\operatorname{argth}m}{operatorname}{\{argth\}m} &
    \SpecCharPair{\arg{x}}{arg}{x} \\
    \hline
    \SpecCharPair{\tan{theta}}{tan}{theta} &
    \SpecCharPair{\tanh{i}}{tanh}{i} &
    \SpecCharPair{\limsup{S}}{limsup}{S} \\
    \hline
    \SpecCharPair{\liminf{I}}{liminf}{I} &
    \SpecCharPair{\lg{X}}{lg}{X} &
    \SpecCharPair{\inf{s}}{inf}{s} \\
    \hline
    \SpecCharPair{\exp{\!t}}{exp}{\textbackslash !t} &
    \SpecCharPair{\ln{X}}{ln}{X} &
    \SpecCharPair{\log{X}}{log}{X} \\
    \hline
    \SpecCharPair{\deg{x}}{deg}{x} &
    \SpecCharPair{\det{x}}{det}{x} &
    \SpecCharPair{\dim{x}}{dim}{x} \\
    \hline
    \SpecCharPair{\log{_\alpha{x}}}{log}{\_\textbackslash alpha\{x\}} &
    \SpecCharPair{\hom{x}}{hom}{x} &
    \SpecCharPair{\lim{_{t\to n}T}}{lim}{\_\{t \textbackslash to n\}T} \\
    \hline
\end{longtable}

\begin{longtable}{|c|m{9em}|c|m{9em}|c|m{9em}|c|m{9em}|}
    \hline
    \SpecCharPair{\nabla}{nabla}{} &
    \SpecCharPair{\partial{x}}{partial}{x} &
    \SpecCharPair{\mathrm{d}{x}}{mathrm\{d\}}{x} \\
    \hline
\end{longtable}

\paragraph{relation symbols}~{}
\paragraph{binary operations}~{}
\paragraph{set and/or logic notation}~{}
\paragraph{delimiters}~{}
\paragraph{greek letters}~{}


\begin{longtable}{|c|m{9em}|c|m{9em}|c|m{9em}|c|m{9em}|}
    \hline
    \SpecCharPair{\alpha}{}{alpha} &
    \SpecCharPair{\beta}{}{beta} &
    \SpecCharPair{\gamma}{}{gamma} &
    \SpecCharPair{\delta}{}{delta} \\
    \hline
    \SpecCharPair{\epsilon}{}{epsilon} &
    \SpecCharPair{\varepsilon}{}{varepsilon} &
    \SpecCharPair{\zeta}{}{zeta} &
    \SpecCharPair{\eta}{}{eta} \\
    \hline
    \SpecCharPair{\theta}{}{theta} &
    \SpecCharPair{\vartheta}{}{vartheta} &
    \SpecCharPair{\iota}{}{iota} &
    \SpecCharPair{\kappa}{}{kappa} \\
    \hline
    \SpecCharPair{\lambda}{}{lambda} &
    \SpecCharPair{\mu}{}{mu} &
    \SpecCharPair{\nu}{}{nu} &
    \SpecCharPair{\xi}{}{xi} \\
    \hline
    \PrintSymbol{o} & \textcolor{DefinedColorGreen}{o} &
    \SpecCharPair{\pi}{}{pi} &
    \SpecCharPair{\varpi}{}{varpi} &
    \SpecCharPair{\rho}{}{rho} \\
    \hline
    \SpecCharPair{\varrho}{}{varrho} &
    \SpecCharPair{\sigma}{}{sigma} &
    \SpecCharPair{\varsigma}{}{varsigma} &
    \SpecCharPair{\tau}{}{tau} \\
    \hline
    \SpecCharPair{\upsilon}{}{upsilon} &
    \SpecCharPair{\phi}{}{phi} &
    \SpecCharPair{\varphi}{}{varphi} &
    \SpecCharPair{\chi}{}{chi} \\
    \hline
    \SpecCharPair{\psi}{}{psi} &
    \SpecCharPair{\omega}{}{omega} &
    \SpecCharPair{\Gamma}{}{Gamma} &
    \SpecCharPair{\Delta}{}{Delta} \\
    \hline
    \SpecCharPair{\Theta}{}{Theta} &
    \SpecCharPair{\Lambda}{}{Lambda} &
    \SpecCharPair{\Xi}{}{Xi} &
    \SpecCharPair{\Pi}{}{Pi} \\
    \hline
    \SpecCharPair{\Sigma}{}{Sigma} &
    \SpecCharPair{\Upsilon}{}{Upsilon} &
    \SpecCharPair{\Phi}{}{Phi} &
    \SpecCharPair{\Psi}{}{Psi} \\
    \hline
    \SpecCharPair{\Omega}{}{Omega} \\
    \cline{1-2}
\end{longtable}

\paragraph{other symbols}~{}

\begin{longtable}{|c|m{9em}|c|m{9em}|c|m{9em}|c|m{9em}|}
    \hline
    \SpecCharPair{\bar{x}}{bar}{x} &
    \SpecCharPair{\acute{\eta}}{acute}{\textbackslash eta} &
    \SpecCharPair{\grave{\eta}}{grave}{\textbackslash eta} &
    \SpecCharPair{\check{\alpha}}{check}{\textbackslash alpha} \\
    \hline
    \SpecCharPair{\breve{a}}{breve}{a} &
    \SpecCharPair{\dot{x}}{dot}{x} &
    \SpecCharPair{\hat{\alpha}}{hat}{\textbackslash alpha} &
    \SpecCharPair{\ddot{y}}{ddot}{y} \\
    \hline
    \SpecCharPair{\tilde{\iota}}{tilde}{\textbackslash iota} \\
    \cline{1-2}
\end{longtable}

\paragraph{trigonometric functions}~{}



\subsubsection{shaded 设置段落阴影}

使用 framed package 的 shaded 环境来进行设置,这个阴影环境支持跨页,颜色设置在导言区 \textcolor{red}{eg. \textbackslash 
colorlet\{shadecolor\}\{gray!20\}};并使用 \textcolor{red}{eg. \textbackslash vspace\{-0.8em\}} 来进行前后间距设置

\begin{lstlisting}[language={sh}]
\vspace{`-`0.8em}
\begin{shaded}
a b c d ... 等具体内容
\end{shaded}
\vspace{`-`0.8em}
\end{lstlisting}

\vspace{-0.8em}
\begin{shaded}
在实际的网络中存在如下图所示的情况,因此实际考虑有 “quantize-op” 与 relu 或者 conv 时,
是与 “quantize-op” 之前的 relu 融合还是与 “quantize-op” 之后的 conv 融合的问题
\end{shaded}
\vspace{-0.8em}

\subsubsection{footnote}

\begin{lstlisting}[language={python}]
AAA \footnote{AAA 的注脚内容}
\end{lstlisting}

\subsubsection{embeded image}

\begin{lstlisting}[language={sh}]
\begin{figure}[htbp!]
\setlength{\belowcaptionskip}{`-`1.3em}
\centering
\includegraphics[scale=0.68]{`{\color{red}images/aaa.jpg}`}
\caption{aaa}
\label{fig:aaa}
\end{figure}
\end{lstlisting}


\subsubsection{section}

章节标题的编码方式

\begin{lstlisting}[language={python}]
\chapter{name}
\section{name}
\subsection{name}
\subsubsection{name}

\paragraph{name}~{}
\subparagraph{name}~{}

\subparagraph*{name}~{} # 添加一个 '*' 不生成标题 
\end{lstlisting}






\subsection{other}

%%\subsubsection{chen shuo config}
%%这个是陈硕的 latex 常用设置,其中的反斜杠使用正斜杠替代,这样方便显示
%%
%%https://blog.csdn.net/solstice/article/details/638
%%
%%\paragraph{段首缩进}~{}
%%
%%中文习惯在段首缩进两格,在LaTeX中,/parindent 表示段首缩进的长度,我们将它设置为当前字号的两个
%%大写字母M的宽度,大约正好是两个汉字的宽度:{\color{DefinedColorRed}/setlength\{/parindent\}\{2em\}}
%%
%%LaTeX 默认每节的第一段的段首不缩进,这不符合中文排版习惯。我们希望正文的每一段都要缩进,使用
%%indentfirst宏包就可办到:{\color{DefinedColorRed}/usepackage\{indentfirst\}}
%%
%%\paragraph{段距行距}~{}
%%
%%LaTeX 用/baselineskip表示当前的行距,其默认值大约是当前字号的1.2倍,如果当前字号是10pt,
%%那么/baselineskip是12pt。这对英文排版是合适的,对中文就显得太拥挤了,因为英文正文多为小
%%写字母,字高与小写的x差不多(即1ex)。如果字号为10pt,那么1ex =4.3pt。我通常把行距设为字
%%号的1.8倍:
%%
%%{\color{DefinedColorRed}/setlength\{/baselineskip\}\{1.8em\}}
%%
%%这个值随时可以改,对更改点以后的文字有效
%%
%%LaTeX 用/parskip表示段距,我一般把它设为1ex:{\color{DefinedColorRed}/setlength\{/parskip\}\{1ex\}}
%%
%%注意这些修改长度的命令最好都放在正文区(即/begin\{document\}之后)
%%
%%\paragraph{页眉页脚}~{}
%%
%%/documentclass[10pt, a4paper]{book}
%%/usepackage{fancyhdr}
%%
%%在 LaTeX 中先把 page style 设为fancy,再设置这个style中的页眉和页脚。但是它默认每章的
%%第一页的page style是plain,需要单独处理。
%%
%%\begin{lstlisting}[language={matlab}]
%%% 设置 plain style 的属性
%%/fancypagestyle{plain}{%
%%/fancyhf{}                    % 清空当前设置
%%
%%% 设置页眉 (head)
%%/fancyhead[RE]{/leftmark}     % 在偶数页的右侧显示章名
%%/fancyhead[LO]{/rightmark}    % 在奇数页的左侧显示小节名
%%/fancyhead[LE,RO]{~/thepage~} % 在偶数页的左侧,奇数页的右侧显示页码
%%
%%% 设置页脚:在每页的右下脚以斜体显示书名
%%/fancyfoot[RO,RE]{/it Typesetting with /LaTeX}
%%
%%/renewcommand{/headrulewidth}{0.7pt} % 页眉与正文之间的水平线粗细
%%/renewcommand{/footrulewidth}{0pt}
%%}
%%
%%/pagestyle{fancy}             % 选用 fancy style
%%% 其余同 plain style
%%/fancyhf{}                   
%%/fancyhead[RE]{/leftmark}
%%/fancyhead[LO]{/rightmark}
%%/fancyhead[LE,RO]{~/thepage~}
%%/fancyfoot[RO,RE]{/it Typesetting with /LaTeX}
%%/renewcommand{/headrulewidth}{0.7pt}
%%/renewcommand{/footrulewidth}{0pt}
%%
%%% 设置章名和节名的显示方式
%%/renewcommand{/chaptermark}[1]{/markboth{~第~/thechapter~章~~~#1~}{}}
%%/renewcommand{/sectionmark}[1]{/markright{~/thesection~~#1~}{}}
%%\end{lstlisting}
%%
%%\paragraph{章节标题}~{}
%%
%%通常用titlesec宏包来设置正文中出现的章节标题的格式:
%%
%%/usepackage{titlesec}
%%
%%设置章名为右对齐,字号为/Huge,字型为黑体,章号用粗体,并设置间距:
%%
%%/titleformat{/chapter}{/flushright/Huge/hei}{{/bf /thechapter}}{0pt}{}
%%
%%/titlespacing{/chapter}{0pt}{-20pt}{25pt}
%%
%%设置节名的字号为/Large,字型为黑体,节号用粗体,并设置间距:
%%
%%/titleformat{/section}{/Large /hei}{{/bf /thesection/space}}{0pt}{}
%%
%%/titlespacing*{/section}{0pt}{1ex plus .3ex minus .2ex}{-.2ex plus .2ex}
%%
%%其中/hei的定义为:
%%
%%/newcommand{/hei}{/CJKfamily{hei}}
%%
%%\paragraph{纸张大小}~{}
%%
%%\begin{lstlisting}[language={matlab}]
%%/documentclass[10pt, b5paper]{report}
%%/usepackage[body={12.6cm, 20cm}, centering, dvipdfm]{geometry}
%%% 以上将版心宽度设为 12.6cm,高度 20cm,版心居中,且自动设置PDF文件的纸张大小
%%\end{lstlisting}
%%
%%\subsubsection{documentclass}
%%
%%latex 有三种文档类型(CTE 环境):ctexart、ctexrep、ctexbook
%%
%%优先选用 ctexrep 文档类型,因为这个支持到 chapter,且\textcolor{red}{目录}后面不会添加一个空白页
%%
%%而 book 类型是要考虑印刷时的奇偶页问题,所以会在\textcolor{red}{目录}后面在不满足 chapter 所在页是奇数页时会添加一个空白页,
%%这个是 book 规范设置
%%


\subsection{experience}

\begin{enumerate}[topsep=0pt,itemsep=0pt,parsep=0pt,leftmargin=3.6em,label=\arabic*>]
    \item enumerate 的 list 的 leftmargin 参考设置
    \begin{enumerate}[topsep=0pt,itemsep=0pt,parsep=0pt,leftmargin=1.8em]
        \item 最外层的\ {\color{DefinedColorRed}leftmargin = 3.6em}
        \item 嵌套第一层的\ {\color{DefinedColorRed}leftmargin = 1.8em},并且可以去掉 label 设置
    \end{enumerate}
    \item 在 lstlisting 环境中,如果需要对某行使用 latex 命令来进行特殊设置,比如高亮、斜体、粗体等
        需要在其逃逸符号\ {\color{DefinedColorRed}(反引号)}\ 中使用
\end{enumerate}





\newpage
\subsection{errors}

\begin{lstlisting}[language={c}]
Runaway argument?
{\376\377\0001\000.\0
! File ended while scanning use of \@@BOOKMARK.
<inserted text>
    \par
l.232 \begin{document}}
\end{lstlisting}

遇到这个错误是上一次编译中断后的 aux/out 文件的部分内容缺失,因此只需要删除 *.aux / *.out 文件之后重新编译即可



 % 

\chapter{Chapter One}


 % chapter one
%
\section{latex}

\textsl{\url{http://www.ctex.org/OnlineDocuments}}

\textsl{\url{http://www.ctex.org/documents/packages/layout/fancyhdr.htm}}

\url{http://cremeronline.com/LaTeX/minimaltikz.pdf} : Tikz 制图

\url{http://www.texample.net/tikz/}

\url{https://www.latexlive.com/} : 在线 tex formula 验证

\url{https://blog.csdn.net/lishoubox/article/details/6783726} : 这个基础知识写的不错

\url{https://mathpretty.com/10864.html} : 各种小技巧


\subsection{install}

\subsubsection{separate installation}

xelatex 可以编译出带目录的 pdf 文档 : (这两个安装包都很大,每个都几百M)

\begin{enumerate}[topsep=0pt,itemsep=0pt,parsep=0pt,leftmargin=3.6em,label=\arabic*>]
    \item sudo apt-get install texlive-xetex
    \item sudo apt-get install latex-cjk-all | chinese support
    \item sudo apt-get install texmaker | texmaker is tex editor
\end{enumerate}

\subsubsection{integrated installation}

推荐使用这种安装方式,且安装也比较简单

TeX Live Home Page : \url{http://tug.org/texlive/acquire-netinstall.html}

Texmaker Home Page : \url{http://www.xm1math.net/texmaker/}

\begin{enumerate}[topsep=0pt,itemsep=0pt,parsep=0pt,leftmargin=3.6em,label=\arabic*>]
    \item download texlive from TexLive HomePage : eg. texlive2019-20190410.iso
    \item sudo mount -o loop texlive2019-20190410.iso /media/iso
    \item cd /media/iso
    \item install texlive \par
        {\color{DefinedColorGreen}./install-tl}
    \item set .bashrc \par
        {\color{violet}export PATH=\verb!"!/usr/local/texlive/2019/bin/x86\_64-linux:\$PATH\verb!"!}
    \item set windows fonts, ref section \ref{sec:use-win-font-in-ubuntu}
\end{enumerate}



\subsection{basic}

\begin{enumerate}[topsep=0pt,itemsep=0pt,parsep=0pt,leftmargin=3.6em,label=\arabic*>]
    \item 输入一个反斜杠 : {\color{red}\ \verb|\textbackslash|} => ``\textbackslash"
    \item 输入大括号 : {\color{red}\ \verb|\{|} => ``\{"
    \item 输入百分号 : {\color{red}\ \verb|\%|} => ``\%"
    \item 输入一个空格 : ``\textbackslash\ " 或者 ``{ }"
    \item 输入\CmdFont{--} : \verb!\CmdFont{!\CmdFont{--}\verb!}! 会比较好看
    \item 中文标点(zh) ‘ | ; | ’ | , | 。| “ | ” \par
        \textcolor{teal}{就正常中文输入法中输入的标点符号,就可以直接显示为中文标点}
        {\color{red}(注意是 Calibri 字体的情况下才可以)}

    \item {\color{blue}Times New Roman 字体下的标点(en)}:\par
        \verb={\fontspec{Times New Roman}' | ' | " | " | , | ; | `}= 
        {\color{DefinedColorRed}$\Rightarrow$} % =>
        {\fontspec{Times New Roman}' | ' | " | " | , | ; | `} \par
        \verb={\fontspec{Times New Roman} ' "A" and "B" '}=
        {\color{DefinedColorRed}$\Rightarrow$} % =>
        {\fontspec{Times New Roman}' "A" and "B" '} \par
    \item {\color{blue}Calibri 字体下的标点(en)}{\color{red}(这种输入方式的字体会比较多)}:\par
        \verb={\fontspec{Calibri} ` | ' | `` | " | , | ; | \textasciigrave }=
        {\color{DefinedColorRed}$\Rightarrow$} % =>
        ` | ' | `` | " | , | ; | \textasciigrave \par
        \verb={\fontspec{Calibri} `\,``A" and ``B"\,' }=
        {\color{DefinedColorRed}$\Rightarrow$} % =>
        `\,``A" and ``B"\,'

    \item using symbol:\textbackslash symbol\{96\} {\color{DefinedColorRed}$\to$} \symbol{96} 、
        \textbackslash symbol\{39\} {\color{DefinedColorRed}$\to$} \symbol{39} \par
        通过使用 symbol 来输入字符,可以输入任意有编码的字符
    \item \^{}:input(\textbackslash \^{}\{\})
    \item \~{}:input(\textbackslash \~{}\{\}),在 \textbackslash CmdFont 中也可以使用这个方法来输入 \~{}
    \item \{\textbackslash songti 宋体\}:{\songti 宋体}
    \item \{\textbackslash kaishu 楷书\}:{\kaishu 楷书}
    \item \{\textbackslash heiti 黑体\}:{\heiti 黑体}
    \item \{\textbackslash fangsong 仿宋\}:{\fangsong 仿宋}
    \item \{\textbackslash fontspec\{YouYuan\} 幼圆\}:{\fontspec{YouYuan}幼圆}  % for win-font
    \item \{\textbackslash fontspec\{Consolas\} Consolas\}:{\fontspec{Consolas}Consolas}  % for win-font
        \textcolor{teal}{| 常规五号}
    \item \{\textbackslash zihao\{-5\}\textbackslash fontspec\{Consolas\} Consolas\}:
        {\zihao{-5}\fontspec{Consolas}Consolas}  % for win-font
        \textcolor{teal}{| 小五号}
    \item \{\textbackslash small\textbackslash fontspec\{Consolas\} Consolas\}:
        {\small\fontspec{Consolas}Consolas}  % for win-font
        \textcolor{teal}{| small 就是小五号}
    \item \textbackslash verb:eg. \verb'\verb!\songti!' 
        $\Rightarrow$ % =>
        \verb!\songti! \par
        \textcolor{red}{`!'} 表示环境符号,也可以用其他的符号替代 \par
        \verb={\color{teal}\verb!$\max_n$!}= => {\color{teal}\verb!$\max_n$!} : 给 \textbackslash verb 包围的上色
    \item 数字的下划线分割(使用公式环境):eg. \verb=0xf000$\underline\ $0000= 
        $\Rightarrow$ 
        0xf000$\underline\ $0000 \par
        {\color{teal}在 Calibri 字体下则可以直接使用\_,而不需要使用\textbackslash underline}
    \item url link:\par
        eg. \textbackslash url\{https://xxxx\} 
        $\Rightarrow$ % =>
        \url{https://xxxx} \par
        eg. \verb=\textsl{\url{/https://xxxx}}=
        $\Rightarrow$ % =>
        \textsl{\url{https://xxxx}} \par
        eg. \verb=\emph{\url{/https://xxxx}}=
        $\Rightarrow$ % =>
        \emph{\url{https://xxxx}} | \textcolor{teal}{在 Calibri 字体下会比较好看}
    \item \textbackslash vspace\{-0.8em\}:设置前后行的间距
    \item \textbackslash dotfill:绘制虚线分割线,可以查看 section \ref{sec:book-list} 中的使用效果
    \item \textbackslash hrulefill:绘制实线分割线
\end{enumerate}

\subsubsection{lstlisting / verbatim}

\paragraph{enumerate}~{}

\begin{lstlisting}[language={python}]
\begin{enumerate}

\begin{enumerate}[topsep=0pt,itemsep=0pt,parsep=0pt,leftmargin=3.6em,label=\arabic*>]
\begin{enumerate}[topsep=0pt,itemsep=0pt,parsep=0pt,leftmargin=3.6em,label=\arabic*.]
\begin{enumerate}[topsep=0pt,itemsep=0pt,parsep=0pt,leftmargin=3.6em,label=\arabic*)]
\begin{enumerate}[topsep=0pt,itemsep=0pt,parsep=0pt,leftmargin=3.6em,label=(\arabic*)]

# 大小写罗马数字
\begin{enumerate}[topsep=0pt,itemsep=0pt,parsep=0pt,leftmargin=3.6em,label=\roman*>]
\begin{enumerate}[topsep=0pt,itemsep=0pt,parsep=0pt,leftmargin=3.6em,label=\Roman*>]

# 大小写字母
\begin{enumerate}[topsep=0pt,itemsep=0pt,parsep=0pt,leftmargin=3.6em,label=\alph*>]
\begin{enumerate}[topsep=0pt,itemsep=0pt,parsep=0pt,leftmargin=3.6em,label=\Alph*>]

# 特殊符号
\begin{enumerate}[topsep=0pt,itemsep=0pt,parsep=0pt,leftmargin=3.6em,label=\fnsymbol*>]
\end{lstlisting}

\begin{lstlisting}[language={python}]
\begin{enumerate}[topsep=0pt,itemsep=0pt,parsep=0pt,leftmargin=3.6em,label=\arabic*>]
    \item 编号
        \arabic{enumi}, \roman{enumi}, \Roman{enumi}, 
        \alph{enumi}, \Alph{enumi}, \fnsymbol{enumi}
    \item 编号
        \arabic{enumi}, \roman{enumi}, \Roman{enumi}, 
        \alph{enumi}, \Alph{enumi}, \fnsymbol{enumi}
    \item 编号
        \arabic{enumi}, \roman{enumi}, \Roman{enumi}, 
        \alph{enumi}, \Alph{enumi}, \fnsymbol{enumi}
\end{enumerate}
\end{lstlisting}

\begin{enumerate}[topsep=0pt,itemsep=0pt,parsep=0pt,leftmargin=3.6em,label=\arabic*>]
    \item 编号
        \arabic{enumi}, \roman{enumi}, \Roman{enumi}, \alph{enumi}, \Alph{enumi}, \fnsymbol{enumi}
    \item 编号
        \arabic{enumi}, \roman{enumi}, \Roman{enumi}, \alph{enumi}, \Alph{enumi}, \fnsymbol{enumi}
    \item 编号
        \arabic{enumi}, \roman{enumi}, \Roman{enumi}, \alph{enumi}, \Alph{enumi}, \fnsymbol{enumi}
\end{enumerate}


\paragraph{lstlisting}~{}

\begin{lstlisting}[language={python}]
`\verb!\begin{lstlisting}[language={python}]!`
xxx
`\verb=\end{lstlisting}=`

`\verb!\begin{lstlisting}[language={c++}]!`
`\verb!\begin{lstlisting}[language={sh}]!`
\end{lstlisting}

可以在 lstlisting 环境中使用 \textbackslash verb 来输入各种特殊的字符,比如 {\color{red}\ \verb=`$CMD_A`=}

\begin{lstlisting}[language={python}]
`\verb!\verb=`$CMD_A`=!` `{\color{red}->}` `\verb=`$CMD_A`=`
\end{lstlisting}

lstlisting 在序列中的项时这么使用:

\begin{lstlisting}[language={python}]
    ...
    `\verb!\item!` a item
`\verb!\begin!`{lstlisting}[language={sh}, `{\color{red}xleftmargin=0em}`]
...
`\verb!\end!`{lstlisting}
    `\verb!\item!` other item
    ...
\end{lstlisting}



\paragraph{verbatim}~{}

缩略写法:\textbackslash verb=\textcolor{red}{xxx}= : \textcolor{red}{xxx} 是要原格式表示的内容

%\begin{verbatim}
%    \usepackage[T1]{fontenc}            % font styling
%    \usepackage{lmodern, mathrsfs}
%\end{verbatim}

\begin{lstlisting}[language={python}]
\makeatletter
\renewcommand*{\verbatim@font}{}
\makeatother
\end{lstlisting}

如果需要设置 \textbackslash verb 环境中的颜色,可以直接在外部包上 \textbackslash color\{\} 来进行设置即可

如果还需要在  \textbackslash verb 环境中设置字体,可以以如下方式来使用:

\begin{lstlisting}[language={c}]
{\small\ttfamily\fontspec{Consolas}\verb!grep `-`in "\`-`\`-`" ./ `-`R!}
\end{lstlisting}

%\begin{lstlisting}[language={python}]
%\verb={\fontspec{Times New Roman} abcdefgxxxx}= 
%\end{lstlisting}


\subsubsection{tabular、tabularx、array}

table to latex code : \url{https://www.tablesgenerator.com/}

\url{https://github.com/krlmlr/Excel2LaTeX}

文本或数学模式都可以使用 tabular,数学模式还可以使用 array 环境(即包含数学符号的公式)

table 浮动表格中的几个命令设置如下:

\begin{lstlisting}[language={tex}]
...
\setlength{\abovecaptionskip}{1.0em} %
\setlength{\belowcaptionskip}{`-`1.3em}% 这两个必须得放在 caption 前面,否则不起作用
\caption{mxnet 的基本概念}
...
\end{lstlisting}

\paragraph{tabular}~{}

tabular 中的参数设置: 
\begin{enumerate}[topsep=0pt,itemsep=0pt,parsep=0pt,leftmargin=3.6em,label=\arabic*>]
    \item \textbackslash raggedleft:表示左边不对齐,\textcolor{red}{即右对齐}
    \item \textbackslash raggedright:表示右边不对齐,\textcolor{red}{即左对齐}
    \item \textbackslash centering:表示居中
    \item p\{width value\}:设置列框,使用了列宽后,需在列内容处使用上述的 对其方式
    \item \textbackslash arraystretch\{val\}:设置行间距,默认值为 1
        eg. \textbackslash renewcommand\textbackslash arraystretch\{2\} | 默认值的 2 倍
    \item \textbackslash multicolumn:合并不同列 \par
        \textbackslash multicolumn\{列数\}\{格式\}\{内容\} \par
        \textbackslash multicolumn\{2\}\{c|\}\{内容\} : {\color{red}c 表示居中,| 表示边框} \par
        \textbackslash multicolumn\{2\}\{l|\}\{内容\} : left \par
        \textbackslash multicolumn\{2\}\{r|\}\{内容\} : right \par
        \textcolor{DefinedColorGreen}{如果只是针对 1 列操作时,相当于是修改对齐、竖线格式}
    \item \textbackslash multirow:跨不同行,类似于合并不同行 | \textbackslash usepackage\{multirow\}\par
        \textbackslash multirow\{行数\}\{宽度\}\{内容\} \par
        \textbackslash multirow\{行数\}*\{内容\} | 宽度由输入内容来决定 \par
        需要结合 \textbackslash cline{start - end} 来使用 eg. \verb!\cline{2-5}!
\end{enumerate}

\begin{lstlisting}[language={tex}]
\begin{table}[htbp!]
    \centering
    \setlength{\abovecaptionskip}{0.5em}    % 这个设置在 tabular 前面的话的参数值
    \setlength{\belowcaptionskip}{`-`0.5em}
    \caption{xxx}
    \label{tab:xxx}
\begin{tabular}{|l|l|}
...
\end{lstlisting}

\begin{lstlisting}[language={tex}]
...
\end{tabular}
    \centering
    \setlength{\abovecaptionskip}{1.0em}    % 设置在 tabular 后面的话的参数值
    \setlength{\belowcaptionskip}{`-`1.3em}
    \caption{xxx}
    \label{tab:xxx}
\end{table}
\end{lstlisting}


\begin{center}
%\renewcommand\arraystretch{2}
\begin{tabular}{|p{8em}|c|l|p{4em}|p{4em}|}
    \hline
    \raggedleft{\textbackslash raggedleft} & abc & abc & abc & abc \\
    \hline
    \raggedright{\textbackslash raggedright} & abc & abc & \multicolumn{2}{c|}{abc} \\
    \hline
    \centering{\textbackslash centering} & abc & abc & abc & abc \\
    \hline
    default & abc & abc & abc & abc \\
    \hline
    \multirow{2}*{multirow} & abc & abc & abc & abc \\

    %
    % 这里表示只需要对 2-5 列画横线 , \hline 表示从第 1 列到最后 1 列都画横线
    %
    \cline{2-5}
                            & abc & abc & abc & abc \\
    \hline
    default & abc & abc & abc & abc \\
    \hline
    \centering{\multirow{2}*{multirow}} & abc & abc & abc & abc \\
    \cline{2-5}
                            & abc & abc & abc & abc \\
    \hline
    \raggedleft{\multirow{2}*{multirow}} & abc & abc & abc & abc \\
    \cline{2-5}
                            & abc & abc & abc & abc \\
    \hline
\end{tabular}
\end{center}

\paragraph{tabularx}~{}

tabularx 环境是固定宽度的表格,这里的固定宽度指\textcolor{red}{铺满页面宽度}

%
% Y 是重新定义的格式,\arraybackslash 表示命令恢复
%
\newcolumntype{C}{>{\centering\arraybackslash}X}
\newcolumntype{R}{>{\raggedleft\arraybackslash}X}
\begin{tabularx}{46em}{|c|X|X|R|C|}
    \hline
    数字 & 1 & 2 & 3 & 4 \\
    \hline
    英文 & one & two & three & four \\
    \hline
\end{tabularx}

\paragraph{longtable}~{}

\begin{lstlisting}[language={tex}]
\begin{longtable}{|l|m{40em}|}
    \hline
    aaa   & xxx \\
    \hline
    bbb   & xxx \\
    \hline
    ccc   & xxx # 这样可以实现换行

          1> xxx

          2> xxx \\
    \hline
\end{longtable}
\end{lstlisting}

\begin{longtable}{|l|m{40em}|}
    \hline
    aaa   & xxx \\
    \hline
    bbb   & xxx \\
    \hline
    ccc   & xxx

          1> xxx

          2> xxx \\
    \hline
\end{longtable}

\paragraph{array | tabu}~{}

\begin{enumerate}[topsep=0pt,itemsep=0pt,parsep=0pt,leftmargin=3.6em,label=\arabic*>]
    \item 列格式控制:\textbackslash usepackage\{array\}
    \item 行格式控制:\textbackslash usepackage\{tabu\}
\end{enumerate}

\subparagraph{array}~{}

array 新增的基本命令(\verb=\usepackage{array}=):
\begin{enumerate}[topsep=0pt,itemsep=0pt,parsep=0pt,leftmargin=3.6em,label=\arabic*>]
    \item \verb=m{宽}=:类似 p 格式,产生具有固定宽度的列,并可以自动换行,垂直方向居中对齐 \par
        这个 m 格式在 tabular 、longtable 中都是可以使用的
    \item \verb=b{宽}=:类似 p 格式,垂直方向与最后一行对齐
    \item \verb=>{xx}=:把 xx 插入后面一列的开头
    \item \verb=<{xx}=:把 xx 插入前面一列的末尾
    \item \verb=!{xx}=:把 xx 作为表格线处理,相当于使用了 \verb=@{xx}= 但左右两边会有额外的间距 \par
        eg. \verb=c!{$\Rightarrow$}=
\end{enumerate}

格式符 > 与 < 通常用来设置整列的格式,例如改变一列表格的字体或者使用数学模式

\begin{tabular}{>{\bfseries}c|>{\itshape}c>{$}c<{$}}
    \hline
    姓名 & \textnormal{得分} & \multicolumn{1}{c}{额外加分} \\
    \hline
    A & 85 & +7 \\
    B & 82 & 0 \\
    C & 70 & -2 \\
    \hline
\end{tabular}

%
% 这个可以自动居中
%
\begin{tabular}{|>{$}r<{$}|>{\setlength\parindent{2em}}m{35em}|>{\centering\arraybackslash}m{4em}|}
    \hline
    \pi & 希腊字母,希腊字母希腊字母希腊字母希腊字母希腊字母希腊字母希腊字母希腊字母 & 常用 \\
    \hline
    \pi & 希腊字母,希腊字母希腊字母希腊字母希腊字母希腊字母希腊字母希腊字母希腊字母 & 常用 \\
    \hline
\end{tabular}

array 宏包提供的\verb=\newcolumntype= 用于定义新的列格式的命令:
\begin{enumerate}[topsep=0pt,itemsep=0pt,parsep=0pt,leftmargin=3.6em,label=\arabic*>]
    \item \verb=\newcolumntype{M}{>{$}c<{$}}=
    \item \verb=\newcolumntype{P}[1]{>{\setlength\parindent{2em}}p{#1}}=
    \item \verb=\newcolumntype{C}[1]{>{\centering\arraybackslash}m{#1}}= \par
        eg. \verb=\begin{tabular}{|M|P{15em}|C{4em}|}=
\end{enumerate}

\subparagraph{tabu}~{}

tabu 宏包提供 \textbackslash rowfont 命令来设置行的格式

\begin{tabu}{ccc}
    \hline
    \rowfont{\bfseries} 姓名 & 得分 & \multicolumn{1}{c}{额外加分} \\
    \hline
    \rowfont{\color{blue}} A & 85 & +7 \\
    \rowfont{\color{red}} B & 82 & 0 \\
    \rowfont{\color{green}} C & 70 & -2 \\
    \hline
\end{tabu}

\paragraph{threeparttable}~{}

\begin{enumerate}[topsep=0pt,itemsep=0pt,parsep=0pt,leftmargin=3.6em,label=\arabic*>]
    \item \verb=usepackage{threeparttable}=
    \item \verb=\tnote{val}=:使用这个来定义表格中的角标 eg.\par
        \textcolor{red}{\textbackslash tnote\{2\}} \par
        \textcolor{red}{\textbackslash tnote\{a\}} \par
        \textcolor{red}{\textbackslash tnote\{I\}} \par
        \textcolor{red}{\textbackslash tnote\{*\}}
\end{enumerate}

\begin{center}
\begin{threeparttable} 
    \centering 
    \caption{threeparttable example} 
    \label{tbl:threeparttable example} 
    \begin{tabular}{l|cc|cc}
        \hline 
        $F_{1}$ &$3.70e-03$ &$4.20e-03$\tnote{2} &$ \textbf{3.70e-03}$ &$3.90e-03$\\
        $F_{2}$ &$3.60e-03$\tnote{a} &$1.55e-02$ &$\textbf{6.70e-03}$ &$8.50e-03$\\ 
        \hline
    \end{tabular} 
    \begin{tablenotes} 
    \item[1] The bolder ones mean better. 
    \item[2] my note
    \item[a] my note a
    \end{tablenotes} 
\end{threeparttable} 
\end{center}



\subsubsection{set color}

\begin{enumerate}[topsep=0pt,itemsep=0pt,parsep=0pt,leftmargin=3.6em,label=\arabic*>]
    \item \{\textbackslash color\{blue\}\{蓝色字\}\}:{\color{blue}{蓝色字}}
    \item \textbackslash textcolor\{red\}\{红色字体\}:\textcolor{red}{红色字体}
    \item \textbackslash colorbox\{yellow\}\{黄色盒子\}:\colorbox{yellow}{黄色盒子}
    \vspace{-0.4em}
    \item \textbackslash fcolorbox\{black\}\{green\}\{黑框绿色背景盒子\}:\fcolorbox{black}{green}{黑框绿色背景盒子}
\end{enumerate}

\begin{longtable}{cp{5em}|cp{5em}|cp{5em}|cp{5em}|cp{5em}|cp{5em}}
    \hline
    \colorbox{black}{\textcolor{black}{color}} & \multicolumn{1}{c|}{black} &
    \colorbox{gray}{\textcolor{gray}{color}} & \multicolumn{1}{c|}{gray} &
    \colorbox{olive}{\textcolor{olive}{color}} & \multicolumn{1}{c|}{olive} &
    \colorbox{blue}{\textcolor{blue}{color}} & \multicolumn{1}{c|}{blue} &
    \colorbox{green}{\textcolor{green}{color}} & \multicolumn{1}{c|}{green} &
    \colorbox{teal}{\textcolor{teal}{color}} & \multicolumn{1}{c}{teal} \\
    \hline
    \colorbox{violet}{\textcolor{violet}{color}} & \multicolumn{1}{c|}{violet} &
    \colorbox{brown}{\textcolor{brown}{color}} & \multicolumn{1}{c|}{brown} &
    \colorbox{lightgray}{\textcolor{lightgray}{color}} & \multicolumn{1}{c|}{lightgray} &
    \colorbox{pink}{\textcolor{pink}{color}} & \multicolumn{1}{c|}{pink} &
    \colorbox{white}{\textcolor{white}{color}} & \multicolumn{1}{c|}{white} &
    \colorbox{lime}{\textcolor{lime}{color}} & \multicolumn{1}{c}{lime} \\
    \hline
    \colorbox{purple}{\textcolor{purple}{color}} & \multicolumn{1}{c|}{purple} &
    \colorbox{yellow}{\textcolor{yellow}{color}} & \multicolumn{1}{c|}{yellow} &
    \colorbox{darkgray}{\textcolor{darkgray}{color}} & \multicolumn{1}{c|}{darkgray} &
    \colorbox{magenta}{\textcolor{magenta}{color}} & \multicolumn{1}{c|}{magenta} &
    \colorbox{red}{\textcolor{red}{color}} & \multicolumn{1}{c|}{red} &
    \colorbox{orange}{\textcolor{orange}{color}} & \multicolumn{1}{c}{orange} \\
    \hline
    \colorbox{cyan}{\textcolor{cyan}{color}} & \multicolumn{1}{c|}{cyan} \\
    \hline
\end{longtable}

\begin{longtable}{cp{6em}|cp{6em}|cp{6em}|cp{6em}|cp{6em}}
    \hline
    \colorbox{Yellow}{\textcolor{Yellow}{color}} & \multicolumn{1}{c|}{Yellow} &
    \colorbox{GreenYellow}{\textcolor{GreenYellow}{color}} & \multicolumn{1}{c|}{GreenYellow} &
    \colorbox{RubineRed}{\textcolor{RubineRed}{color}} & \multicolumn{1}{c|}{RubineRed} &
    \colorbox{RoyalPurple}{\textcolor{RoyalPurple}{color}} & \multicolumn{1}{c|}{RoyalPurple} &
    \colorbox{Emerald}{\textcolor{Emerald}{color}} & \multicolumn{1}{c}{Emerald} \\
    \hline
    \colorbox{WildStrawberry}{\textcolor{WildStrawberry}{color}} & \multicolumn{1}{c|}{WildStrawberry} &
    \colorbox{BlueViolet}{\textcolor{BlueViolet}{color}} & \multicolumn{1}{c|}{BlueViolet} &
    \colorbox{Emerald}{\textcolor{Emerald}{color}} & \multicolumn{1}{c}{Emerald} \\
    \hline
\end{longtable}

\begin{enumerate}[topsep=0pt,itemsep=0pt,parsep=0pt,leftmargin=3.6em,label=\arabic*>]
    \item \textbackslash columncolor:eg. \verb!\begin{tabular}{>{\columncolor{gray}}c >{\columncolor{red}}c}!
    \item \textbackslash rowcolor:eg. \verb!\rowcolor{red} A & B & C \\!
    \item \verb=\setlength{\arrayrulewidth}{0.19mm}=:设置表格横线与竖线的粗细,实验下来设置 0.19 mm 最好,
        设置颜色时不会把线给遮挡了,而且线条显示也均匀 \par
        可以把这个参数设置到\textcolor{red}{导言区}
    \item \verb=\arrayrulecolor{green}= 用来设置表格线条为\textcolor{green}{绿色} \par
        后面的表格如果想要恢复成黑色,需要再重新设置一下
\end{enumerate}


\arrayrulecolor{green}
\begin{tabular}{|l|c|r|}
    \arrayrulecolor{black}\hline
    United Kingdom & London & Thames\\
    \arrayrulecolor{blue}\hline
    France & Paris & Seine \\
    \arrayrulecolor{black}\cline{1-1}
    \arrayrulecolor{red}\cline{2-3}
    Russia & Moscow & Moskva \\ \hline
\end{tabular}


\arrayrulecolor{black}
\begin{tabular}{|>{\columncolor{lime}[6pt][6pt]}c|l|l|r|c|}
    \hline
    数字 & 1 & 2 & 3 & \cellcolor{red}4 \\
    \hline
    英文 & one & two & three & four \\
    \hline
    \rowcolor{lightgray}[\tabcolsep][\tabcolsep] 汉文 & 一 & 二 & 三 & 四 \\
    \hline
\end{tabular}




\subsubsection{define command and environment}

\paragraph{define command}~{}

使用 \verb=\newcommand<命令>[<参数个数>][<首参数默认值>]{<具体定义>}= 来定义一个新的命令
\begin{enumerate}[topsep=0pt,itemsep=0pt,parsep=0pt,leftmargin=3.6em,label=\arabic*>]
    \item 无参数定义:\par
        \verb=\newcommand<命令>{<具体定义>}= 
    \item 有参数定义:\par
        参数编号使用\verb=#1、#2、#3、...、#9=
\end{enumerate}

以下是一些定义实例:

\newcommand\PRCa{\emph{China}}
\newcommand\PRCb{\textcolor{red}{\emph{China}}}
\newcommand\PRCc[1]{\textcolor{blue}{\textcolor{red}{#1} country is \emph{China}}}
\newcommand\PRCd[1][my]{\textcolor{blue}{\textcolor{red}{#1} country is \emph{China}}}

\begin{enumerate}[topsep=0pt,itemsep=0pt,parsep=0pt,leftmargin=3.6em,label=\arabic*>]
    \item \verb=\newcommand\PRCa{\emph{China}}= \par
        \verb=\PRCa= $\Rightarrow$ \PRCa
    \item \verb=\newcommand\PRCb{\textcolor{red}{\emph{China}}}= \par
        \verb=\PRCb= $\Rightarrow$ \PRCb
    \item \verb=\newcommand\PRCc[1]{\textcolor{blue}{\textcolor{red}{#1} country is \emph{China}}}= \par
        \verb=\PRCc{my}= $\Rightarrow$ \PRCc{my}
    \item \verb=\newcommand\PRCd[1][my]{\textcolor{blue}{\textcolor{red}{#1} country is \emph{China}}}= \par
        \verb=\PRCd= $\Rightarrow$ \PRCd \par
        \verb=\PRCd[your]= $\Rightarrow$ \PRCd[your] \par
\end{enumerate}

\paragraph{renew command}~{}

使用 \verb=\renewcommand<命令>[<参数个数>][<首参数默认值>]{<具体定义>}= 来重定义命令

\paragraph{new/renew environment}~{}



\subsubsection{font set}

在 linux 中 latex 字体如果想使用 win 的字体,需要先把 win 的字体拷贝到 linux 中,具体操作可以参考 
\ref{sec:use-win-font-in-ubuntu}

在导言区设置如下,就可以使用 win 的字体:
\begin{lstlisting}[language={matlab}]
\usepackage{fontspec}
%\setmainfont{Times New Roman}   % 这些字体可以用 fc`-`list 中查看名字
\setmainfont{Calibri}
\setsansfont{Verdana}
%\setmonofont{Courier New}
%\setmonofont{Consolas}
\end{lstlisting}

\begin{lstlisting}[language={sh}]
minphone@mcp:~$fc`-`list
/usr/share/fonts/windows_fonts/BROADW.TTF: `\textcolor{red}{Broadway}`:style=Regular,obyčejné
/usr/share/fonts/windows_fonts/consola.ttf: `\textcolor{red}{Consolas}`:style=Regular
/usr/share/fonts/windows_fonts/constanb.ttf: `\textcolor{red}{Constantia}`:style=Bold
/usr/share/fonts/windows_fonts/ONYX.TTF: `\textcolor{red}{Onyx}`:style=Regular,obyčejné
\end{lstlisting}

{\color{red}按以下方式来使用字体:}
\begin{enumerate}[topsep=0pt,itemsep=0pt,parsep=0pt,leftmargin=3.6em,label=\arabic*>]
    \item \{\textbackslash fontspec\{Microsoft YaHei Light\} 微软雅黑 Light\}:
        {\fontspec{Microsoft YaHei Light}微软雅黑 Light}  % for win-font
    \item \{\textbackslash fontspec\{YouYuan\} 幼圆\}:{\fontspec{YouYuan}幼圆}  % for win-font
    \item \{\textbackslash fontspec\{Comic Sans MS\} Comic Sans MS\}:{\fontspec{Comic Sans MS}Comic Sans MS}  % for win-font
    \item \{\textbackslash small\textbackslash fontspec\{Comic Sans MS\} Comic Sans MS\}
        {\small\fontspec{Comic Sans MS}Comic Sans MS}  % for win-font

    \item \{\textbackslash fontsize\{8pt\}\{0\}\textbackslash fontspec\{Comic Sans MS\} Comic Sans MS\}
        {\fontsize{8pt}{0}\fontspec{Comic Sans MS}Comic Sans MS}  % for win-font

    \item \{\textbackslash small\textbackslash fontspec\{Consolas\} Consolas\}:
        {\small\fontspec{Consolas}Consolas}  % for win-font
        | small {\color{teal}就是小五号}
    \item \verb=\textup{upright shape}= : \textup{upright shape}
    \item \verb=\textit{italic shape}= : \textit{italic shape}
    \item \verb=\textsl{slanted shape}= : \textsl{slanted shape}
    \item \verb=\textsc{small capitals shape}= : \textsc{small capticals shape}
    \item \verb=\textbf{apple}= : \textbf{apple}
\end{enumerate}

\paragraph{set font size}~{}

\begin{enumerate}[topsep=0pt,itemsep=0pt,parsep=0pt,leftmargin=3.6em,label=\arabic*>]
    \item \verb=\tiny{Text}= : {\tiny{Text}}
    \item \verb=\small{Text}= : {\small{Text}}
    \item \verb=\normalsize{Text}= : {\normalsize{Text}}
    \item \verb=\large{Text}= : {\large{Text}}
    \item \verb=\huge{Text}= : {\huge{Text}}
    \item \verb={\fontsize{8pt}{0}Text}= : {\fontsize{8pt}{0}Text}
\end{enumerate}

各个字号的榜值 : \url{https://blog.csdn.net/weixin_39679367/article/details/115794548}

\paragraph{font style of math}~{}

\begin{enumerate}[topsep=0pt,itemsep=0pt,parsep=0pt,leftmargin=3.6em,label=\arabic*>]
    \item \verb=$\mathnormal{ABCDEF, abcdef, 123456}$= : {$\mathnormal{ABCDEF, abcdef, 123456}$}
    \item \verb=$\mathnormal{123456}$= : {$\mathnormal{123456}$}
    \item \verb=$\mathrm{ABCDEF, abcdef, 123456}$= : {$\mathrm{ABCDEF, abcdef, 123456}$}
    \item \verb=$\mathit{ABCDEF, abcdef, 123456}$= : {$\mathit{ABCDEF, abcdef, 123456}$}
    \item \verb=$\mathbf{ABCDEF, abcdef, 123456}$= : {$\mathbf{ABCDEF, abcdef, 123456}$}
    \item \verb=$\mathsf{ABCDEF, abcdef, 123456}$= : {$\mathsf{ABCDEF, abcdef, 123456}$}
    \item \verb=$\mathtt{ABCDEF, abcdef, 123456}$= : {$\mathtt{ABCDEF, abcdef, 123456}$}
    \item \verb=$\mathcal{ABCDEF}$= : {$\mathcal{ABCDEF}$}
\end{enumerate}



\newpage

\subsubsection{formula | \$ xxx \$}

\begin{enumerate}[topsep=0pt,itemsep=0pt,parsep=0pt,leftmargin=3.6em,label=\arabic*>]
    \item {\color{teal}\verb!$a \dots b$!} => $a \dots b$
    \item {\color{teal}\verb!$a \vdots b$!} => $a \vdots b$
    \item {\color{teal}\verb!$a \ddots b$!} => $a \ddots b$
    \item {\color{teal}\verb!$\{aaa\}$!} => $\{aaa\}$
    \item {\color{teal}\verb!$[aaa]$!} => $[aaa]$
    %\item {\color{teal}\verb!$a \iddots b$!} => $a \iddots b$
\end{enumerate}

\paragraph{上标/下标}~{}

\begin{enumerate}[topsep=0pt,itemsep=0pt,parsep=0pt,leftmargin=3.6em,label=\arabic*>]
    \item {\color{teal}\verb!$x_{1}$!} => $x_{1}$
    \item {\color{teal}\verb!$x^{2}$!} => $x^{2}$
    \item {\color{teal}\verb!$A_{ij} = 2^{i+j}$!} => $A_{ij} = 2^{i+j}$
    \item {\color{teal}\verb!$A\prime$!} => $A\prime$
    \item {\color{teal}\verb!$A'$!} => $A'$
    \item {\color{teal}\verb!$A''$!} => $A''$
    \item {\color{teal}\verb!$A = 90^\circ$!} => $A = 90^\circ$
    \item {\color{teal}\verb!$\max_n f(n)$!} => $\max_n f(n)$
    \item {\color{teal}\verb!$A_m{}^n$!} => $A_m{}^n$
    \item {\color{teal}\verb!$\int_0^1 f(t)$!} => $\int_0^1 f(t)$
    \item {\color{teal}\verb!$\iint_0^1 f(t)$!} => $\iint_0^1 f(t)$
    \item {\color{teal}\verb!$\int_0^1 f(t)dt$!} => $\int_0^1 f(t)dt$
    \item {\color{teal}\verb!$\iint_0^1 g(x, y)dxdy$!} => $\iint_0^1 g(x, y)dxdy$
    \item {\color{teal}\verb!$\int_0^1 f(t)\mathrm{d}t$!} => $\int_0^1 f(t)\mathrm{d}t$
    \item {\color{teal}\verb!$\vec x = \overrightarrow{AB}$!} => $\vec x = \overrightarrow{AB}$

    \item {\color{teal}\verb!$\overset{*}{X}$!} => $\overset{*}{X}$ {\color{red}|} 
          {\color{teal}\verb!$\underset{*}{X}$!} => $\underset{*}{X}$ {\color{red}|} 
          {\color{teal}\verb!$\underset{m}{X}$!} => $\underset{m}{X}$
    \item {\color{teal}\verb!\[\max_n f(n)\]!} => \[\max_n f(n)\]
\end{enumerate}

\paragraph{分式}~{}

\begin{enumerate}[topsep=0pt,itemsep=0pt,parsep=0pt,leftmargin=3.6em,label=\arabic*>]
    \item {\color{teal}\verb!$\frac 12 + \frac 1a = \frac{2+a}{2a}$!} => $\frac 12 + \frac 1a = \frac{2+a}{2a}$
    \item {\color{teal}\verb!\[\tfrac 12 f(x) = \frac{1}{\dfrac 1a + \dfrac 1b + c}\]!} 
        => \[\tfrac 12 f(x) = \frac{1}{\dfrac 1a + \dfrac 1b + c}\] \par
        {\color{teal}\verb!\dfrac!} : display style, looks smaller \par
        {\color{teal}\verb!\tfrac!} : text style, looks normal
\end{enumerate}

\paragraph{根式}~{}

\begin{enumerate}[topsep=0pt,itemsep=0pt,parsep=0pt,leftmargin=3.6em,label=\arabic*>]
    \item {\color{teal}\verb!$\sqrt 4 = \sqrt[3]{8} = 2$!} => $\sqrt 4 = \sqrt[3]{8} = 2$
\end{enumerate}

\paragraph{inline formula : 行内公式}~{}

行内公式是嵌在文本中的,其输入可以有三种方式:

\begin{enumerate}[topsep=0pt,itemsep=0pt,parsep=0pt,leftmargin=3.6em,label=\arabic*>]
    \item \$ \emph{formula} \$ \par
        {\color{DefinedColorGreen}\$ a + b > c \$} 
        $\Rightarrow$ % =>
        {\color{DefinedColorRed}$a + b > c$}
    \item \textbackslash( \emph{formula} \textbackslash) \par
        {\color{DefinedColorGreen}\textbackslash( a + b > c \textbackslash)} 
        $\Rightarrow$ % =>
        {\color{DefinedColorRed}\(a + b > c\)}
    \item \textbackslash begin\{math\} \emph{formula} \textbackslash end\{math\} \par
        {\color{DefinedColorGreen} \textbackslash begin\{math\} a + b > c \textbackslash end\{math\}} 
        $\Rightarrow$ % =>
        {\color{DefinedColorRed}\begin{math} a + b > c \end{math} }
\end{enumerate}

\paragraph{line formula : 行间公式}~{}

行间公式是单独占据一行居中显示,其输入也有三种:
\begin{enumerate}[topsep=0pt,itemsep=0pt,parsep=0pt,leftmargin=3.6em,label=\arabic*>]
    \item {\color{teal}\verb!\[\max_n f(n)\]!} => \[\max_n f(n)\]
\end{enumerate}

\paragraph{matrix}~{}

matrix 环境共有以下几种:

\begin{enumerate}[topsep=0pt,itemsep=0pt,parsep=0pt,leftmargin=3.6em,label=\arabic*>]
    \item matrix
    \item bmatrix
    \item vmatrix
    \item pmatrix
    \item Bmatrix
    \item Vmatrix
\end{enumerate}

\[ A_{\mathrm{matrix}} = 
\begin{matrix}
    a_{11} & a_{12} & a_{13} \\
    a_{21} & a_{22} & a_{23} \\
    a_{31} & a_{32} & a_{33}
\end{matrix}
\]

\[ A_{\mathrm{bmatrix}} = 
\begin{bmatrix}
    a_{11} & a_{12} & a_{13} \\
    a_{21} & a_{22} & a_{23} \\
    a_{31} & a_{32} & a_{33}
\end{bmatrix}
\]

\[ A_{\mathrm{vmatrix}} = 
\begin{vmatrix}
    a_{11} & a_{12} & a_{13} \\
    a_{21} & a_{22} & a_{23} \\
    a_{31} & a_{32} & a_{33}
\end{vmatrix}
\]

\[ A_{\mathrm{pmatrix}} = 
\begin{pmatrix}
    a_{11} & a_{12} & a_{13} \\
    a_{21} & a_{22} & a_{23} \\
    a_{31} & a_{32} & a_{33}
\end{pmatrix}
\]

\[ A_{\mathrm{Vmatrix}} = 
\begin{Vmatrix}
    a_{11} & a_{12} & a_{13} \\
    a_{21} & a_{22} & a_{23} \\
    a_{31} & a_{32} & a_{33}
\end{Vmatrix}
\]

%%%%%%%%%%%%%%%%%%%%%%%%%%%%%%%%%%%%%%%%%%

\begin{lstlisting}[language={python}]
\[ A_{\mathrm{matrix}} = 
\begin{matrix}
    a_{11} & a_{12} & a_{13} \\
    a_{21} & a_{22} & a_{23} \\
    a_{31} & a_{32} & a_{33}
\end{matrix}
\]
\end{lstlisting}

\begin{lstlisting}[language={python}]
\[ A_{\mathrm{bmatrix}} = 
\begin{bmatrix}
    a_{11} & a_{12} & a_{13} \\
    a_{21} & a_{22} & a_{23} \\
    a_{31} & a_{32} & a_{33}
\end{bmatrix}
\]
\end{lstlisting}

\begin{lstlisting}[language={python}]
\[ A_{\mathrm{vmatrix}} = 
\begin{vmatrix}
    a_{11} & a_{12} & a_{13} \\
    a_{21} & a_{22} & a_{23} \\
    a_{31} & a_{32} & a_{33}
\end{vmatrix}
\]
\end{lstlisting}

\begin{lstlisting}[language={python}]
\[ A_{\mathrm{pmatrix}} = 
\begin{pmatrix}
    a_{11} & a_{12} & a_{13} \\
    a_{21} & a_{22} & a_{23} \\
    a_{31} & a_{32} & a_{33}
\end{pmatrix}
\]
\end{lstlisting}

\begin{lstlisting}[language={python}]
\[ A_{\mathrm{Vmatrix}} = 
\begin{Vmatrix}
    a_{11} & a_{12} & a_{13} \\
    a_{21} & a_{22} & a_{23} \\
    a_{31} & a_{32} & a_{33}
\end{Vmatrix}
\]
\end{lstlisting}



\newpage

\subsubsection{special mark}

\paragraph{mark key point}~{}

\begin{enumerate}[topsep=0pt,itemsep=0pt,parsep=0pt,leftmargin=3.6em,label=\arabic*>]
    \item \verb!\underline{xxx}! \par
        \verb!\underline{abc}! \ColorStr{red}{=>} \underline{abc}
    \item \verb!\emph{xxx}! \par
        \verb!\emph{abc}! \ColorStr{red}{=>} \emph{abc}
    \item \verb!\uuline{xxx}! \par
        \verb!\uuline{abc}! \ColorStr{red}{=>} \uuline{abc}
    \item \verb!\uwave{xxx}! \par
        \verb!\uwave{abc}! \ColorStr{red}{=>} \uwave{abc}
    \item \verb!\sout{xxx}! \par
        \verb!\sout{abc}! \ColorStr{red}{=>} \sout{abc}
    \item \verb!\xout{xxx}! \par
        \verb!\xout{abc}! \ColorStr{red}{=>} \xout{abc}
    \item \verb!\dashuline{xxx}! \par
        \verb!\dashuline{abc}! \ColorStr{red}{=>} \dashuline{abc}
    \item \verb!\dotuline{xxx}! \par
        \verb!\dotuline{abc}! \ColorStr{red}{=>} \dotuline{abc}
\end{enumerate}

\paragraph{todo list}~{}

\begin{enumerate}[topsep=0pt,itemsep=0pt,parsep=0pt,leftmargin=3.6em,label=\arabic*>]
    \item \verb!$\square$! \ColorStr{red}{=>} $\square$
    \item \verb!$\boxtimes$! \ColorStr{red}{=>} $\boxtimes$
    \item \verb!$\surd$! \ColorStr{red}{=>} $\surd$
\end{enumerate}

\begin{enumerate}[topsep=0pt,itemsep=0pt,parsep=0pt,leftmargin=3.6em,label=\arabic*>]
    \item $\square$《笔记本》- 笔记本
    \item $\surd$《\sout{笔记本}》- 笔记本
    \item $\boxtimes$\sout{《笔记本》- 笔记本} $\surd$\ColorStr{red}{(done)}
    \item $\square$《笔记本》- 笔记本
\end{enumerate}

\paragraph{roman numerals}~{}

\begin{lstlisting}[language={python}]
I       `-`> 1                          VI      `-`> 6
II      `-`> 2                          VII     `-`> 7
III     `-`> 3                          VIII    `-`> 8
IV      `-`> 4                          IX      `-`> 9
V       `-`> 5                          X       `-`> 10
\end{lstlisting}





\newpage

\subsubsection{special symbols}

\url{https://www.latexlive.com/help#d13}

\url{http://mohu.org/info/symbols/symbols.htm}

\url{https://www.caam.rice.edu/~heinken/latex/symbols.pdf}

特殊符号都是在 \$ xxx \$ 中输入的,或者在输入公式的其他方式中输入,{\color{DefinedColorRed}eg. \$\textbackslash 
bar\{x\}\$}

\newcommand\PrintCode[2]{\textcolor{DefinedColorGreen}{\textbackslash #1\{#2\}}}
\newcommand\PrintSymbol[1]{\textcolor{red}{$#1$}}
\newcommand\SpecCharPair[3]{\PrintSymbol{#1} & \PrintCode{#2}{#3}}

\paragraph{math related}~{}

\begin{longtable}{|c|m{10em}|c|m{10em}|c|m{10em}|}
    \hline
    \SpecCharPair{\sin\theta}{sin}{\textbackslash theta} &
    \SpecCharPair{\cos{\theta}}{cos}{\textbackslash theta} &
    \SpecCharPair{\cosh{h}}{cosh}{h} \\
    \hline
    \SpecCharPair{\sinh g}{sinh}{g} &
    \SpecCharPair{\min{L}}{min}{L} &
    \SpecCharPair{\max{H}}{max}{H} \\
    \hline
    \SpecCharPair{\sup{t}}{sup}{t} &
    \SpecCharPair{\ker{x}}{ker}{x} &
    \SpecCharPair{\Pr{x}}{Pr}{x} \\
    \hline
    \SpecCharPair{\arctan{\frac{L}{T}}}{arctan}{\textbackslash frac\{L\}\{T\}} &
    \SpecCharPair{\arcsin\frac{L}{r}}{arcsin}{\textbackslash frac\{L\}\{r\}} &
    \SpecCharPair{\arccos{\frac{T}{r}}}{arccos}{\textbackslash frac\{T\}\{r\}} \\
    \hline
    \SpecCharPair{\operatorname{argch}l}{operatorname}{\{argch\}l} &
    \SpecCharPair{\operatorname{argsh}k}{operatorname}{\{argsh\}k} &
    \SpecCharPair{\operatorname{th}i}{operatorname}{\{th\}i} \\
    \hline
    \SpecCharPair{\operatorname{ch}h}{operatorname}{\{ch\}h} &
    \SpecCharPair{\operatorname{argth}m}{operatorname}{\{argth\}m} &
    \SpecCharPair{\arg{x}}{arg}{x} \\
    \hline
    \SpecCharPair{\tan{theta}}{tan}{theta} &
    \SpecCharPair{\tanh{i}}{tanh}{i} &
    \SpecCharPair{\limsup{S}}{limsup}{S} \\
    \hline
    \SpecCharPair{\liminf{I}}{liminf}{I} &
    \SpecCharPair{\lg{X}}{lg}{X} &
    \SpecCharPair{\inf{s}}{inf}{s} \\
    \hline
    \SpecCharPair{\exp{\!t}}{exp}{\textbackslash !t} &
    \SpecCharPair{\ln{X}}{ln}{X} &
    \SpecCharPair{\log{X}}{log}{X} \\
    \hline
    \SpecCharPair{\deg{x}}{deg}{x} &
    \SpecCharPair{\det{x}}{det}{x} &
    \SpecCharPair{\dim{x}}{dim}{x} \\
    \hline
    \SpecCharPair{\log{_\alpha{x}}}{log}{\_\textbackslash alpha\{x\}} &
    \SpecCharPair{\hom{x}}{hom}{x} &
    \SpecCharPair{\lim{_{t\to n}T}}{lim}{\_\{t \textbackslash to n\}T} \\
    \hline
\end{longtable}

\begin{longtable}{|c|m{9em}|c|m{9em}|c|m{9em}|c|m{9em}|}
    \hline
    \SpecCharPair{\nabla}{nabla}{} &
    \SpecCharPair{\partial{x}}{partial}{x} &
    \SpecCharPair{\mathrm{d}{x}}{mathrm\{d\}}{x} \\
    \hline
\end{longtable}

\paragraph{relation symbols}~{}
\paragraph{binary operations}~{}
\paragraph{set and/or logic notation}~{}
\paragraph{delimiters}~{}
\paragraph{greek letters}~{}


\begin{longtable}{|c|m{9em}|c|m{9em}|c|m{9em}|c|m{9em}|}
    \hline
    \SpecCharPair{\alpha}{}{alpha} &
    \SpecCharPair{\beta}{}{beta} &
    \SpecCharPair{\gamma}{}{gamma} &
    \SpecCharPair{\delta}{}{delta} \\
    \hline
    \SpecCharPair{\epsilon}{}{epsilon} &
    \SpecCharPair{\varepsilon}{}{varepsilon} &
    \SpecCharPair{\zeta}{}{zeta} &
    \SpecCharPair{\eta}{}{eta} \\
    \hline
    \SpecCharPair{\theta}{}{theta} &
    \SpecCharPair{\vartheta}{}{vartheta} &
    \SpecCharPair{\iota}{}{iota} &
    \SpecCharPair{\kappa}{}{kappa} \\
    \hline
    \SpecCharPair{\lambda}{}{lambda} &
    \SpecCharPair{\mu}{}{mu} &
    \SpecCharPair{\nu}{}{nu} &
    \SpecCharPair{\xi}{}{xi} \\
    \hline
    \PrintSymbol{o} & \textcolor{DefinedColorGreen}{o} &
    \SpecCharPair{\pi}{}{pi} &
    \SpecCharPair{\varpi}{}{varpi} &
    \SpecCharPair{\rho}{}{rho} \\
    \hline
    \SpecCharPair{\varrho}{}{varrho} &
    \SpecCharPair{\sigma}{}{sigma} &
    \SpecCharPair{\varsigma}{}{varsigma} &
    \SpecCharPair{\tau}{}{tau} \\
    \hline
    \SpecCharPair{\upsilon}{}{upsilon} &
    \SpecCharPair{\phi}{}{phi} &
    \SpecCharPair{\varphi}{}{varphi} &
    \SpecCharPair{\chi}{}{chi} \\
    \hline
    \SpecCharPair{\psi}{}{psi} &
    \SpecCharPair{\omega}{}{omega} &
    \SpecCharPair{\Gamma}{}{Gamma} &
    \SpecCharPair{\Delta}{}{Delta} \\
    \hline
    \SpecCharPair{\Theta}{}{Theta} &
    \SpecCharPair{\Lambda}{}{Lambda} &
    \SpecCharPair{\Xi}{}{Xi} &
    \SpecCharPair{\Pi}{}{Pi} \\
    \hline
    \SpecCharPair{\Sigma}{}{Sigma} &
    \SpecCharPair{\Upsilon}{}{Upsilon} &
    \SpecCharPair{\Phi}{}{Phi} &
    \SpecCharPair{\Psi}{}{Psi} \\
    \hline
    \SpecCharPair{\Omega}{}{Omega} \\
    \cline{1-2}
\end{longtable}

\paragraph{other symbols}~{}

\begin{longtable}{|c|m{9em}|c|m{9em}|c|m{9em}|c|m{9em}|}
    \hline
    \SpecCharPair{\bar{x}}{bar}{x} &
    \SpecCharPair{\acute{\eta}}{acute}{\textbackslash eta} &
    \SpecCharPair{\grave{\eta}}{grave}{\textbackslash eta} &
    \SpecCharPair{\check{\alpha}}{check}{\textbackslash alpha} \\
    \hline
    \SpecCharPair{\breve{a}}{breve}{a} &
    \SpecCharPair{\dot{x}}{dot}{x} &
    \SpecCharPair{\hat{\alpha}}{hat}{\textbackslash alpha} &
    \SpecCharPair{\ddot{y}}{ddot}{y} \\
    \hline
    \SpecCharPair{\tilde{\iota}}{tilde}{\textbackslash iota} \\
    \cline{1-2}
\end{longtable}

\paragraph{trigonometric functions}~{}



\subsubsection{shaded 设置段落阴影}

使用 framed package 的 shaded 环境来进行设置,这个阴影环境支持跨页,颜色设置在导言区 \textcolor{red}{eg. \textbackslash 
colorlet\{shadecolor\}\{gray!20\}};并使用 \textcolor{red}{eg. \textbackslash vspace\{-0.8em\}} 来进行前后间距设置

\begin{lstlisting}[language={sh}]
\vspace{`-`0.8em}
\begin{shaded}
a b c d ... 等具体内容
\end{shaded}
\vspace{`-`0.8em}
\end{lstlisting}

\vspace{-0.8em}
\begin{shaded}
在实际的网络中存在如下图所示的情况,因此实际考虑有 “quantize-op” 与 relu 或者 conv 时,
是与 “quantize-op” 之前的 relu 融合还是与 “quantize-op” 之后的 conv 融合的问题
\end{shaded}
\vspace{-0.8em}

\subsubsection{footnote}

\begin{lstlisting}[language={python}]
AAA \footnote{AAA 的注脚内容}
\end{lstlisting}

\subsubsection{embeded image}

\begin{lstlisting}[language={sh}]
\begin{figure}[htbp!]
\setlength{\belowcaptionskip}{`-`1.3em}
\centering
\includegraphics[scale=0.68]{`{\color{red}images/aaa.jpg}`}
\caption{aaa}
\label{fig:aaa}
\end{figure}
\end{lstlisting}


\subsubsection{section}

章节标题的编码方式

\begin{lstlisting}[language={python}]
\chapter{name}
\section{name}
\subsection{name}
\subsubsection{name}

\paragraph{name}~{}
\subparagraph{name}~{}

\subparagraph*{name}~{} # 添加一个 '*' 不生成标题 
\end{lstlisting}






\subsection{other}

%%\subsubsection{chen shuo config}
%%这个是陈硕的 latex 常用设置,其中的反斜杠使用正斜杠替代,这样方便显示
%%
%%https://blog.csdn.net/solstice/article/details/638
%%
%%\paragraph{段首缩进}~{}
%%
%%中文习惯在段首缩进两格,在LaTeX中,/parindent 表示段首缩进的长度,我们将它设置为当前字号的两个
%%大写字母M的宽度,大约正好是两个汉字的宽度:{\color{DefinedColorRed}/setlength\{/parindent\}\{2em\}}
%%
%%LaTeX 默认每节的第一段的段首不缩进,这不符合中文排版习惯。我们希望正文的每一段都要缩进,使用
%%indentfirst宏包就可办到:{\color{DefinedColorRed}/usepackage\{indentfirst\}}
%%
%%\paragraph{段距行距}~{}
%%
%%LaTeX 用/baselineskip表示当前的行距,其默认值大约是当前字号的1.2倍,如果当前字号是10pt,
%%那么/baselineskip是12pt。这对英文排版是合适的,对中文就显得太拥挤了,因为英文正文多为小
%%写字母,字高与小写的x差不多(即1ex)。如果字号为10pt,那么1ex =4.3pt。我通常把行距设为字
%%号的1.8倍:
%%
%%{\color{DefinedColorRed}/setlength\{/baselineskip\}\{1.8em\}}
%%
%%这个值随时可以改,对更改点以后的文字有效
%%
%%LaTeX 用/parskip表示段距,我一般把它设为1ex:{\color{DefinedColorRed}/setlength\{/parskip\}\{1ex\}}
%%
%%注意这些修改长度的命令最好都放在正文区(即/begin\{document\}之后)
%%
%%\paragraph{页眉页脚}~{}
%%
%%/documentclass[10pt, a4paper]{book}
%%/usepackage{fancyhdr}
%%
%%在 LaTeX 中先把 page style 设为fancy,再设置这个style中的页眉和页脚。但是它默认每章的
%%第一页的page style是plain,需要单独处理。
%%
%%\begin{lstlisting}[language={matlab}]
%%% 设置 plain style 的属性
%%/fancypagestyle{plain}{%
%%/fancyhf{}                    % 清空当前设置
%%
%%% 设置页眉 (head)
%%/fancyhead[RE]{/leftmark}     % 在偶数页的右侧显示章名
%%/fancyhead[LO]{/rightmark}    % 在奇数页的左侧显示小节名
%%/fancyhead[LE,RO]{~/thepage~} % 在偶数页的左侧,奇数页的右侧显示页码
%%
%%% 设置页脚:在每页的右下脚以斜体显示书名
%%/fancyfoot[RO,RE]{/it Typesetting with /LaTeX}
%%
%%/renewcommand{/headrulewidth}{0.7pt} % 页眉与正文之间的水平线粗细
%%/renewcommand{/footrulewidth}{0pt}
%%}
%%
%%/pagestyle{fancy}             % 选用 fancy style
%%% 其余同 plain style
%%/fancyhf{}                   
%%/fancyhead[RE]{/leftmark}
%%/fancyhead[LO]{/rightmark}
%%/fancyhead[LE,RO]{~/thepage~}
%%/fancyfoot[RO,RE]{/it Typesetting with /LaTeX}
%%/renewcommand{/headrulewidth}{0.7pt}
%%/renewcommand{/footrulewidth}{0pt}
%%
%%% 设置章名和节名的显示方式
%%/renewcommand{/chaptermark}[1]{/markboth{~第~/thechapter~章~~~#1~}{}}
%%/renewcommand{/sectionmark}[1]{/markright{~/thesection~~#1~}{}}
%%\end{lstlisting}
%%
%%\paragraph{章节标题}~{}
%%
%%通常用titlesec宏包来设置正文中出现的章节标题的格式:
%%
%%/usepackage{titlesec}
%%
%%设置章名为右对齐,字号为/Huge,字型为黑体,章号用粗体,并设置间距:
%%
%%/titleformat{/chapter}{/flushright/Huge/hei}{{/bf /thechapter}}{0pt}{}
%%
%%/titlespacing{/chapter}{0pt}{-20pt}{25pt}
%%
%%设置节名的字号为/Large,字型为黑体,节号用粗体,并设置间距:
%%
%%/titleformat{/section}{/Large /hei}{{/bf /thesection/space}}{0pt}{}
%%
%%/titlespacing*{/section}{0pt}{1ex plus .3ex minus .2ex}{-.2ex plus .2ex}
%%
%%其中/hei的定义为:
%%
%%/newcommand{/hei}{/CJKfamily{hei}}
%%
%%\paragraph{纸张大小}~{}
%%
%%\begin{lstlisting}[language={matlab}]
%%/documentclass[10pt, b5paper]{report}
%%/usepackage[body={12.6cm, 20cm}, centering, dvipdfm]{geometry}
%%% 以上将版心宽度设为 12.6cm,高度 20cm,版心居中,且自动设置PDF文件的纸张大小
%%\end{lstlisting}
%%
%%\subsubsection{documentclass}
%%
%%latex 有三种文档类型(CTE 环境):ctexart、ctexrep、ctexbook
%%
%%优先选用 ctexrep 文档类型,因为这个支持到 chapter,且\textcolor{red}{目录}后面不会添加一个空白页
%%
%%而 book 类型是要考虑印刷时的奇偶页问题,所以会在\textcolor{red}{目录}后面在不满足 chapter 所在页是奇数页时会添加一个空白页,
%%这个是 book 规范设置
%%


\subsection{experience}

\begin{enumerate}[topsep=0pt,itemsep=0pt,parsep=0pt,leftmargin=3.6em,label=\arabic*>]
    \item enumerate 的 list 的 leftmargin 参考设置
    \begin{enumerate}[topsep=0pt,itemsep=0pt,parsep=0pt,leftmargin=1.8em]
        \item 最外层的\ {\color{DefinedColorRed}leftmargin = 3.6em}
        \item 嵌套第一层的\ {\color{DefinedColorRed}leftmargin = 1.8em},并且可以去掉 label 设置
    \end{enumerate}
    \item 在 lstlisting 环境中,如果需要对某行使用 latex 命令来进行特殊设置,比如高亮、斜体、粗体等
        需要在其逃逸符号\ {\color{DefinedColorRed}(反引号)}\ 中使用
\end{enumerate}





\newpage
\subsection{errors}

\begin{lstlisting}[language={c}]
Runaway argument?
{\376\377\0001\000.\0
! File ended while scanning use of \@@BOOKMARK.
<inserted text>
    \par
l.232 \begin{document}}
\end{lstlisting}

遇到这个错误是上一次编译中断后的 aux/out 文件的部分内容缺失,因此只需要删除 *.aux / *.out 文件之后重新编译即可



 % chapter two
%
\section{latex}

\textsl{\url{http://www.ctex.org/OnlineDocuments}}

\textsl{\url{http://www.ctex.org/documents/packages/layout/fancyhdr.htm}}

\url{http://cremeronline.com/LaTeX/minimaltikz.pdf} : Tikz 制图

\url{http://www.texample.net/tikz/}

\url{https://www.latexlive.com/} : 在线 tex formula 验证

\url{https://blog.csdn.net/lishoubox/article/details/6783726} : 这个基础知识写的不错

\url{https://mathpretty.com/10864.html} : 各种小技巧


\subsection{install}

\subsubsection{separate installation}

xelatex 可以编译出带目录的 pdf 文档 : (这两个安装包都很大,每个都几百M)

\begin{enumerate}[topsep=0pt,itemsep=0pt,parsep=0pt,leftmargin=3.6em,label=\arabic*>]
    \item sudo apt-get install texlive-xetex
    \item sudo apt-get install latex-cjk-all | chinese support
    \item sudo apt-get install texmaker | texmaker is tex editor
\end{enumerate}

\subsubsection{integrated installation}

推荐使用这种安装方式,且安装也比较简单

TeX Live Home Page : \url{http://tug.org/texlive/acquire-netinstall.html}

Texmaker Home Page : \url{http://www.xm1math.net/texmaker/}

\begin{enumerate}[topsep=0pt,itemsep=0pt,parsep=0pt,leftmargin=3.6em,label=\arabic*>]
    \item download texlive from TexLive HomePage : eg. texlive2019-20190410.iso
    \item sudo mount -o loop texlive2019-20190410.iso /media/iso
    \item cd /media/iso
    \item install texlive \par
        {\color{DefinedColorGreen}./install-tl}
    \item set .bashrc \par
        {\color{violet}export PATH=\verb!"!/usr/local/texlive/2019/bin/x86\_64-linux:\$PATH\verb!"!}
    \item set windows fonts, ref section \ref{sec:use-win-font-in-ubuntu}
\end{enumerate}



\subsection{basic}

\begin{enumerate}[topsep=0pt,itemsep=0pt,parsep=0pt,leftmargin=3.6em,label=\arabic*>]
    \item 输入一个反斜杠 : {\color{red}\ \verb|\textbackslash|} => ``\textbackslash"
    \item 输入大括号 : {\color{red}\ \verb|\{|} => ``\{"
    \item 输入百分号 : {\color{red}\ \verb|\%|} => ``\%"
    \item 输入一个空格 : ``\textbackslash\ " 或者 ``{ }"
    \item 输入\CmdFont{--} : \verb!\CmdFont{!\CmdFont{--}\verb!}! 会比较好看
    \item 中文标点(zh) ‘ | ; | ’ | , | 。| “ | ” \par
        \textcolor{teal}{就正常中文输入法中输入的标点符号,就可以直接显示为中文标点}
        {\color{red}(注意是 Calibri 字体的情况下才可以)}

    \item {\color{blue}Times New Roman 字体下的标点(en)}:\par
        \verb={\fontspec{Times New Roman}' | ' | " | " | , | ; | `}= 
        {\color{DefinedColorRed}$\Rightarrow$} % =>
        {\fontspec{Times New Roman}' | ' | " | " | , | ; | `} \par
        \verb={\fontspec{Times New Roman} ' "A" and "B" '}=
        {\color{DefinedColorRed}$\Rightarrow$} % =>
        {\fontspec{Times New Roman}' "A" and "B" '} \par
    \item {\color{blue}Calibri 字体下的标点(en)}{\color{red}(这种输入方式的字体会比较多)}:\par
        \verb={\fontspec{Calibri} ` | ' | `` | " | , | ; | \textasciigrave }=
        {\color{DefinedColorRed}$\Rightarrow$} % =>
        ` | ' | `` | " | , | ; | \textasciigrave \par
        \verb={\fontspec{Calibri} `\,``A" and ``B"\,' }=
        {\color{DefinedColorRed}$\Rightarrow$} % =>
        `\,``A" and ``B"\,'

    \item using symbol:\textbackslash symbol\{96\} {\color{DefinedColorRed}$\to$} \symbol{96} 、
        \textbackslash symbol\{39\} {\color{DefinedColorRed}$\to$} \symbol{39} \par
        通过使用 symbol 来输入字符,可以输入任意有编码的字符
    \item \^{}:input(\textbackslash \^{}\{\})
    \item \~{}:input(\textbackslash \~{}\{\}),在 \textbackslash CmdFont 中也可以使用这个方法来输入 \~{}
    \item \{\textbackslash songti 宋体\}:{\songti 宋体}
    \item \{\textbackslash kaishu 楷书\}:{\kaishu 楷书}
    \item \{\textbackslash heiti 黑体\}:{\heiti 黑体}
    \item \{\textbackslash fangsong 仿宋\}:{\fangsong 仿宋}
    \item \{\textbackslash fontspec\{YouYuan\} 幼圆\}:{\fontspec{YouYuan}幼圆}  % for win-font
    \item \{\textbackslash fontspec\{Consolas\} Consolas\}:{\fontspec{Consolas}Consolas}  % for win-font
        \textcolor{teal}{| 常规五号}
    \item \{\textbackslash zihao\{-5\}\textbackslash fontspec\{Consolas\} Consolas\}:
        {\zihao{-5}\fontspec{Consolas}Consolas}  % for win-font
        \textcolor{teal}{| 小五号}
    \item \{\textbackslash small\textbackslash fontspec\{Consolas\} Consolas\}:
        {\small\fontspec{Consolas}Consolas}  % for win-font
        \textcolor{teal}{| small 就是小五号}
    \item \textbackslash verb:eg. \verb'\verb!\songti!' 
        $\Rightarrow$ % =>
        \verb!\songti! \par
        \textcolor{red}{`!'} 表示环境符号,也可以用其他的符号替代 \par
        \verb={\color{teal}\verb!$\max_n$!}= => {\color{teal}\verb!$\max_n$!} : 给 \textbackslash verb 包围的上色
    \item 数字的下划线分割(使用公式环境):eg. \verb=0xf000$\underline\ $0000= 
        $\Rightarrow$ 
        0xf000$\underline\ $0000 \par
        {\color{teal}在 Calibri 字体下则可以直接使用\_,而不需要使用\textbackslash underline}
    \item url link:\par
        eg. \textbackslash url\{https://xxxx\} 
        $\Rightarrow$ % =>
        \url{https://xxxx} \par
        eg. \verb=\textsl{\url{/https://xxxx}}=
        $\Rightarrow$ % =>
        \textsl{\url{https://xxxx}} \par
        eg. \verb=\emph{\url{/https://xxxx}}=
        $\Rightarrow$ % =>
        \emph{\url{https://xxxx}} | \textcolor{teal}{在 Calibri 字体下会比较好看}
    \item \textbackslash vspace\{-0.8em\}:设置前后行的间距
    \item \textbackslash dotfill:绘制虚线分割线,可以查看 section \ref{sec:book-list} 中的使用效果
    \item \textbackslash hrulefill:绘制实线分割线
\end{enumerate}

\subsubsection{lstlisting / verbatim}

\paragraph{enumerate}~{}

\begin{lstlisting}[language={python}]
\begin{enumerate}

\begin{enumerate}[topsep=0pt,itemsep=0pt,parsep=0pt,leftmargin=3.6em,label=\arabic*>]
\begin{enumerate}[topsep=0pt,itemsep=0pt,parsep=0pt,leftmargin=3.6em,label=\arabic*.]
\begin{enumerate}[topsep=0pt,itemsep=0pt,parsep=0pt,leftmargin=3.6em,label=\arabic*)]
\begin{enumerate}[topsep=0pt,itemsep=0pt,parsep=0pt,leftmargin=3.6em,label=(\arabic*)]

# 大小写罗马数字
\begin{enumerate}[topsep=0pt,itemsep=0pt,parsep=0pt,leftmargin=3.6em,label=\roman*>]
\begin{enumerate}[topsep=0pt,itemsep=0pt,parsep=0pt,leftmargin=3.6em,label=\Roman*>]

# 大小写字母
\begin{enumerate}[topsep=0pt,itemsep=0pt,parsep=0pt,leftmargin=3.6em,label=\alph*>]
\begin{enumerate}[topsep=0pt,itemsep=0pt,parsep=0pt,leftmargin=3.6em,label=\Alph*>]

# 特殊符号
\begin{enumerate}[topsep=0pt,itemsep=0pt,parsep=0pt,leftmargin=3.6em,label=\fnsymbol*>]
\end{lstlisting}

\begin{lstlisting}[language={python}]
\begin{enumerate}[topsep=0pt,itemsep=0pt,parsep=0pt,leftmargin=3.6em,label=\arabic*>]
    \item 编号
        \arabic{enumi}, \roman{enumi}, \Roman{enumi}, 
        \alph{enumi}, \Alph{enumi}, \fnsymbol{enumi}
    \item 编号
        \arabic{enumi}, \roman{enumi}, \Roman{enumi}, 
        \alph{enumi}, \Alph{enumi}, \fnsymbol{enumi}
    \item 编号
        \arabic{enumi}, \roman{enumi}, \Roman{enumi}, 
        \alph{enumi}, \Alph{enumi}, \fnsymbol{enumi}
\end{enumerate}
\end{lstlisting}

\begin{enumerate}[topsep=0pt,itemsep=0pt,parsep=0pt,leftmargin=3.6em,label=\arabic*>]
    \item 编号
        \arabic{enumi}, \roman{enumi}, \Roman{enumi}, \alph{enumi}, \Alph{enumi}, \fnsymbol{enumi}
    \item 编号
        \arabic{enumi}, \roman{enumi}, \Roman{enumi}, \alph{enumi}, \Alph{enumi}, \fnsymbol{enumi}
    \item 编号
        \arabic{enumi}, \roman{enumi}, \Roman{enumi}, \alph{enumi}, \Alph{enumi}, \fnsymbol{enumi}
\end{enumerate}


\paragraph{lstlisting}~{}

\begin{lstlisting}[language={python}]
`\verb!\begin{lstlisting}[language={python}]!`
xxx
`\verb=\end{lstlisting}=`

`\verb!\begin{lstlisting}[language={c++}]!`
`\verb!\begin{lstlisting}[language={sh}]!`
\end{lstlisting}

可以在 lstlisting 环境中使用 \textbackslash verb 来输入各种特殊的字符,比如 {\color{red}\ \verb=`$CMD_A`=}

\begin{lstlisting}[language={python}]
`\verb!\verb=`$CMD_A`=!` `{\color{red}->}` `\verb=`$CMD_A`=`
\end{lstlisting}

lstlisting 在序列中的项时这么使用:

\begin{lstlisting}[language={python}]
    ...
    `\verb!\item!` a item
`\verb!\begin!`{lstlisting}[language={sh}, `{\color{red}xleftmargin=0em}`]
...
`\verb!\end!`{lstlisting}
    `\verb!\item!` other item
    ...
\end{lstlisting}



\paragraph{verbatim}~{}

缩略写法:\textbackslash verb=\textcolor{red}{xxx}= : \textcolor{red}{xxx} 是要原格式表示的内容

%\begin{verbatim}
%    \usepackage[T1]{fontenc}            % font styling
%    \usepackage{lmodern, mathrsfs}
%\end{verbatim}

\begin{lstlisting}[language={python}]
\makeatletter
\renewcommand*{\verbatim@font}{}
\makeatother
\end{lstlisting}

如果需要设置 \textbackslash verb 环境中的颜色,可以直接在外部包上 \textbackslash color\{\} 来进行设置即可

如果还需要在  \textbackslash verb 环境中设置字体,可以以如下方式来使用:

\begin{lstlisting}[language={c}]
{\small\ttfamily\fontspec{Consolas}\verb!grep `-`in "\`-`\`-`" ./ `-`R!}
\end{lstlisting}

%\begin{lstlisting}[language={python}]
%\verb={\fontspec{Times New Roman} abcdefgxxxx}= 
%\end{lstlisting}


\subsubsection{tabular、tabularx、array}

table to latex code : \url{https://www.tablesgenerator.com/}

\url{https://github.com/krlmlr/Excel2LaTeX}

文本或数学模式都可以使用 tabular,数学模式还可以使用 array 环境(即包含数学符号的公式)

table 浮动表格中的几个命令设置如下:

\begin{lstlisting}[language={tex}]
...
\setlength{\abovecaptionskip}{1.0em} %
\setlength{\belowcaptionskip}{`-`1.3em}% 这两个必须得放在 caption 前面,否则不起作用
\caption{mxnet 的基本概念}
...
\end{lstlisting}

\paragraph{tabular}~{}

tabular 中的参数设置: 
\begin{enumerate}[topsep=0pt,itemsep=0pt,parsep=0pt,leftmargin=3.6em,label=\arabic*>]
    \item \textbackslash raggedleft:表示左边不对齐,\textcolor{red}{即右对齐}
    \item \textbackslash raggedright:表示右边不对齐,\textcolor{red}{即左对齐}
    \item \textbackslash centering:表示居中
    \item p\{width value\}:设置列框,使用了列宽后,需在列内容处使用上述的 对其方式
    \item \textbackslash arraystretch\{val\}:设置行间距,默认值为 1
        eg. \textbackslash renewcommand\textbackslash arraystretch\{2\} | 默认值的 2 倍
    \item \textbackslash multicolumn:合并不同列 \par
        \textbackslash multicolumn\{列数\}\{格式\}\{内容\} \par
        \textbackslash multicolumn\{2\}\{c|\}\{内容\} : {\color{red}c 表示居中,| 表示边框} \par
        \textbackslash multicolumn\{2\}\{l|\}\{内容\} : left \par
        \textbackslash multicolumn\{2\}\{r|\}\{内容\} : right \par
        \textcolor{DefinedColorGreen}{如果只是针对 1 列操作时,相当于是修改对齐、竖线格式}
    \item \textbackslash multirow:跨不同行,类似于合并不同行 | \textbackslash usepackage\{multirow\}\par
        \textbackslash multirow\{行数\}\{宽度\}\{内容\} \par
        \textbackslash multirow\{行数\}*\{内容\} | 宽度由输入内容来决定 \par
        需要结合 \textbackslash cline{start - end} 来使用 eg. \verb!\cline{2-5}!
\end{enumerate}

\begin{lstlisting}[language={tex}]
\begin{table}[htbp!]
    \centering
    \setlength{\abovecaptionskip}{0.5em}    % 这个设置在 tabular 前面的话的参数值
    \setlength{\belowcaptionskip}{`-`0.5em}
    \caption{xxx}
    \label{tab:xxx}
\begin{tabular}{|l|l|}
...
\end{lstlisting}

\begin{lstlisting}[language={tex}]
...
\end{tabular}
    \centering
    \setlength{\abovecaptionskip}{1.0em}    % 设置在 tabular 后面的话的参数值
    \setlength{\belowcaptionskip}{`-`1.3em}
    \caption{xxx}
    \label{tab:xxx}
\end{table}
\end{lstlisting}


\begin{center}
%\renewcommand\arraystretch{2}
\begin{tabular}{|p{8em}|c|l|p{4em}|p{4em}|}
    \hline
    \raggedleft{\textbackslash raggedleft} & abc & abc & abc & abc \\
    \hline
    \raggedright{\textbackslash raggedright} & abc & abc & \multicolumn{2}{c|}{abc} \\
    \hline
    \centering{\textbackslash centering} & abc & abc & abc & abc \\
    \hline
    default & abc & abc & abc & abc \\
    \hline
    \multirow{2}*{multirow} & abc & abc & abc & abc \\

    %
    % 这里表示只需要对 2-5 列画横线 , \hline 表示从第 1 列到最后 1 列都画横线
    %
    \cline{2-5}
                            & abc & abc & abc & abc \\
    \hline
    default & abc & abc & abc & abc \\
    \hline
    \centering{\multirow{2}*{multirow}} & abc & abc & abc & abc \\
    \cline{2-5}
                            & abc & abc & abc & abc \\
    \hline
    \raggedleft{\multirow{2}*{multirow}} & abc & abc & abc & abc \\
    \cline{2-5}
                            & abc & abc & abc & abc \\
    \hline
\end{tabular}
\end{center}

\paragraph{tabularx}~{}

tabularx 环境是固定宽度的表格,这里的固定宽度指\textcolor{red}{铺满页面宽度}

%
% Y 是重新定义的格式,\arraybackslash 表示命令恢复
%
\newcolumntype{C}{>{\centering\arraybackslash}X}
\newcolumntype{R}{>{\raggedleft\arraybackslash}X}
\begin{tabularx}{46em}{|c|X|X|R|C|}
    \hline
    数字 & 1 & 2 & 3 & 4 \\
    \hline
    英文 & one & two & three & four \\
    \hline
\end{tabularx}

\paragraph{longtable}~{}

\begin{lstlisting}[language={tex}]
\begin{longtable}{|l|m{40em}|}
    \hline
    aaa   & xxx \\
    \hline
    bbb   & xxx \\
    \hline
    ccc   & xxx # 这样可以实现换行

          1> xxx

          2> xxx \\
    \hline
\end{longtable}
\end{lstlisting}

\begin{longtable}{|l|m{40em}|}
    \hline
    aaa   & xxx \\
    \hline
    bbb   & xxx \\
    \hline
    ccc   & xxx

          1> xxx

          2> xxx \\
    \hline
\end{longtable}

\paragraph{array | tabu}~{}

\begin{enumerate}[topsep=0pt,itemsep=0pt,parsep=0pt,leftmargin=3.6em,label=\arabic*>]
    \item 列格式控制:\textbackslash usepackage\{array\}
    \item 行格式控制:\textbackslash usepackage\{tabu\}
\end{enumerate}

\subparagraph{array}~{}

array 新增的基本命令(\verb=\usepackage{array}=):
\begin{enumerate}[topsep=0pt,itemsep=0pt,parsep=0pt,leftmargin=3.6em,label=\arabic*>]
    \item \verb=m{宽}=:类似 p 格式,产生具有固定宽度的列,并可以自动换行,垂直方向居中对齐 \par
        这个 m 格式在 tabular 、longtable 中都是可以使用的
    \item \verb=b{宽}=:类似 p 格式,垂直方向与最后一行对齐
    \item \verb=>{xx}=:把 xx 插入后面一列的开头
    \item \verb=<{xx}=:把 xx 插入前面一列的末尾
    \item \verb=!{xx}=:把 xx 作为表格线处理,相当于使用了 \verb=@{xx}= 但左右两边会有额外的间距 \par
        eg. \verb=c!{$\Rightarrow$}=
\end{enumerate}

格式符 > 与 < 通常用来设置整列的格式,例如改变一列表格的字体或者使用数学模式

\begin{tabular}{>{\bfseries}c|>{\itshape}c>{$}c<{$}}
    \hline
    姓名 & \textnormal{得分} & \multicolumn{1}{c}{额外加分} \\
    \hline
    A & 85 & +7 \\
    B & 82 & 0 \\
    C & 70 & -2 \\
    \hline
\end{tabular}

%
% 这个可以自动居中
%
\begin{tabular}{|>{$}r<{$}|>{\setlength\parindent{2em}}m{35em}|>{\centering\arraybackslash}m{4em}|}
    \hline
    \pi & 希腊字母,希腊字母希腊字母希腊字母希腊字母希腊字母希腊字母希腊字母希腊字母 & 常用 \\
    \hline
    \pi & 希腊字母,希腊字母希腊字母希腊字母希腊字母希腊字母希腊字母希腊字母希腊字母 & 常用 \\
    \hline
\end{tabular}

array 宏包提供的\verb=\newcolumntype= 用于定义新的列格式的命令:
\begin{enumerate}[topsep=0pt,itemsep=0pt,parsep=0pt,leftmargin=3.6em,label=\arabic*>]
    \item \verb=\newcolumntype{M}{>{$}c<{$}}=
    \item \verb=\newcolumntype{P}[1]{>{\setlength\parindent{2em}}p{#1}}=
    \item \verb=\newcolumntype{C}[1]{>{\centering\arraybackslash}m{#1}}= \par
        eg. \verb=\begin{tabular}{|M|P{15em}|C{4em}|}=
\end{enumerate}

\subparagraph{tabu}~{}

tabu 宏包提供 \textbackslash rowfont 命令来设置行的格式

\begin{tabu}{ccc}
    \hline
    \rowfont{\bfseries} 姓名 & 得分 & \multicolumn{1}{c}{额外加分} \\
    \hline
    \rowfont{\color{blue}} A & 85 & +7 \\
    \rowfont{\color{red}} B & 82 & 0 \\
    \rowfont{\color{green}} C & 70 & -2 \\
    \hline
\end{tabu}

\paragraph{threeparttable}~{}

\begin{enumerate}[topsep=0pt,itemsep=0pt,parsep=0pt,leftmargin=3.6em,label=\arabic*>]
    \item \verb=usepackage{threeparttable}=
    \item \verb=\tnote{val}=:使用这个来定义表格中的角标 eg.\par
        \textcolor{red}{\textbackslash tnote\{2\}} \par
        \textcolor{red}{\textbackslash tnote\{a\}} \par
        \textcolor{red}{\textbackslash tnote\{I\}} \par
        \textcolor{red}{\textbackslash tnote\{*\}}
\end{enumerate}

\begin{center}
\begin{threeparttable} 
    \centering 
    \caption{threeparttable example} 
    \label{tbl:threeparttable example} 
    \begin{tabular}{l|cc|cc}
        \hline 
        $F_{1}$ &$3.70e-03$ &$4.20e-03$\tnote{2} &$ \textbf{3.70e-03}$ &$3.90e-03$\\
        $F_{2}$ &$3.60e-03$\tnote{a} &$1.55e-02$ &$\textbf{6.70e-03}$ &$8.50e-03$\\ 
        \hline
    \end{tabular} 
    \begin{tablenotes} 
    \item[1] The bolder ones mean better. 
    \item[2] my note
    \item[a] my note a
    \end{tablenotes} 
\end{threeparttable} 
\end{center}



\subsubsection{set color}

\begin{enumerate}[topsep=0pt,itemsep=0pt,parsep=0pt,leftmargin=3.6em,label=\arabic*>]
    \item \{\textbackslash color\{blue\}\{蓝色字\}\}:{\color{blue}{蓝色字}}
    \item \textbackslash textcolor\{red\}\{红色字体\}:\textcolor{red}{红色字体}
    \item \textbackslash colorbox\{yellow\}\{黄色盒子\}:\colorbox{yellow}{黄色盒子}
    \vspace{-0.4em}
    \item \textbackslash fcolorbox\{black\}\{green\}\{黑框绿色背景盒子\}:\fcolorbox{black}{green}{黑框绿色背景盒子}
\end{enumerate}

\begin{longtable}{cp{5em}|cp{5em}|cp{5em}|cp{5em}|cp{5em}|cp{5em}}
    \hline
    \colorbox{black}{\textcolor{black}{color}} & \multicolumn{1}{c|}{black} &
    \colorbox{gray}{\textcolor{gray}{color}} & \multicolumn{1}{c|}{gray} &
    \colorbox{olive}{\textcolor{olive}{color}} & \multicolumn{1}{c|}{olive} &
    \colorbox{blue}{\textcolor{blue}{color}} & \multicolumn{1}{c|}{blue} &
    \colorbox{green}{\textcolor{green}{color}} & \multicolumn{1}{c|}{green} &
    \colorbox{teal}{\textcolor{teal}{color}} & \multicolumn{1}{c}{teal} \\
    \hline
    \colorbox{violet}{\textcolor{violet}{color}} & \multicolumn{1}{c|}{violet} &
    \colorbox{brown}{\textcolor{brown}{color}} & \multicolumn{1}{c|}{brown} &
    \colorbox{lightgray}{\textcolor{lightgray}{color}} & \multicolumn{1}{c|}{lightgray} &
    \colorbox{pink}{\textcolor{pink}{color}} & \multicolumn{1}{c|}{pink} &
    \colorbox{white}{\textcolor{white}{color}} & \multicolumn{1}{c|}{white} &
    \colorbox{lime}{\textcolor{lime}{color}} & \multicolumn{1}{c}{lime} \\
    \hline
    \colorbox{purple}{\textcolor{purple}{color}} & \multicolumn{1}{c|}{purple} &
    \colorbox{yellow}{\textcolor{yellow}{color}} & \multicolumn{1}{c|}{yellow} &
    \colorbox{darkgray}{\textcolor{darkgray}{color}} & \multicolumn{1}{c|}{darkgray} &
    \colorbox{magenta}{\textcolor{magenta}{color}} & \multicolumn{1}{c|}{magenta} &
    \colorbox{red}{\textcolor{red}{color}} & \multicolumn{1}{c|}{red} &
    \colorbox{orange}{\textcolor{orange}{color}} & \multicolumn{1}{c}{orange} \\
    \hline
    \colorbox{cyan}{\textcolor{cyan}{color}} & \multicolumn{1}{c|}{cyan} \\
    \hline
\end{longtable}

\begin{longtable}{cp{6em}|cp{6em}|cp{6em}|cp{6em}|cp{6em}}
    \hline
    \colorbox{Yellow}{\textcolor{Yellow}{color}} & \multicolumn{1}{c|}{Yellow} &
    \colorbox{GreenYellow}{\textcolor{GreenYellow}{color}} & \multicolumn{1}{c|}{GreenYellow} &
    \colorbox{RubineRed}{\textcolor{RubineRed}{color}} & \multicolumn{1}{c|}{RubineRed} &
    \colorbox{RoyalPurple}{\textcolor{RoyalPurple}{color}} & \multicolumn{1}{c|}{RoyalPurple} &
    \colorbox{Emerald}{\textcolor{Emerald}{color}} & \multicolumn{1}{c}{Emerald} \\
    \hline
    \colorbox{WildStrawberry}{\textcolor{WildStrawberry}{color}} & \multicolumn{1}{c|}{WildStrawberry} &
    \colorbox{BlueViolet}{\textcolor{BlueViolet}{color}} & \multicolumn{1}{c|}{BlueViolet} &
    \colorbox{Emerald}{\textcolor{Emerald}{color}} & \multicolumn{1}{c}{Emerald} \\
    \hline
\end{longtable}

\begin{enumerate}[topsep=0pt,itemsep=0pt,parsep=0pt,leftmargin=3.6em,label=\arabic*>]
    \item \textbackslash columncolor:eg. \verb!\begin{tabular}{>{\columncolor{gray}}c >{\columncolor{red}}c}!
    \item \textbackslash rowcolor:eg. \verb!\rowcolor{red} A & B & C \\!
    \item \verb=\setlength{\arrayrulewidth}{0.19mm}=:设置表格横线与竖线的粗细,实验下来设置 0.19 mm 最好,
        设置颜色时不会把线给遮挡了,而且线条显示也均匀 \par
        可以把这个参数设置到\textcolor{red}{导言区}
    \item \verb=\arrayrulecolor{green}= 用来设置表格线条为\textcolor{green}{绿色} \par
        后面的表格如果想要恢复成黑色,需要再重新设置一下
\end{enumerate}


\arrayrulecolor{green}
\begin{tabular}{|l|c|r|}
    \arrayrulecolor{black}\hline
    United Kingdom & London & Thames\\
    \arrayrulecolor{blue}\hline
    France & Paris & Seine \\
    \arrayrulecolor{black}\cline{1-1}
    \arrayrulecolor{red}\cline{2-3}
    Russia & Moscow & Moskva \\ \hline
\end{tabular}


\arrayrulecolor{black}
\begin{tabular}{|>{\columncolor{lime}[6pt][6pt]}c|l|l|r|c|}
    \hline
    数字 & 1 & 2 & 3 & \cellcolor{red}4 \\
    \hline
    英文 & one & two & three & four \\
    \hline
    \rowcolor{lightgray}[\tabcolsep][\tabcolsep] 汉文 & 一 & 二 & 三 & 四 \\
    \hline
\end{tabular}




\subsubsection{define command and environment}

\paragraph{define command}~{}

使用 \verb=\newcommand<命令>[<参数个数>][<首参数默认值>]{<具体定义>}= 来定义一个新的命令
\begin{enumerate}[topsep=0pt,itemsep=0pt,parsep=0pt,leftmargin=3.6em,label=\arabic*>]
    \item 无参数定义:\par
        \verb=\newcommand<命令>{<具体定义>}= 
    \item 有参数定义:\par
        参数编号使用\verb=#1、#2、#3、...、#9=
\end{enumerate}

以下是一些定义实例:

\newcommand\PRCa{\emph{China}}
\newcommand\PRCb{\textcolor{red}{\emph{China}}}
\newcommand\PRCc[1]{\textcolor{blue}{\textcolor{red}{#1} country is \emph{China}}}
\newcommand\PRCd[1][my]{\textcolor{blue}{\textcolor{red}{#1} country is \emph{China}}}

\begin{enumerate}[topsep=0pt,itemsep=0pt,parsep=0pt,leftmargin=3.6em,label=\arabic*>]
    \item \verb=\newcommand\PRCa{\emph{China}}= \par
        \verb=\PRCa= $\Rightarrow$ \PRCa
    \item \verb=\newcommand\PRCb{\textcolor{red}{\emph{China}}}= \par
        \verb=\PRCb= $\Rightarrow$ \PRCb
    \item \verb=\newcommand\PRCc[1]{\textcolor{blue}{\textcolor{red}{#1} country is \emph{China}}}= \par
        \verb=\PRCc{my}= $\Rightarrow$ \PRCc{my}
    \item \verb=\newcommand\PRCd[1][my]{\textcolor{blue}{\textcolor{red}{#1} country is \emph{China}}}= \par
        \verb=\PRCd= $\Rightarrow$ \PRCd \par
        \verb=\PRCd[your]= $\Rightarrow$ \PRCd[your] \par
\end{enumerate}

\paragraph{renew command}~{}

使用 \verb=\renewcommand<命令>[<参数个数>][<首参数默认值>]{<具体定义>}= 来重定义命令

\paragraph{new/renew environment}~{}



\subsubsection{font set}

在 linux 中 latex 字体如果想使用 win 的字体,需要先把 win 的字体拷贝到 linux 中,具体操作可以参考 
\ref{sec:use-win-font-in-ubuntu}

在导言区设置如下,就可以使用 win 的字体:
\begin{lstlisting}[language={matlab}]
\usepackage{fontspec}
%\setmainfont{Times New Roman}   % 这些字体可以用 fc`-`list 中查看名字
\setmainfont{Calibri}
\setsansfont{Verdana}
%\setmonofont{Courier New}
%\setmonofont{Consolas}
\end{lstlisting}

\begin{lstlisting}[language={sh}]
minphone@mcp:~$fc`-`list
/usr/share/fonts/windows_fonts/BROADW.TTF: `\textcolor{red}{Broadway}`:style=Regular,obyčejné
/usr/share/fonts/windows_fonts/consola.ttf: `\textcolor{red}{Consolas}`:style=Regular
/usr/share/fonts/windows_fonts/constanb.ttf: `\textcolor{red}{Constantia}`:style=Bold
/usr/share/fonts/windows_fonts/ONYX.TTF: `\textcolor{red}{Onyx}`:style=Regular,obyčejné
\end{lstlisting}

{\color{red}按以下方式来使用字体:}
\begin{enumerate}[topsep=0pt,itemsep=0pt,parsep=0pt,leftmargin=3.6em,label=\arabic*>]
    \item \{\textbackslash fontspec\{Microsoft YaHei Light\} 微软雅黑 Light\}:
        {\fontspec{Microsoft YaHei Light}微软雅黑 Light}  % for win-font
    \item \{\textbackslash fontspec\{YouYuan\} 幼圆\}:{\fontspec{YouYuan}幼圆}  % for win-font
    \item \{\textbackslash fontspec\{Comic Sans MS\} Comic Sans MS\}:{\fontspec{Comic Sans MS}Comic Sans MS}  % for win-font
    \item \{\textbackslash small\textbackslash fontspec\{Comic Sans MS\} Comic Sans MS\}
        {\small\fontspec{Comic Sans MS}Comic Sans MS}  % for win-font

    \item \{\textbackslash fontsize\{8pt\}\{0\}\textbackslash fontspec\{Comic Sans MS\} Comic Sans MS\}
        {\fontsize{8pt}{0}\fontspec{Comic Sans MS}Comic Sans MS}  % for win-font

    \item \{\textbackslash small\textbackslash fontspec\{Consolas\} Consolas\}:
        {\small\fontspec{Consolas}Consolas}  % for win-font
        | small {\color{teal}就是小五号}
    \item \verb=\textup{upright shape}= : \textup{upright shape}
    \item \verb=\textit{italic shape}= : \textit{italic shape}
    \item \verb=\textsl{slanted shape}= : \textsl{slanted shape}
    \item \verb=\textsc{small capitals shape}= : \textsc{small capticals shape}
    \item \verb=\textbf{apple}= : \textbf{apple}
\end{enumerate}

\paragraph{set font size}~{}

\begin{enumerate}[topsep=0pt,itemsep=0pt,parsep=0pt,leftmargin=3.6em,label=\arabic*>]
    \item \verb=\tiny{Text}= : {\tiny{Text}}
    \item \verb=\small{Text}= : {\small{Text}}
    \item \verb=\normalsize{Text}= : {\normalsize{Text}}
    \item \verb=\large{Text}= : {\large{Text}}
    \item \verb=\huge{Text}= : {\huge{Text}}
    \item \verb={\fontsize{8pt}{0}Text}= : {\fontsize{8pt}{0}Text}
\end{enumerate}

各个字号的榜值 : \url{https://blog.csdn.net/weixin_39679367/article/details/115794548}

\paragraph{font style of math}~{}

\begin{enumerate}[topsep=0pt,itemsep=0pt,parsep=0pt,leftmargin=3.6em,label=\arabic*>]
    \item \verb=$\mathnormal{ABCDEF, abcdef, 123456}$= : {$\mathnormal{ABCDEF, abcdef, 123456}$}
    \item \verb=$\mathnormal{123456}$= : {$\mathnormal{123456}$}
    \item \verb=$\mathrm{ABCDEF, abcdef, 123456}$= : {$\mathrm{ABCDEF, abcdef, 123456}$}
    \item \verb=$\mathit{ABCDEF, abcdef, 123456}$= : {$\mathit{ABCDEF, abcdef, 123456}$}
    \item \verb=$\mathbf{ABCDEF, abcdef, 123456}$= : {$\mathbf{ABCDEF, abcdef, 123456}$}
    \item \verb=$\mathsf{ABCDEF, abcdef, 123456}$= : {$\mathsf{ABCDEF, abcdef, 123456}$}
    \item \verb=$\mathtt{ABCDEF, abcdef, 123456}$= : {$\mathtt{ABCDEF, abcdef, 123456}$}
    \item \verb=$\mathcal{ABCDEF}$= : {$\mathcal{ABCDEF}$}
\end{enumerate}



\newpage

\subsubsection{formula | \$ xxx \$}

\begin{enumerate}[topsep=0pt,itemsep=0pt,parsep=0pt,leftmargin=3.6em,label=\arabic*>]
    \item {\color{teal}\verb!$a \dots b$!} => $a \dots b$
    \item {\color{teal}\verb!$a \vdots b$!} => $a \vdots b$
    \item {\color{teal}\verb!$a \ddots b$!} => $a \ddots b$
    \item {\color{teal}\verb!$\{aaa\}$!} => $\{aaa\}$
    \item {\color{teal}\verb!$[aaa]$!} => $[aaa]$
    %\item {\color{teal}\verb!$a \iddots b$!} => $a \iddots b$
\end{enumerate}

\paragraph{上标/下标}~{}

\begin{enumerate}[topsep=0pt,itemsep=0pt,parsep=0pt,leftmargin=3.6em,label=\arabic*>]
    \item {\color{teal}\verb!$x_{1}$!} => $x_{1}$
    \item {\color{teal}\verb!$x^{2}$!} => $x^{2}$
    \item {\color{teal}\verb!$A_{ij} = 2^{i+j}$!} => $A_{ij} = 2^{i+j}$
    \item {\color{teal}\verb!$A\prime$!} => $A\prime$
    \item {\color{teal}\verb!$A'$!} => $A'$
    \item {\color{teal}\verb!$A''$!} => $A''$
    \item {\color{teal}\verb!$A = 90^\circ$!} => $A = 90^\circ$
    \item {\color{teal}\verb!$\max_n f(n)$!} => $\max_n f(n)$
    \item {\color{teal}\verb!$A_m{}^n$!} => $A_m{}^n$
    \item {\color{teal}\verb!$\int_0^1 f(t)$!} => $\int_0^1 f(t)$
    \item {\color{teal}\verb!$\iint_0^1 f(t)$!} => $\iint_0^1 f(t)$
    \item {\color{teal}\verb!$\int_0^1 f(t)dt$!} => $\int_0^1 f(t)dt$
    \item {\color{teal}\verb!$\iint_0^1 g(x, y)dxdy$!} => $\iint_0^1 g(x, y)dxdy$
    \item {\color{teal}\verb!$\int_0^1 f(t)\mathrm{d}t$!} => $\int_0^1 f(t)\mathrm{d}t$
    \item {\color{teal}\verb!$\vec x = \overrightarrow{AB}$!} => $\vec x = \overrightarrow{AB}$

    \item {\color{teal}\verb!$\overset{*}{X}$!} => $\overset{*}{X}$ {\color{red}|} 
          {\color{teal}\verb!$\underset{*}{X}$!} => $\underset{*}{X}$ {\color{red}|} 
          {\color{teal}\verb!$\underset{m}{X}$!} => $\underset{m}{X}$
    \item {\color{teal}\verb!\[\max_n f(n)\]!} => \[\max_n f(n)\]
\end{enumerate}

\paragraph{分式}~{}

\begin{enumerate}[topsep=0pt,itemsep=0pt,parsep=0pt,leftmargin=3.6em,label=\arabic*>]
    \item {\color{teal}\verb!$\frac 12 + \frac 1a = \frac{2+a}{2a}$!} => $\frac 12 + \frac 1a = \frac{2+a}{2a}$
    \item {\color{teal}\verb!\[\tfrac 12 f(x) = \frac{1}{\dfrac 1a + \dfrac 1b + c}\]!} 
        => \[\tfrac 12 f(x) = \frac{1}{\dfrac 1a + \dfrac 1b + c}\] \par
        {\color{teal}\verb!\dfrac!} : display style, looks smaller \par
        {\color{teal}\verb!\tfrac!} : text style, looks normal
\end{enumerate}

\paragraph{根式}~{}

\begin{enumerate}[topsep=0pt,itemsep=0pt,parsep=0pt,leftmargin=3.6em,label=\arabic*>]
    \item {\color{teal}\verb!$\sqrt 4 = \sqrt[3]{8} = 2$!} => $\sqrt 4 = \sqrt[3]{8} = 2$
\end{enumerate}

\paragraph{inline formula : 行内公式}~{}

行内公式是嵌在文本中的,其输入可以有三种方式:

\begin{enumerate}[topsep=0pt,itemsep=0pt,parsep=0pt,leftmargin=3.6em,label=\arabic*>]
    \item \$ \emph{formula} \$ \par
        {\color{DefinedColorGreen}\$ a + b > c \$} 
        $\Rightarrow$ % =>
        {\color{DefinedColorRed}$a + b > c$}
    \item \textbackslash( \emph{formula} \textbackslash) \par
        {\color{DefinedColorGreen}\textbackslash( a + b > c \textbackslash)} 
        $\Rightarrow$ % =>
        {\color{DefinedColorRed}\(a + b > c\)}
    \item \textbackslash begin\{math\} \emph{formula} \textbackslash end\{math\} \par
        {\color{DefinedColorGreen} \textbackslash begin\{math\} a + b > c \textbackslash end\{math\}} 
        $\Rightarrow$ % =>
        {\color{DefinedColorRed}\begin{math} a + b > c \end{math} }
\end{enumerate}

\paragraph{line formula : 行间公式}~{}

行间公式是单独占据一行居中显示,其输入也有三种:
\begin{enumerate}[topsep=0pt,itemsep=0pt,parsep=0pt,leftmargin=3.6em,label=\arabic*>]
    \item {\color{teal}\verb!\[\max_n f(n)\]!} => \[\max_n f(n)\]
\end{enumerate}

\paragraph{matrix}~{}

matrix 环境共有以下几种:

\begin{enumerate}[topsep=0pt,itemsep=0pt,parsep=0pt,leftmargin=3.6em,label=\arabic*>]
    \item matrix
    \item bmatrix
    \item vmatrix
    \item pmatrix
    \item Bmatrix
    \item Vmatrix
\end{enumerate}

\[ A_{\mathrm{matrix}} = 
\begin{matrix}
    a_{11} & a_{12} & a_{13} \\
    a_{21} & a_{22} & a_{23} \\
    a_{31} & a_{32} & a_{33}
\end{matrix}
\]

\[ A_{\mathrm{bmatrix}} = 
\begin{bmatrix}
    a_{11} & a_{12} & a_{13} \\
    a_{21} & a_{22} & a_{23} \\
    a_{31} & a_{32} & a_{33}
\end{bmatrix}
\]

\[ A_{\mathrm{vmatrix}} = 
\begin{vmatrix}
    a_{11} & a_{12} & a_{13} \\
    a_{21} & a_{22} & a_{23} \\
    a_{31} & a_{32} & a_{33}
\end{vmatrix}
\]

\[ A_{\mathrm{pmatrix}} = 
\begin{pmatrix}
    a_{11} & a_{12} & a_{13} \\
    a_{21} & a_{22} & a_{23} \\
    a_{31} & a_{32} & a_{33}
\end{pmatrix}
\]

\[ A_{\mathrm{Vmatrix}} = 
\begin{Vmatrix}
    a_{11} & a_{12} & a_{13} \\
    a_{21} & a_{22} & a_{23} \\
    a_{31} & a_{32} & a_{33}
\end{Vmatrix}
\]

%%%%%%%%%%%%%%%%%%%%%%%%%%%%%%%%%%%%%%%%%%

\begin{lstlisting}[language={python}]
\[ A_{\mathrm{matrix}} = 
\begin{matrix}
    a_{11} & a_{12} & a_{13} \\
    a_{21} & a_{22} & a_{23} \\
    a_{31} & a_{32} & a_{33}
\end{matrix}
\]
\end{lstlisting}

\begin{lstlisting}[language={python}]
\[ A_{\mathrm{bmatrix}} = 
\begin{bmatrix}
    a_{11} & a_{12} & a_{13} \\
    a_{21} & a_{22} & a_{23} \\
    a_{31} & a_{32} & a_{33}
\end{bmatrix}
\]
\end{lstlisting}

\begin{lstlisting}[language={python}]
\[ A_{\mathrm{vmatrix}} = 
\begin{vmatrix}
    a_{11} & a_{12} & a_{13} \\
    a_{21} & a_{22} & a_{23} \\
    a_{31} & a_{32} & a_{33}
\end{vmatrix}
\]
\end{lstlisting}

\begin{lstlisting}[language={python}]
\[ A_{\mathrm{pmatrix}} = 
\begin{pmatrix}
    a_{11} & a_{12} & a_{13} \\
    a_{21} & a_{22} & a_{23} \\
    a_{31} & a_{32} & a_{33}
\end{pmatrix}
\]
\end{lstlisting}

\begin{lstlisting}[language={python}]
\[ A_{\mathrm{Vmatrix}} = 
\begin{Vmatrix}
    a_{11} & a_{12} & a_{13} \\
    a_{21} & a_{22} & a_{23} \\
    a_{31} & a_{32} & a_{33}
\end{Vmatrix}
\]
\end{lstlisting}



\newpage

\subsubsection{special mark}

\paragraph{mark key point}~{}

\begin{enumerate}[topsep=0pt,itemsep=0pt,parsep=0pt,leftmargin=3.6em,label=\arabic*>]
    \item \verb!\underline{xxx}! \par
        \verb!\underline{abc}! \ColorStr{red}{=>} \underline{abc}
    \item \verb!\emph{xxx}! \par
        \verb!\emph{abc}! \ColorStr{red}{=>} \emph{abc}
    \item \verb!\uuline{xxx}! \par
        \verb!\uuline{abc}! \ColorStr{red}{=>} \uuline{abc}
    \item \verb!\uwave{xxx}! \par
        \verb!\uwave{abc}! \ColorStr{red}{=>} \uwave{abc}
    \item \verb!\sout{xxx}! \par
        \verb!\sout{abc}! \ColorStr{red}{=>} \sout{abc}
    \item \verb!\xout{xxx}! \par
        \verb!\xout{abc}! \ColorStr{red}{=>} \xout{abc}
    \item \verb!\dashuline{xxx}! \par
        \verb!\dashuline{abc}! \ColorStr{red}{=>} \dashuline{abc}
    \item \verb!\dotuline{xxx}! \par
        \verb!\dotuline{abc}! \ColorStr{red}{=>} \dotuline{abc}
\end{enumerate}

\paragraph{todo list}~{}

\begin{enumerate}[topsep=0pt,itemsep=0pt,parsep=0pt,leftmargin=3.6em,label=\arabic*>]
    \item \verb!$\square$! \ColorStr{red}{=>} $\square$
    \item \verb!$\boxtimes$! \ColorStr{red}{=>} $\boxtimes$
    \item \verb!$\surd$! \ColorStr{red}{=>} $\surd$
\end{enumerate}

\begin{enumerate}[topsep=0pt,itemsep=0pt,parsep=0pt,leftmargin=3.6em,label=\arabic*>]
    \item $\square$《笔记本》- 笔记本
    \item $\surd$《\sout{笔记本}》- 笔记本
    \item $\boxtimes$\sout{《笔记本》- 笔记本} $\surd$\ColorStr{red}{(done)}
    \item $\square$《笔记本》- 笔记本
\end{enumerate}

\paragraph{roman numerals}~{}

\begin{lstlisting}[language={python}]
I       `-`> 1                          VI      `-`> 6
II      `-`> 2                          VII     `-`> 7
III     `-`> 3                          VIII    `-`> 8
IV      `-`> 4                          IX      `-`> 9
V       `-`> 5                          X       `-`> 10
\end{lstlisting}





\newpage

\subsubsection{special symbols}

\url{https://www.latexlive.com/help#d13}

\url{http://mohu.org/info/symbols/symbols.htm}

\url{https://www.caam.rice.edu/~heinken/latex/symbols.pdf}

特殊符号都是在 \$ xxx \$ 中输入的,或者在输入公式的其他方式中输入,{\color{DefinedColorRed}eg. \$\textbackslash 
bar\{x\}\$}

\newcommand\PrintCode[2]{\textcolor{DefinedColorGreen}{\textbackslash #1\{#2\}}}
\newcommand\PrintSymbol[1]{\textcolor{red}{$#1$}}
\newcommand\SpecCharPair[3]{\PrintSymbol{#1} & \PrintCode{#2}{#3}}

\paragraph{math related}~{}

\begin{longtable}{|c|m{10em}|c|m{10em}|c|m{10em}|}
    \hline
    \SpecCharPair{\sin\theta}{sin}{\textbackslash theta} &
    \SpecCharPair{\cos{\theta}}{cos}{\textbackslash theta} &
    \SpecCharPair{\cosh{h}}{cosh}{h} \\
    \hline
    \SpecCharPair{\sinh g}{sinh}{g} &
    \SpecCharPair{\min{L}}{min}{L} &
    \SpecCharPair{\max{H}}{max}{H} \\
    \hline
    \SpecCharPair{\sup{t}}{sup}{t} &
    \SpecCharPair{\ker{x}}{ker}{x} &
    \SpecCharPair{\Pr{x}}{Pr}{x} \\
    \hline
    \SpecCharPair{\arctan{\frac{L}{T}}}{arctan}{\textbackslash frac\{L\}\{T\}} &
    \SpecCharPair{\arcsin\frac{L}{r}}{arcsin}{\textbackslash frac\{L\}\{r\}} &
    \SpecCharPair{\arccos{\frac{T}{r}}}{arccos}{\textbackslash frac\{T\}\{r\}} \\
    \hline
    \SpecCharPair{\operatorname{argch}l}{operatorname}{\{argch\}l} &
    \SpecCharPair{\operatorname{argsh}k}{operatorname}{\{argsh\}k} &
    \SpecCharPair{\operatorname{th}i}{operatorname}{\{th\}i} \\
    \hline
    \SpecCharPair{\operatorname{ch}h}{operatorname}{\{ch\}h} &
    \SpecCharPair{\operatorname{argth}m}{operatorname}{\{argth\}m} &
    \SpecCharPair{\arg{x}}{arg}{x} \\
    \hline
    \SpecCharPair{\tan{theta}}{tan}{theta} &
    \SpecCharPair{\tanh{i}}{tanh}{i} &
    \SpecCharPair{\limsup{S}}{limsup}{S} \\
    \hline
    \SpecCharPair{\liminf{I}}{liminf}{I} &
    \SpecCharPair{\lg{X}}{lg}{X} &
    \SpecCharPair{\inf{s}}{inf}{s} \\
    \hline
    \SpecCharPair{\exp{\!t}}{exp}{\textbackslash !t} &
    \SpecCharPair{\ln{X}}{ln}{X} &
    \SpecCharPair{\log{X}}{log}{X} \\
    \hline
    \SpecCharPair{\deg{x}}{deg}{x} &
    \SpecCharPair{\det{x}}{det}{x} &
    \SpecCharPair{\dim{x}}{dim}{x} \\
    \hline
    \SpecCharPair{\log{_\alpha{x}}}{log}{\_\textbackslash alpha\{x\}} &
    \SpecCharPair{\hom{x}}{hom}{x} &
    \SpecCharPair{\lim{_{t\to n}T}}{lim}{\_\{t \textbackslash to n\}T} \\
    \hline
\end{longtable}

\begin{longtable}{|c|m{9em}|c|m{9em}|c|m{9em}|c|m{9em}|}
    \hline
    \SpecCharPair{\nabla}{nabla}{} &
    \SpecCharPair{\partial{x}}{partial}{x} &
    \SpecCharPair{\mathrm{d}{x}}{mathrm\{d\}}{x} \\
    \hline
\end{longtable}

\paragraph{relation symbols}~{}
\paragraph{binary operations}~{}
\paragraph{set and/or logic notation}~{}
\paragraph{delimiters}~{}
\paragraph{greek letters}~{}


\begin{longtable}{|c|m{9em}|c|m{9em}|c|m{9em}|c|m{9em}|}
    \hline
    \SpecCharPair{\alpha}{}{alpha} &
    \SpecCharPair{\beta}{}{beta} &
    \SpecCharPair{\gamma}{}{gamma} &
    \SpecCharPair{\delta}{}{delta} \\
    \hline
    \SpecCharPair{\epsilon}{}{epsilon} &
    \SpecCharPair{\varepsilon}{}{varepsilon} &
    \SpecCharPair{\zeta}{}{zeta} &
    \SpecCharPair{\eta}{}{eta} \\
    \hline
    \SpecCharPair{\theta}{}{theta} &
    \SpecCharPair{\vartheta}{}{vartheta} &
    \SpecCharPair{\iota}{}{iota} &
    \SpecCharPair{\kappa}{}{kappa} \\
    \hline
    \SpecCharPair{\lambda}{}{lambda} &
    \SpecCharPair{\mu}{}{mu} &
    \SpecCharPair{\nu}{}{nu} &
    \SpecCharPair{\xi}{}{xi} \\
    \hline
    \PrintSymbol{o} & \textcolor{DefinedColorGreen}{o} &
    \SpecCharPair{\pi}{}{pi} &
    \SpecCharPair{\varpi}{}{varpi} &
    \SpecCharPair{\rho}{}{rho} \\
    \hline
    \SpecCharPair{\varrho}{}{varrho} &
    \SpecCharPair{\sigma}{}{sigma} &
    \SpecCharPair{\varsigma}{}{varsigma} &
    \SpecCharPair{\tau}{}{tau} \\
    \hline
    \SpecCharPair{\upsilon}{}{upsilon} &
    \SpecCharPair{\phi}{}{phi} &
    \SpecCharPair{\varphi}{}{varphi} &
    \SpecCharPair{\chi}{}{chi} \\
    \hline
    \SpecCharPair{\psi}{}{psi} &
    \SpecCharPair{\omega}{}{omega} &
    \SpecCharPair{\Gamma}{}{Gamma} &
    \SpecCharPair{\Delta}{}{Delta} \\
    \hline
    \SpecCharPair{\Theta}{}{Theta} &
    \SpecCharPair{\Lambda}{}{Lambda} &
    \SpecCharPair{\Xi}{}{Xi} &
    \SpecCharPair{\Pi}{}{Pi} \\
    \hline
    \SpecCharPair{\Sigma}{}{Sigma} &
    \SpecCharPair{\Upsilon}{}{Upsilon} &
    \SpecCharPair{\Phi}{}{Phi} &
    \SpecCharPair{\Psi}{}{Psi} \\
    \hline
    \SpecCharPair{\Omega}{}{Omega} \\
    \cline{1-2}
\end{longtable}

\paragraph{other symbols}~{}

\begin{longtable}{|c|m{9em}|c|m{9em}|c|m{9em}|c|m{9em}|}
    \hline
    \SpecCharPair{\bar{x}}{bar}{x} &
    \SpecCharPair{\acute{\eta}}{acute}{\textbackslash eta} &
    \SpecCharPair{\grave{\eta}}{grave}{\textbackslash eta} &
    \SpecCharPair{\check{\alpha}}{check}{\textbackslash alpha} \\
    \hline
    \SpecCharPair{\breve{a}}{breve}{a} &
    \SpecCharPair{\dot{x}}{dot}{x} &
    \SpecCharPair{\hat{\alpha}}{hat}{\textbackslash alpha} &
    \SpecCharPair{\ddot{y}}{ddot}{y} \\
    \hline
    \SpecCharPair{\tilde{\iota}}{tilde}{\textbackslash iota} \\
    \cline{1-2}
\end{longtable}

\paragraph{trigonometric functions}~{}



\subsubsection{shaded 设置段落阴影}

使用 framed package 的 shaded 环境来进行设置,这个阴影环境支持跨页,颜色设置在导言区 \textcolor{red}{eg. \textbackslash 
colorlet\{shadecolor\}\{gray!20\}};并使用 \textcolor{red}{eg. \textbackslash vspace\{-0.8em\}} 来进行前后间距设置

\begin{lstlisting}[language={sh}]
\vspace{`-`0.8em}
\begin{shaded}
a b c d ... 等具体内容
\end{shaded}
\vspace{`-`0.8em}
\end{lstlisting}

\vspace{-0.8em}
\begin{shaded}
在实际的网络中存在如下图所示的情况,因此实际考虑有 “quantize-op” 与 relu 或者 conv 时,
是与 “quantize-op” 之前的 relu 融合还是与 “quantize-op” 之后的 conv 融合的问题
\end{shaded}
\vspace{-0.8em}

\subsubsection{footnote}

\begin{lstlisting}[language={python}]
AAA \footnote{AAA 的注脚内容}
\end{lstlisting}

\subsubsection{embeded image}

\begin{lstlisting}[language={sh}]
\begin{figure}[htbp!]
\setlength{\belowcaptionskip}{`-`1.3em}
\centering
\includegraphics[scale=0.68]{`{\color{red}images/aaa.jpg}`}
\caption{aaa}
\label{fig:aaa}
\end{figure}
\end{lstlisting}


\subsubsection{section}

章节标题的编码方式

\begin{lstlisting}[language={python}]
\chapter{name}
\section{name}
\subsection{name}
\subsubsection{name}

\paragraph{name}~{}
\subparagraph{name}~{}

\subparagraph*{name}~{} # 添加一个 '*' 不生成标题 
\end{lstlisting}






\subsection{other}

%%\subsubsection{chen shuo config}
%%这个是陈硕的 latex 常用设置,其中的反斜杠使用正斜杠替代,这样方便显示
%%
%%https://blog.csdn.net/solstice/article/details/638
%%
%%\paragraph{段首缩进}~{}
%%
%%中文习惯在段首缩进两格,在LaTeX中,/parindent 表示段首缩进的长度,我们将它设置为当前字号的两个
%%大写字母M的宽度,大约正好是两个汉字的宽度:{\color{DefinedColorRed}/setlength\{/parindent\}\{2em\}}
%%
%%LaTeX 默认每节的第一段的段首不缩进,这不符合中文排版习惯。我们希望正文的每一段都要缩进,使用
%%indentfirst宏包就可办到:{\color{DefinedColorRed}/usepackage\{indentfirst\}}
%%
%%\paragraph{段距行距}~{}
%%
%%LaTeX 用/baselineskip表示当前的行距,其默认值大约是当前字号的1.2倍,如果当前字号是10pt,
%%那么/baselineskip是12pt。这对英文排版是合适的,对中文就显得太拥挤了,因为英文正文多为小
%%写字母,字高与小写的x差不多(即1ex)。如果字号为10pt,那么1ex =4.3pt。我通常把行距设为字
%%号的1.8倍:
%%
%%{\color{DefinedColorRed}/setlength\{/baselineskip\}\{1.8em\}}
%%
%%这个值随时可以改,对更改点以后的文字有效
%%
%%LaTeX 用/parskip表示段距,我一般把它设为1ex:{\color{DefinedColorRed}/setlength\{/parskip\}\{1ex\}}
%%
%%注意这些修改长度的命令最好都放在正文区(即/begin\{document\}之后)
%%
%%\paragraph{页眉页脚}~{}
%%
%%/documentclass[10pt, a4paper]{book}
%%/usepackage{fancyhdr}
%%
%%在 LaTeX 中先把 page style 设为fancy,再设置这个style中的页眉和页脚。但是它默认每章的
%%第一页的page style是plain,需要单独处理。
%%
%%\begin{lstlisting}[language={matlab}]
%%% 设置 plain style 的属性
%%/fancypagestyle{plain}{%
%%/fancyhf{}                    % 清空当前设置
%%
%%% 设置页眉 (head)
%%/fancyhead[RE]{/leftmark}     % 在偶数页的右侧显示章名
%%/fancyhead[LO]{/rightmark}    % 在奇数页的左侧显示小节名
%%/fancyhead[LE,RO]{~/thepage~} % 在偶数页的左侧,奇数页的右侧显示页码
%%
%%% 设置页脚:在每页的右下脚以斜体显示书名
%%/fancyfoot[RO,RE]{/it Typesetting with /LaTeX}
%%
%%/renewcommand{/headrulewidth}{0.7pt} % 页眉与正文之间的水平线粗细
%%/renewcommand{/footrulewidth}{0pt}
%%}
%%
%%/pagestyle{fancy}             % 选用 fancy style
%%% 其余同 plain style
%%/fancyhf{}                   
%%/fancyhead[RE]{/leftmark}
%%/fancyhead[LO]{/rightmark}
%%/fancyhead[LE,RO]{~/thepage~}
%%/fancyfoot[RO,RE]{/it Typesetting with /LaTeX}
%%/renewcommand{/headrulewidth}{0.7pt}
%%/renewcommand{/footrulewidth}{0pt}
%%
%%% 设置章名和节名的显示方式
%%/renewcommand{/chaptermark}[1]{/markboth{~第~/thechapter~章~~~#1~}{}}
%%/renewcommand{/sectionmark}[1]{/markright{~/thesection~~#1~}{}}
%%\end{lstlisting}
%%
%%\paragraph{章节标题}~{}
%%
%%通常用titlesec宏包来设置正文中出现的章节标题的格式:
%%
%%/usepackage{titlesec}
%%
%%设置章名为右对齐,字号为/Huge,字型为黑体,章号用粗体,并设置间距:
%%
%%/titleformat{/chapter}{/flushright/Huge/hei}{{/bf /thechapter}}{0pt}{}
%%
%%/titlespacing{/chapter}{0pt}{-20pt}{25pt}
%%
%%设置节名的字号为/Large,字型为黑体,节号用粗体,并设置间距:
%%
%%/titleformat{/section}{/Large /hei}{{/bf /thesection/space}}{0pt}{}
%%
%%/titlespacing*{/section}{0pt}{1ex plus .3ex minus .2ex}{-.2ex plus .2ex}
%%
%%其中/hei的定义为:
%%
%%/newcommand{/hei}{/CJKfamily{hei}}
%%
%%\paragraph{纸张大小}~{}
%%
%%\begin{lstlisting}[language={matlab}]
%%/documentclass[10pt, b5paper]{report}
%%/usepackage[body={12.6cm, 20cm}, centering, dvipdfm]{geometry}
%%% 以上将版心宽度设为 12.6cm,高度 20cm,版心居中,且自动设置PDF文件的纸张大小
%%\end{lstlisting}
%%
%%\subsubsection{documentclass}
%%
%%latex 有三种文档类型(CTE 环境):ctexart、ctexrep、ctexbook
%%
%%优先选用 ctexrep 文档类型,因为这个支持到 chapter,且\textcolor{red}{目录}后面不会添加一个空白页
%%
%%而 book 类型是要考虑印刷时的奇偶页问题,所以会在\textcolor{red}{目录}后面在不满足 chapter 所在页是奇数页时会添加一个空白页,
%%这个是 book 规范设置
%%


\subsection{experience}

\begin{enumerate}[topsep=0pt,itemsep=0pt,parsep=0pt,leftmargin=3.6em,label=\arabic*>]
    \item enumerate 的 list 的 leftmargin 参考设置
    \begin{enumerate}[topsep=0pt,itemsep=0pt,parsep=0pt,leftmargin=1.8em]
        \item 最外层的\ {\color{DefinedColorRed}leftmargin = 3.6em}
        \item 嵌套第一层的\ {\color{DefinedColorRed}leftmargin = 1.8em},并且可以去掉 label 设置
    \end{enumerate}
    \item 在 lstlisting 环境中,如果需要对某行使用 latex 命令来进行特殊设置,比如高亮、斜体、粗体等
        需要在其逃逸符号\ {\color{DefinedColorRed}(反引号)}\ 中使用
\end{enumerate}





\newpage
\subsection{errors}

\begin{lstlisting}[language={c}]
Runaway argument?
{\376\377\0001\000.\0
! File ended while scanning use of \@@BOOKMARK.
<inserted text>
    \par
l.232 \begin{document}}
\end{lstlisting}

遇到这个错误是上一次编译中断后的 aux/out 文件的部分内容缺失,因此只需要删除 *.aux / *.out 文件之后重新编译即可



 % chapter three
%
\section{latex}

\textsl{\url{http://www.ctex.org/OnlineDocuments}}

\textsl{\url{http://www.ctex.org/documents/packages/layout/fancyhdr.htm}}

\url{http://cremeronline.com/LaTeX/minimaltikz.pdf} : Tikz 制图

\url{http://www.texample.net/tikz/}

\url{https://www.latexlive.com/} : 在线 tex formula 验证

\url{https://blog.csdn.net/lishoubox/article/details/6783726} : 这个基础知识写的不错

\url{https://mathpretty.com/10864.html} : 各种小技巧


\subsection{install}

\subsubsection{separate installation}

xelatex 可以编译出带目录的 pdf 文档 : (这两个安装包都很大,每个都几百M)

\begin{enumerate}[topsep=0pt,itemsep=0pt,parsep=0pt,leftmargin=3.6em,label=\arabic*>]
    \item sudo apt-get install texlive-xetex
    \item sudo apt-get install latex-cjk-all | chinese support
    \item sudo apt-get install texmaker | texmaker is tex editor
\end{enumerate}

\subsubsection{integrated installation}

推荐使用这种安装方式,且安装也比较简单

TeX Live Home Page : \url{http://tug.org/texlive/acquire-netinstall.html}

Texmaker Home Page : \url{http://www.xm1math.net/texmaker/}

\begin{enumerate}[topsep=0pt,itemsep=0pt,parsep=0pt,leftmargin=3.6em,label=\arabic*>]
    \item download texlive from TexLive HomePage : eg. texlive2019-20190410.iso
    \item sudo mount -o loop texlive2019-20190410.iso /media/iso
    \item cd /media/iso
    \item install texlive \par
        {\color{DefinedColorGreen}./install-tl}
    \item set .bashrc \par
        {\color{violet}export PATH=\verb!"!/usr/local/texlive/2019/bin/x86\_64-linux:\$PATH\verb!"!}
    \item set windows fonts, ref section \ref{sec:use-win-font-in-ubuntu}
\end{enumerate}



\subsection{basic}

\begin{enumerate}[topsep=0pt,itemsep=0pt,parsep=0pt,leftmargin=3.6em,label=\arabic*>]
    \item 输入一个反斜杠 : {\color{red}\ \verb|\textbackslash|} => ``\textbackslash"
    \item 输入大括号 : {\color{red}\ \verb|\{|} => ``\{"
    \item 输入百分号 : {\color{red}\ \verb|\%|} => ``\%"
    \item 输入一个空格 : ``\textbackslash\ " 或者 ``{ }"
    \item 输入\CmdFont{--} : \verb!\CmdFont{!\CmdFont{--}\verb!}! 会比较好看
    \item 中文标点(zh) ‘ | ; | ’ | , | 。| “ | ” \par
        \textcolor{teal}{就正常中文输入法中输入的标点符号,就可以直接显示为中文标点}
        {\color{red}(注意是 Calibri 字体的情况下才可以)}

    \item {\color{blue}Times New Roman 字体下的标点(en)}:\par
        \verb={\fontspec{Times New Roman}' | ' | " | " | , | ; | `}= 
        {\color{DefinedColorRed}$\Rightarrow$} % =>
        {\fontspec{Times New Roman}' | ' | " | " | , | ; | `} \par
        \verb={\fontspec{Times New Roman} ' "A" and "B" '}=
        {\color{DefinedColorRed}$\Rightarrow$} % =>
        {\fontspec{Times New Roman}' "A" and "B" '} \par
    \item {\color{blue}Calibri 字体下的标点(en)}{\color{red}(这种输入方式的字体会比较多)}:\par
        \verb={\fontspec{Calibri} ` | ' | `` | " | , | ; | \textasciigrave }=
        {\color{DefinedColorRed}$\Rightarrow$} % =>
        ` | ' | `` | " | , | ; | \textasciigrave \par
        \verb={\fontspec{Calibri} `\,``A" and ``B"\,' }=
        {\color{DefinedColorRed}$\Rightarrow$} % =>
        `\,``A" and ``B"\,'

    \item using symbol:\textbackslash symbol\{96\} {\color{DefinedColorRed}$\to$} \symbol{96} 、
        \textbackslash symbol\{39\} {\color{DefinedColorRed}$\to$} \symbol{39} \par
        通过使用 symbol 来输入字符,可以输入任意有编码的字符
    \item \^{}:input(\textbackslash \^{}\{\})
    \item \~{}:input(\textbackslash \~{}\{\}),在 \textbackslash CmdFont 中也可以使用这个方法来输入 \~{}
    \item \{\textbackslash songti 宋体\}:{\songti 宋体}
    \item \{\textbackslash kaishu 楷书\}:{\kaishu 楷书}
    \item \{\textbackslash heiti 黑体\}:{\heiti 黑体}
    \item \{\textbackslash fangsong 仿宋\}:{\fangsong 仿宋}
    \item \{\textbackslash fontspec\{YouYuan\} 幼圆\}:{\fontspec{YouYuan}幼圆}  % for win-font
    \item \{\textbackslash fontspec\{Consolas\} Consolas\}:{\fontspec{Consolas}Consolas}  % for win-font
        \textcolor{teal}{| 常规五号}
    \item \{\textbackslash zihao\{-5\}\textbackslash fontspec\{Consolas\} Consolas\}:
        {\zihao{-5}\fontspec{Consolas}Consolas}  % for win-font
        \textcolor{teal}{| 小五号}
    \item \{\textbackslash small\textbackslash fontspec\{Consolas\} Consolas\}:
        {\small\fontspec{Consolas}Consolas}  % for win-font
        \textcolor{teal}{| small 就是小五号}
    \item \textbackslash verb:eg. \verb'\verb!\songti!' 
        $\Rightarrow$ % =>
        \verb!\songti! \par
        \textcolor{red}{`!'} 表示环境符号,也可以用其他的符号替代 \par
        \verb={\color{teal}\verb!$\max_n$!}= => {\color{teal}\verb!$\max_n$!} : 给 \textbackslash verb 包围的上色
    \item 数字的下划线分割(使用公式环境):eg. \verb=0xf000$\underline\ $0000= 
        $\Rightarrow$ 
        0xf000$\underline\ $0000 \par
        {\color{teal}在 Calibri 字体下则可以直接使用\_,而不需要使用\textbackslash underline}
    \item url link:\par
        eg. \textbackslash url\{https://xxxx\} 
        $\Rightarrow$ % =>
        \url{https://xxxx} \par
        eg. \verb=\textsl{\url{/https://xxxx}}=
        $\Rightarrow$ % =>
        \textsl{\url{https://xxxx}} \par
        eg. \verb=\emph{\url{/https://xxxx}}=
        $\Rightarrow$ % =>
        \emph{\url{https://xxxx}} | \textcolor{teal}{在 Calibri 字体下会比较好看}
    \item \textbackslash vspace\{-0.8em\}:设置前后行的间距
    \item \textbackslash dotfill:绘制虚线分割线,可以查看 section \ref{sec:book-list} 中的使用效果
    \item \textbackslash hrulefill:绘制实线分割线
\end{enumerate}

\subsubsection{lstlisting / verbatim}

\paragraph{enumerate}~{}

\begin{lstlisting}[language={python}]
\begin{enumerate}

\begin{enumerate}[topsep=0pt,itemsep=0pt,parsep=0pt,leftmargin=3.6em,label=\arabic*>]
\begin{enumerate}[topsep=0pt,itemsep=0pt,parsep=0pt,leftmargin=3.6em,label=\arabic*.]
\begin{enumerate}[topsep=0pt,itemsep=0pt,parsep=0pt,leftmargin=3.6em,label=\arabic*)]
\begin{enumerate}[topsep=0pt,itemsep=0pt,parsep=0pt,leftmargin=3.6em,label=(\arabic*)]

# 大小写罗马数字
\begin{enumerate}[topsep=0pt,itemsep=0pt,parsep=0pt,leftmargin=3.6em,label=\roman*>]
\begin{enumerate}[topsep=0pt,itemsep=0pt,parsep=0pt,leftmargin=3.6em,label=\Roman*>]

# 大小写字母
\begin{enumerate}[topsep=0pt,itemsep=0pt,parsep=0pt,leftmargin=3.6em,label=\alph*>]
\begin{enumerate}[topsep=0pt,itemsep=0pt,parsep=0pt,leftmargin=3.6em,label=\Alph*>]

# 特殊符号
\begin{enumerate}[topsep=0pt,itemsep=0pt,parsep=0pt,leftmargin=3.6em,label=\fnsymbol*>]
\end{lstlisting}

\begin{lstlisting}[language={python}]
\begin{enumerate}[topsep=0pt,itemsep=0pt,parsep=0pt,leftmargin=3.6em,label=\arabic*>]
    \item 编号
        \arabic{enumi}, \roman{enumi}, \Roman{enumi}, 
        \alph{enumi}, \Alph{enumi}, \fnsymbol{enumi}
    \item 编号
        \arabic{enumi}, \roman{enumi}, \Roman{enumi}, 
        \alph{enumi}, \Alph{enumi}, \fnsymbol{enumi}
    \item 编号
        \arabic{enumi}, \roman{enumi}, \Roman{enumi}, 
        \alph{enumi}, \Alph{enumi}, \fnsymbol{enumi}
\end{enumerate}
\end{lstlisting}

\begin{enumerate}[topsep=0pt,itemsep=0pt,parsep=0pt,leftmargin=3.6em,label=\arabic*>]
    \item 编号
        \arabic{enumi}, \roman{enumi}, \Roman{enumi}, \alph{enumi}, \Alph{enumi}, \fnsymbol{enumi}
    \item 编号
        \arabic{enumi}, \roman{enumi}, \Roman{enumi}, \alph{enumi}, \Alph{enumi}, \fnsymbol{enumi}
    \item 编号
        \arabic{enumi}, \roman{enumi}, \Roman{enumi}, \alph{enumi}, \Alph{enumi}, \fnsymbol{enumi}
\end{enumerate}


\paragraph{lstlisting}~{}

\begin{lstlisting}[language={python}]
`\verb!\begin{lstlisting}[language={python}]!`
xxx
`\verb=\end{lstlisting}=`

`\verb!\begin{lstlisting}[language={c++}]!`
`\verb!\begin{lstlisting}[language={sh}]!`
\end{lstlisting}

可以在 lstlisting 环境中使用 \textbackslash verb 来输入各种特殊的字符,比如 {\color{red}\ \verb=`$CMD_A`=}

\begin{lstlisting}[language={python}]
`\verb!\verb=`$CMD_A`=!` `{\color{red}->}` `\verb=`$CMD_A`=`
\end{lstlisting}

lstlisting 在序列中的项时这么使用:

\begin{lstlisting}[language={python}]
    ...
    `\verb!\item!` a item
`\verb!\begin!`{lstlisting}[language={sh}, `{\color{red}xleftmargin=0em}`]
...
`\verb!\end!`{lstlisting}
    `\verb!\item!` other item
    ...
\end{lstlisting}



\paragraph{verbatim}~{}

缩略写法:\textbackslash verb=\textcolor{red}{xxx}= : \textcolor{red}{xxx} 是要原格式表示的内容

%\begin{verbatim}
%    \usepackage[T1]{fontenc}            % font styling
%    \usepackage{lmodern, mathrsfs}
%\end{verbatim}

\begin{lstlisting}[language={python}]
\makeatletter
\renewcommand*{\verbatim@font}{}
\makeatother
\end{lstlisting}

如果需要设置 \textbackslash verb 环境中的颜色,可以直接在外部包上 \textbackslash color\{\} 来进行设置即可

如果还需要在  \textbackslash verb 环境中设置字体,可以以如下方式来使用:

\begin{lstlisting}[language={c}]
{\small\ttfamily\fontspec{Consolas}\verb!grep `-`in "\`-`\`-`" ./ `-`R!}
\end{lstlisting}

%\begin{lstlisting}[language={python}]
%\verb={\fontspec{Times New Roman} abcdefgxxxx}= 
%\end{lstlisting}


\subsubsection{tabular、tabularx、array}

table to latex code : \url{https://www.tablesgenerator.com/}

\url{https://github.com/krlmlr/Excel2LaTeX}

文本或数学模式都可以使用 tabular,数学模式还可以使用 array 环境(即包含数学符号的公式)

table 浮动表格中的几个命令设置如下:

\begin{lstlisting}[language={tex}]
...
\setlength{\abovecaptionskip}{1.0em} %
\setlength{\belowcaptionskip}{`-`1.3em}% 这两个必须得放在 caption 前面,否则不起作用
\caption{mxnet 的基本概念}
...
\end{lstlisting}

\paragraph{tabular}~{}

tabular 中的参数设置: 
\begin{enumerate}[topsep=0pt,itemsep=0pt,parsep=0pt,leftmargin=3.6em,label=\arabic*>]
    \item \textbackslash raggedleft:表示左边不对齐,\textcolor{red}{即右对齐}
    \item \textbackslash raggedright:表示右边不对齐,\textcolor{red}{即左对齐}
    \item \textbackslash centering:表示居中
    \item p\{width value\}:设置列框,使用了列宽后,需在列内容处使用上述的 对其方式
    \item \textbackslash arraystretch\{val\}:设置行间距,默认值为 1
        eg. \textbackslash renewcommand\textbackslash arraystretch\{2\} | 默认值的 2 倍
    \item \textbackslash multicolumn:合并不同列 \par
        \textbackslash multicolumn\{列数\}\{格式\}\{内容\} \par
        \textbackslash multicolumn\{2\}\{c|\}\{内容\} : {\color{red}c 表示居中,| 表示边框} \par
        \textbackslash multicolumn\{2\}\{l|\}\{内容\} : left \par
        \textbackslash multicolumn\{2\}\{r|\}\{内容\} : right \par
        \textcolor{DefinedColorGreen}{如果只是针对 1 列操作时,相当于是修改对齐、竖线格式}
    \item \textbackslash multirow:跨不同行,类似于合并不同行 | \textbackslash usepackage\{multirow\}\par
        \textbackslash multirow\{行数\}\{宽度\}\{内容\} \par
        \textbackslash multirow\{行数\}*\{内容\} | 宽度由输入内容来决定 \par
        需要结合 \textbackslash cline{start - end} 来使用 eg. \verb!\cline{2-5}!
\end{enumerate}

\begin{lstlisting}[language={tex}]
\begin{table}[htbp!]
    \centering
    \setlength{\abovecaptionskip}{0.5em}    % 这个设置在 tabular 前面的话的参数值
    \setlength{\belowcaptionskip}{`-`0.5em}
    \caption{xxx}
    \label{tab:xxx}
\begin{tabular}{|l|l|}
...
\end{lstlisting}

\begin{lstlisting}[language={tex}]
...
\end{tabular}
    \centering
    \setlength{\abovecaptionskip}{1.0em}    % 设置在 tabular 后面的话的参数值
    \setlength{\belowcaptionskip}{`-`1.3em}
    \caption{xxx}
    \label{tab:xxx}
\end{table}
\end{lstlisting}


\begin{center}
%\renewcommand\arraystretch{2}
\begin{tabular}{|p{8em}|c|l|p{4em}|p{4em}|}
    \hline
    \raggedleft{\textbackslash raggedleft} & abc & abc & abc & abc \\
    \hline
    \raggedright{\textbackslash raggedright} & abc & abc & \multicolumn{2}{c|}{abc} \\
    \hline
    \centering{\textbackslash centering} & abc & abc & abc & abc \\
    \hline
    default & abc & abc & abc & abc \\
    \hline
    \multirow{2}*{multirow} & abc & abc & abc & abc \\

    %
    % 这里表示只需要对 2-5 列画横线 , \hline 表示从第 1 列到最后 1 列都画横线
    %
    \cline{2-5}
                            & abc & abc & abc & abc \\
    \hline
    default & abc & abc & abc & abc \\
    \hline
    \centering{\multirow{2}*{multirow}} & abc & abc & abc & abc \\
    \cline{2-5}
                            & abc & abc & abc & abc \\
    \hline
    \raggedleft{\multirow{2}*{multirow}} & abc & abc & abc & abc \\
    \cline{2-5}
                            & abc & abc & abc & abc \\
    \hline
\end{tabular}
\end{center}

\paragraph{tabularx}~{}

tabularx 环境是固定宽度的表格,这里的固定宽度指\textcolor{red}{铺满页面宽度}

%
% Y 是重新定义的格式,\arraybackslash 表示命令恢复
%
\newcolumntype{C}{>{\centering\arraybackslash}X}
\newcolumntype{R}{>{\raggedleft\arraybackslash}X}
\begin{tabularx}{46em}{|c|X|X|R|C|}
    \hline
    数字 & 1 & 2 & 3 & 4 \\
    \hline
    英文 & one & two & three & four \\
    \hline
\end{tabularx}

\paragraph{longtable}~{}

\begin{lstlisting}[language={tex}]
\begin{longtable}{|l|m{40em}|}
    \hline
    aaa   & xxx \\
    \hline
    bbb   & xxx \\
    \hline
    ccc   & xxx # 这样可以实现换行

          1> xxx

          2> xxx \\
    \hline
\end{longtable}
\end{lstlisting}

\begin{longtable}{|l|m{40em}|}
    \hline
    aaa   & xxx \\
    \hline
    bbb   & xxx \\
    \hline
    ccc   & xxx

          1> xxx

          2> xxx \\
    \hline
\end{longtable}

\paragraph{array | tabu}~{}

\begin{enumerate}[topsep=0pt,itemsep=0pt,parsep=0pt,leftmargin=3.6em,label=\arabic*>]
    \item 列格式控制:\textbackslash usepackage\{array\}
    \item 行格式控制:\textbackslash usepackage\{tabu\}
\end{enumerate}

\subparagraph{array}~{}

array 新增的基本命令(\verb=\usepackage{array}=):
\begin{enumerate}[topsep=0pt,itemsep=0pt,parsep=0pt,leftmargin=3.6em,label=\arabic*>]
    \item \verb=m{宽}=:类似 p 格式,产生具有固定宽度的列,并可以自动换行,垂直方向居中对齐 \par
        这个 m 格式在 tabular 、longtable 中都是可以使用的
    \item \verb=b{宽}=:类似 p 格式,垂直方向与最后一行对齐
    \item \verb=>{xx}=:把 xx 插入后面一列的开头
    \item \verb=<{xx}=:把 xx 插入前面一列的末尾
    \item \verb=!{xx}=:把 xx 作为表格线处理,相当于使用了 \verb=@{xx}= 但左右两边会有额外的间距 \par
        eg. \verb=c!{$\Rightarrow$}=
\end{enumerate}

格式符 > 与 < 通常用来设置整列的格式,例如改变一列表格的字体或者使用数学模式

\begin{tabular}{>{\bfseries}c|>{\itshape}c>{$}c<{$}}
    \hline
    姓名 & \textnormal{得分} & \multicolumn{1}{c}{额外加分} \\
    \hline
    A & 85 & +7 \\
    B & 82 & 0 \\
    C & 70 & -2 \\
    \hline
\end{tabular}

%
% 这个可以自动居中
%
\begin{tabular}{|>{$}r<{$}|>{\setlength\parindent{2em}}m{35em}|>{\centering\arraybackslash}m{4em}|}
    \hline
    \pi & 希腊字母,希腊字母希腊字母希腊字母希腊字母希腊字母希腊字母希腊字母希腊字母 & 常用 \\
    \hline
    \pi & 希腊字母,希腊字母希腊字母希腊字母希腊字母希腊字母希腊字母希腊字母希腊字母 & 常用 \\
    \hline
\end{tabular}

array 宏包提供的\verb=\newcolumntype= 用于定义新的列格式的命令:
\begin{enumerate}[topsep=0pt,itemsep=0pt,parsep=0pt,leftmargin=3.6em,label=\arabic*>]
    \item \verb=\newcolumntype{M}{>{$}c<{$}}=
    \item \verb=\newcolumntype{P}[1]{>{\setlength\parindent{2em}}p{#1}}=
    \item \verb=\newcolumntype{C}[1]{>{\centering\arraybackslash}m{#1}}= \par
        eg. \verb=\begin{tabular}{|M|P{15em}|C{4em}|}=
\end{enumerate}

\subparagraph{tabu}~{}

tabu 宏包提供 \textbackslash rowfont 命令来设置行的格式

\begin{tabu}{ccc}
    \hline
    \rowfont{\bfseries} 姓名 & 得分 & \multicolumn{1}{c}{额外加分} \\
    \hline
    \rowfont{\color{blue}} A & 85 & +7 \\
    \rowfont{\color{red}} B & 82 & 0 \\
    \rowfont{\color{green}} C & 70 & -2 \\
    \hline
\end{tabu}

\paragraph{threeparttable}~{}

\begin{enumerate}[topsep=0pt,itemsep=0pt,parsep=0pt,leftmargin=3.6em,label=\arabic*>]
    \item \verb=usepackage{threeparttable}=
    \item \verb=\tnote{val}=:使用这个来定义表格中的角标 eg.\par
        \textcolor{red}{\textbackslash tnote\{2\}} \par
        \textcolor{red}{\textbackslash tnote\{a\}} \par
        \textcolor{red}{\textbackslash tnote\{I\}} \par
        \textcolor{red}{\textbackslash tnote\{*\}}
\end{enumerate}

\begin{center}
\begin{threeparttable} 
    \centering 
    \caption{threeparttable example} 
    \label{tbl:threeparttable example} 
    \begin{tabular}{l|cc|cc}
        \hline 
        $F_{1}$ &$3.70e-03$ &$4.20e-03$\tnote{2} &$ \textbf{3.70e-03}$ &$3.90e-03$\\
        $F_{2}$ &$3.60e-03$\tnote{a} &$1.55e-02$ &$\textbf{6.70e-03}$ &$8.50e-03$\\ 
        \hline
    \end{tabular} 
    \begin{tablenotes} 
    \item[1] The bolder ones mean better. 
    \item[2] my note
    \item[a] my note a
    \end{tablenotes} 
\end{threeparttable} 
\end{center}



\subsubsection{set color}

\begin{enumerate}[topsep=0pt,itemsep=0pt,parsep=0pt,leftmargin=3.6em,label=\arabic*>]
    \item \{\textbackslash color\{blue\}\{蓝色字\}\}:{\color{blue}{蓝色字}}
    \item \textbackslash textcolor\{red\}\{红色字体\}:\textcolor{red}{红色字体}
    \item \textbackslash colorbox\{yellow\}\{黄色盒子\}:\colorbox{yellow}{黄色盒子}
    \vspace{-0.4em}
    \item \textbackslash fcolorbox\{black\}\{green\}\{黑框绿色背景盒子\}:\fcolorbox{black}{green}{黑框绿色背景盒子}
\end{enumerate}

\begin{longtable}{cp{5em}|cp{5em}|cp{5em}|cp{5em}|cp{5em}|cp{5em}}
    \hline
    \colorbox{black}{\textcolor{black}{color}} & \multicolumn{1}{c|}{black} &
    \colorbox{gray}{\textcolor{gray}{color}} & \multicolumn{1}{c|}{gray} &
    \colorbox{olive}{\textcolor{olive}{color}} & \multicolumn{1}{c|}{olive} &
    \colorbox{blue}{\textcolor{blue}{color}} & \multicolumn{1}{c|}{blue} &
    \colorbox{green}{\textcolor{green}{color}} & \multicolumn{1}{c|}{green} &
    \colorbox{teal}{\textcolor{teal}{color}} & \multicolumn{1}{c}{teal} \\
    \hline
    \colorbox{violet}{\textcolor{violet}{color}} & \multicolumn{1}{c|}{violet} &
    \colorbox{brown}{\textcolor{brown}{color}} & \multicolumn{1}{c|}{brown} &
    \colorbox{lightgray}{\textcolor{lightgray}{color}} & \multicolumn{1}{c|}{lightgray} &
    \colorbox{pink}{\textcolor{pink}{color}} & \multicolumn{1}{c|}{pink} &
    \colorbox{white}{\textcolor{white}{color}} & \multicolumn{1}{c|}{white} &
    \colorbox{lime}{\textcolor{lime}{color}} & \multicolumn{1}{c}{lime} \\
    \hline
    \colorbox{purple}{\textcolor{purple}{color}} & \multicolumn{1}{c|}{purple} &
    \colorbox{yellow}{\textcolor{yellow}{color}} & \multicolumn{1}{c|}{yellow} &
    \colorbox{darkgray}{\textcolor{darkgray}{color}} & \multicolumn{1}{c|}{darkgray} &
    \colorbox{magenta}{\textcolor{magenta}{color}} & \multicolumn{1}{c|}{magenta} &
    \colorbox{red}{\textcolor{red}{color}} & \multicolumn{1}{c|}{red} &
    \colorbox{orange}{\textcolor{orange}{color}} & \multicolumn{1}{c}{orange} \\
    \hline
    \colorbox{cyan}{\textcolor{cyan}{color}} & \multicolumn{1}{c|}{cyan} \\
    \hline
\end{longtable}

\begin{longtable}{cp{6em}|cp{6em}|cp{6em}|cp{6em}|cp{6em}}
    \hline
    \colorbox{Yellow}{\textcolor{Yellow}{color}} & \multicolumn{1}{c|}{Yellow} &
    \colorbox{GreenYellow}{\textcolor{GreenYellow}{color}} & \multicolumn{1}{c|}{GreenYellow} &
    \colorbox{RubineRed}{\textcolor{RubineRed}{color}} & \multicolumn{1}{c|}{RubineRed} &
    \colorbox{RoyalPurple}{\textcolor{RoyalPurple}{color}} & \multicolumn{1}{c|}{RoyalPurple} &
    \colorbox{Emerald}{\textcolor{Emerald}{color}} & \multicolumn{1}{c}{Emerald} \\
    \hline
    \colorbox{WildStrawberry}{\textcolor{WildStrawberry}{color}} & \multicolumn{1}{c|}{WildStrawberry} &
    \colorbox{BlueViolet}{\textcolor{BlueViolet}{color}} & \multicolumn{1}{c|}{BlueViolet} &
    \colorbox{Emerald}{\textcolor{Emerald}{color}} & \multicolumn{1}{c}{Emerald} \\
    \hline
\end{longtable}

\begin{enumerate}[topsep=0pt,itemsep=0pt,parsep=0pt,leftmargin=3.6em,label=\arabic*>]
    \item \textbackslash columncolor:eg. \verb!\begin{tabular}{>{\columncolor{gray}}c >{\columncolor{red}}c}!
    \item \textbackslash rowcolor:eg. \verb!\rowcolor{red} A & B & C \\!
    \item \verb=\setlength{\arrayrulewidth}{0.19mm}=:设置表格横线与竖线的粗细,实验下来设置 0.19 mm 最好,
        设置颜色时不会把线给遮挡了,而且线条显示也均匀 \par
        可以把这个参数设置到\textcolor{red}{导言区}
    \item \verb=\arrayrulecolor{green}= 用来设置表格线条为\textcolor{green}{绿色} \par
        后面的表格如果想要恢复成黑色,需要再重新设置一下
\end{enumerate}


\arrayrulecolor{green}
\begin{tabular}{|l|c|r|}
    \arrayrulecolor{black}\hline
    United Kingdom & London & Thames\\
    \arrayrulecolor{blue}\hline
    France & Paris & Seine \\
    \arrayrulecolor{black}\cline{1-1}
    \arrayrulecolor{red}\cline{2-3}
    Russia & Moscow & Moskva \\ \hline
\end{tabular}


\arrayrulecolor{black}
\begin{tabular}{|>{\columncolor{lime}[6pt][6pt]}c|l|l|r|c|}
    \hline
    数字 & 1 & 2 & 3 & \cellcolor{red}4 \\
    \hline
    英文 & one & two & three & four \\
    \hline
    \rowcolor{lightgray}[\tabcolsep][\tabcolsep] 汉文 & 一 & 二 & 三 & 四 \\
    \hline
\end{tabular}




\subsubsection{define command and environment}

\paragraph{define command}~{}

使用 \verb=\newcommand<命令>[<参数个数>][<首参数默认值>]{<具体定义>}= 来定义一个新的命令
\begin{enumerate}[topsep=0pt,itemsep=0pt,parsep=0pt,leftmargin=3.6em,label=\arabic*>]
    \item 无参数定义:\par
        \verb=\newcommand<命令>{<具体定义>}= 
    \item 有参数定义:\par
        参数编号使用\verb=#1、#2、#3、...、#9=
\end{enumerate}

以下是一些定义实例:

\newcommand\PRCa{\emph{China}}
\newcommand\PRCb{\textcolor{red}{\emph{China}}}
\newcommand\PRCc[1]{\textcolor{blue}{\textcolor{red}{#1} country is \emph{China}}}
\newcommand\PRCd[1][my]{\textcolor{blue}{\textcolor{red}{#1} country is \emph{China}}}

\begin{enumerate}[topsep=0pt,itemsep=0pt,parsep=0pt,leftmargin=3.6em,label=\arabic*>]
    \item \verb=\newcommand\PRCa{\emph{China}}= \par
        \verb=\PRCa= $\Rightarrow$ \PRCa
    \item \verb=\newcommand\PRCb{\textcolor{red}{\emph{China}}}= \par
        \verb=\PRCb= $\Rightarrow$ \PRCb
    \item \verb=\newcommand\PRCc[1]{\textcolor{blue}{\textcolor{red}{#1} country is \emph{China}}}= \par
        \verb=\PRCc{my}= $\Rightarrow$ \PRCc{my}
    \item \verb=\newcommand\PRCd[1][my]{\textcolor{blue}{\textcolor{red}{#1} country is \emph{China}}}= \par
        \verb=\PRCd= $\Rightarrow$ \PRCd \par
        \verb=\PRCd[your]= $\Rightarrow$ \PRCd[your] \par
\end{enumerate}

\paragraph{renew command}~{}

使用 \verb=\renewcommand<命令>[<参数个数>][<首参数默认值>]{<具体定义>}= 来重定义命令

\paragraph{new/renew environment}~{}



\subsubsection{font set}

在 linux 中 latex 字体如果想使用 win 的字体,需要先把 win 的字体拷贝到 linux 中,具体操作可以参考 
\ref{sec:use-win-font-in-ubuntu}

在导言区设置如下,就可以使用 win 的字体:
\begin{lstlisting}[language={matlab}]
\usepackage{fontspec}
%\setmainfont{Times New Roman}   % 这些字体可以用 fc`-`list 中查看名字
\setmainfont{Calibri}
\setsansfont{Verdana}
%\setmonofont{Courier New}
%\setmonofont{Consolas}
\end{lstlisting}

\begin{lstlisting}[language={sh}]
minphone@mcp:~$fc`-`list
/usr/share/fonts/windows_fonts/BROADW.TTF: `\textcolor{red}{Broadway}`:style=Regular,obyčejné
/usr/share/fonts/windows_fonts/consola.ttf: `\textcolor{red}{Consolas}`:style=Regular
/usr/share/fonts/windows_fonts/constanb.ttf: `\textcolor{red}{Constantia}`:style=Bold
/usr/share/fonts/windows_fonts/ONYX.TTF: `\textcolor{red}{Onyx}`:style=Regular,obyčejné
\end{lstlisting}

{\color{red}按以下方式来使用字体:}
\begin{enumerate}[topsep=0pt,itemsep=0pt,parsep=0pt,leftmargin=3.6em,label=\arabic*>]
    \item \{\textbackslash fontspec\{Microsoft YaHei Light\} 微软雅黑 Light\}:
        {\fontspec{Microsoft YaHei Light}微软雅黑 Light}  % for win-font
    \item \{\textbackslash fontspec\{YouYuan\} 幼圆\}:{\fontspec{YouYuan}幼圆}  % for win-font
    \item \{\textbackslash fontspec\{Comic Sans MS\} Comic Sans MS\}:{\fontspec{Comic Sans MS}Comic Sans MS}  % for win-font
    \item \{\textbackslash small\textbackslash fontspec\{Comic Sans MS\} Comic Sans MS\}
        {\small\fontspec{Comic Sans MS}Comic Sans MS}  % for win-font

    \item \{\textbackslash fontsize\{8pt\}\{0\}\textbackslash fontspec\{Comic Sans MS\} Comic Sans MS\}
        {\fontsize{8pt}{0}\fontspec{Comic Sans MS}Comic Sans MS}  % for win-font

    \item \{\textbackslash small\textbackslash fontspec\{Consolas\} Consolas\}:
        {\small\fontspec{Consolas}Consolas}  % for win-font
        | small {\color{teal}就是小五号}
    \item \verb=\textup{upright shape}= : \textup{upright shape}
    \item \verb=\textit{italic shape}= : \textit{italic shape}
    \item \verb=\textsl{slanted shape}= : \textsl{slanted shape}
    \item \verb=\textsc{small capitals shape}= : \textsc{small capticals shape}
    \item \verb=\textbf{apple}= : \textbf{apple}
\end{enumerate}

\paragraph{set font size}~{}

\begin{enumerate}[topsep=0pt,itemsep=0pt,parsep=0pt,leftmargin=3.6em,label=\arabic*>]
    \item \verb=\tiny{Text}= : {\tiny{Text}}
    \item \verb=\small{Text}= : {\small{Text}}
    \item \verb=\normalsize{Text}= : {\normalsize{Text}}
    \item \verb=\large{Text}= : {\large{Text}}
    \item \verb=\huge{Text}= : {\huge{Text}}
    \item \verb={\fontsize{8pt}{0}Text}= : {\fontsize{8pt}{0}Text}
\end{enumerate}

各个字号的榜值 : \url{https://blog.csdn.net/weixin_39679367/article/details/115794548}

\paragraph{font style of math}~{}

\begin{enumerate}[topsep=0pt,itemsep=0pt,parsep=0pt,leftmargin=3.6em,label=\arabic*>]
    \item \verb=$\mathnormal{ABCDEF, abcdef, 123456}$= : {$\mathnormal{ABCDEF, abcdef, 123456}$}
    \item \verb=$\mathnormal{123456}$= : {$\mathnormal{123456}$}
    \item \verb=$\mathrm{ABCDEF, abcdef, 123456}$= : {$\mathrm{ABCDEF, abcdef, 123456}$}
    \item \verb=$\mathit{ABCDEF, abcdef, 123456}$= : {$\mathit{ABCDEF, abcdef, 123456}$}
    \item \verb=$\mathbf{ABCDEF, abcdef, 123456}$= : {$\mathbf{ABCDEF, abcdef, 123456}$}
    \item \verb=$\mathsf{ABCDEF, abcdef, 123456}$= : {$\mathsf{ABCDEF, abcdef, 123456}$}
    \item \verb=$\mathtt{ABCDEF, abcdef, 123456}$= : {$\mathtt{ABCDEF, abcdef, 123456}$}
    \item \verb=$\mathcal{ABCDEF}$= : {$\mathcal{ABCDEF}$}
\end{enumerate}



\newpage

\subsubsection{formula | \$ xxx \$}

\begin{enumerate}[topsep=0pt,itemsep=0pt,parsep=0pt,leftmargin=3.6em,label=\arabic*>]
    \item {\color{teal}\verb!$a \dots b$!} => $a \dots b$
    \item {\color{teal}\verb!$a \vdots b$!} => $a \vdots b$
    \item {\color{teal}\verb!$a \ddots b$!} => $a \ddots b$
    \item {\color{teal}\verb!$\{aaa\}$!} => $\{aaa\}$
    \item {\color{teal}\verb!$[aaa]$!} => $[aaa]$
    %\item {\color{teal}\verb!$a \iddots b$!} => $a \iddots b$
\end{enumerate}

\paragraph{上标/下标}~{}

\begin{enumerate}[topsep=0pt,itemsep=0pt,parsep=0pt,leftmargin=3.6em,label=\arabic*>]
    \item {\color{teal}\verb!$x_{1}$!} => $x_{1}$
    \item {\color{teal}\verb!$x^{2}$!} => $x^{2}$
    \item {\color{teal}\verb!$A_{ij} = 2^{i+j}$!} => $A_{ij} = 2^{i+j}$
    \item {\color{teal}\verb!$A\prime$!} => $A\prime$
    \item {\color{teal}\verb!$A'$!} => $A'$
    \item {\color{teal}\verb!$A''$!} => $A''$
    \item {\color{teal}\verb!$A = 90^\circ$!} => $A = 90^\circ$
    \item {\color{teal}\verb!$\max_n f(n)$!} => $\max_n f(n)$
    \item {\color{teal}\verb!$A_m{}^n$!} => $A_m{}^n$
    \item {\color{teal}\verb!$\int_0^1 f(t)$!} => $\int_0^1 f(t)$
    \item {\color{teal}\verb!$\iint_0^1 f(t)$!} => $\iint_0^1 f(t)$
    \item {\color{teal}\verb!$\int_0^1 f(t)dt$!} => $\int_0^1 f(t)dt$
    \item {\color{teal}\verb!$\iint_0^1 g(x, y)dxdy$!} => $\iint_0^1 g(x, y)dxdy$
    \item {\color{teal}\verb!$\int_0^1 f(t)\mathrm{d}t$!} => $\int_0^1 f(t)\mathrm{d}t$
    \item {\color{teal}\verb!$\vec x = \overrightarrow{AB}$!} => $\vec x = \overrightarrow{AB}$

    \item {\color{teal}\verb!$\overset{*}{X}$!} => $\overset{*}{X}$ {\color{red}|} 
          {\color{teal}\verb!$\underset{*}{X}$!} => $\underset{*}{X}$ {\color{red}|} 
          {\color{teal}\verb!$\underset{m}{X}$!} => $\underset{m}{X}$
    \item {\color{teal}\verb!\[\max_n f(n)\]!} => \[\max_n f(n)\]
\end{enumerate}

\paragraph{分式}~{}

\begin{enumerate}[topsep=0pt,itemsep=0pt,parsep=0pt,leftmargin=3.6em,label=\arabic*>]
    \item {\color{teal}\verb!$\frac 12 + \frac 1a = \frac{2+a}{2a}$!} => $\frac 12 + \frac 1a = \frac{2+a}{2a}$
    \item {\color{teal}\verb!\[\tfrac 12 f(x) = \frac{1}{\dfrac 1a + \dfrac 1b + c}\]!} 
        => \[\tfrac 12 f(x) = \frac{1}{\dfrac 1a + \dfrac 1b + c}\] \par
        {\color{teal}\verb!\dfrac!} : display style, looks smaller \par
        {\color{teal}\verb!\tfrac!} : text style, looks normal
\end{enumerate}

\paragraph{根式}~{}

\begin{enumerate}[topsep=0pt,itemsep=0pt,parsep=0pt,leftmargin=3.6em,label=\arabic*>]
    \item {\color{teal}\verb!$\sqrt 4 = \sqrt[3]{8} = 2$!} => $\sqrt 4 = \sqrt[3]{8} = 2$
\end{enumerate}

\paragraph{inline formula : 行内公式}~{}

行内公式是嵌在文本中的,其输入可以有三种方式:

\begin{enumerate}[topsep=0pt,itemsep=0pt,parsep=0pt,leftmargin=3.6em,label=\arabic*>]
    \item \$ \emph{formula} \$ \par
        {\color{DefinedColorGreen}\$ a + b > c \$} 
        $\Rightarrow$ % =>
        {\color{DefinedColorRed}$a + b > c$}
    \item \textbackslash( \emph{formula} \textbackslash) \par
        {\color{DefinedColorGreen}\textbackslash( a + b > c \textbackslash)} 
        $\Rightarrow$ % =>
        {\color{DefinedColorRed}\(a + b > c\)}
    \item \textbackslash begin\{math\} \emph{formula} \textbackslash end\{math\} \par
        {\color{DefinedColorGreen} \textbackslash begin\{math\} a + b > c \textbackslash end\{math\}} 
        $\Rightarrow$ % =>
        {\color{DefinedColorRed}\begin{math} a + b > c \end{math} }
\end{enumerate}

\paragraph{line formula : 行间公式}~{}

行间公式是单独占据一行居中显示,其输入也有三种:
\begin{enumerate}[topsep=0pt,itemsep=0pt,parsep=0pt,leftmargin=3.6em,label=\arabic*>]
    \item {\color{teal}\verb!\[\max_n f(n)\]!} => \[\max_n f(n)\]
\end{enumerate}

\paragraph{matrix}~{}

matrix 环境共有以下几种:

\begin{enumerate}[topsep=0pt,itemsep=0pt,parsep=0pt,leftmargin=3.6em,label=\arabic*>]
    \item matrix
    \item bmatrix
    \item vmatrix
    \item pmatrix
    \item Bmatrix
    \item Vmatrix
\end{enumerate}

\[ A_{\mathrm{matrix}} = 
\begin{matrix}
    a_{11} & a_{12} & a_{13} \\
    a_{21} & a_{22} & a_{23} \\
    a_{31} & a_{32} & a_{33}
\end{matrix}
\]

\[ A_{\mathrm{bmatrix}} = 
\begin{bmatrix}
    a_{11} & a_{12} & a_{13} \\
    a_{21} & a_{22} & a_{23} \\
    a_{31} & a_{32} & a_{33}
\end{bmatrix}
\]

\[ A_{\mathrm{vmatrix}} = 
\begin{vmatrix}
    a_{11} & a_{12} & a_{13} \\
    a_{21} & a_{22} & a_{23} \\
    a_{31} & a_{32} & a_{33}
\end{vmatrix}
\]

\[ A_{\mathrm{pmatrix}} = 
\begin{pmatrix}
    a_{11} & a_{12} & a_{13} \\
    a_{21} & a_{22} & a_{23} \\
    a_{31} & a_{32} & a_{33}
\end{pmatrix}
\]

\[ A_{\mathrm{Vmatrix}} = 
\begin{Vmatrix}
    a_{11} & a_{12} & a_{13} \\
    a_{21} & a_{22} & a_{23} \\
    a_{31} & a_{32} & a_{33}
\end{Vmatrix}
\]

%%%%%%%%%%%%%%%%%%%%%%%%%%%%%%%%%%%%%%%%%%

\begin{lstlisting}[language={python}]
\[ A_{\mathrm{matrix}} = 
\begin{matrix}
    a_{11} & a_{12} & a_{13} \\
    a_{21} & a_{22} & a_{23} \\
    a_{31} & a_{32} & a_{33}
\end{matrix}
\]
\end{lstlisting}

\begin{lstlisting}[language={python}]
\[ A_{\mathrm{bmatrix}} = 
\begin{bmatrix}
    a_{11} & a_{12} & a_{13} \\
    a_{21} & a_{22} & a_{23} \\
    a_{31} & a_{32} & a_{33}
\end{bmatrix}
\]
\end{lstlisting}

\begin{lstlisting}[language={python}]
\[ A_{\mathrm{vmatrix}} = 
\begin{vmatrix}
    a_{11} & a_{12} & a_{13} \\
    a_{21} & a_{22} & a_{23} \\
    a_{31} & a_{32} & a_{33}
\end{vmatrix}
\]
\end{lstlisting}

\begin{lstlisting}[language={python}]
\[ A_{\mathrm{pmatrix}} = 
\begin{pmatrix}
    a_{11} & a_{12} & a_{13} \\
    a_{21} & a_{22} & a_{23} \\
    a_{31} & a_{32} & a_{33}
\end{pmatrix}
\]
\end{lstlisting}

\begin{lstlisting}[language={python}]
\[ A_{\mathrm{Vmatrix}} = 
\begin{Vmatrix}
    a_{11} & a_{12} & a_{13} \\
    a_{21} & a_{22} & a_{23} \\
    a_{31} & a_{32} & a_{33}
\end{Vmatrix}
\]
\end{lstlisting}



\newpage

\subsubsection{special mark}

\paragraph{mark key point}~{}

\begin{enumerate}[topsep=0pt,itemsep=0pt,parsep=0pt,leftmargin=3.6em,label=\arabic*>]
    \item \verb!\underline{xxx}! \par
        \verb!\underline{abc}! \ColorStr{red}{=>} \underline{abc}
    \item \verb!\emph{xxx}! \par
        \verb!\emph{abc}! \ColorStr{red}{=>} \emph{abc}
    \item \verb!\uuline{xxx}! \par
        \verb!\uuline{abc}! \ColorStr{red}{=>} \uuline{abc}
    \item \verb!\uwave{xxx}! \par
        \verb!\uwave{abc}! \ColorStr{red}{=>} \uwave{abc}
    \item \verb!\sout{xxx}! \par
        \verb!\sout{abc}! \ColorStr{red}{=>} \sout{abc}
    \item \verb!\xout{xxx}! \par
        \verb!\xout{abc}! \ColorStr{red}{=>} \xout{abc}
    \item \verb!\dashuline{xxx}! \par
        \verb!\dashuline{abc}! \ColorStr{red}{=>} \dashuline{abc}
    \item \verb!\dotuline{xxx}! \par
        \verb!\dotuline{abc}! \ColorStr{red}{=>} \dotuline{abc}
\end{enumerate}

\paragraph{todo list}~{}

\begin{enumerate}[topsep=0pt,itemsep=0pt,parsep=0pt,leftmargin=3.6em,label=\arabic*>]
    \item \verb!$\square$! \ColorStr{red}{=>} $\square$
    \item \verb!$\boxtimes$! \ColorStr{red}{=>} $\boxtimes$
    \item \verb!$\surd$! \ColorStr{red}{=>} $\surd$
\end{enumerate}

\begin{enumerate}[topsep=0pt,itemsep=0pt,parsep=0pt,leftmargin=3.6em,label=\arabic*>]
    \item $\square$《笔记本》- 笔记本
    \item $\surd$《\sout{笔记本}》- 笔记本
    \item $\boxtimes$\sout{《笔记本》- 笔记本} $\surd$\ColorStr{red}{(done)}
    \item $\square$《笔记本》- 笔记本
\end{enumerate}

\paragraph{roman numerals}~{}

\begin{lstlisting}[language={python}]
I       `-`> 1                          VI      `-`> 6
II      `-`> 2                          VII     `-`> 7
III     `-`> 3                          VIII    `-`> 8
IV      `-`> 4                          IX      `-`> 9
V       `-`> 5                          X       `-`> 10
\end{lstlisting}





\newpage

\subsubsection{special symbols}

\url{https://www.latexlive.com/help#d13}

\url{http://mohu.org/info/symbols/symbols.htm}

\url{https://www.caam.rice.edu/~heinken/latex/symbols.pdf}

特殊符号都是在 \$ xxx \$ 中输入的,或者在输入公式的其他方式中输入,{\color{DefinedColorRed}eg. \$\textbackslash 
bar\{x\}\$}

\newcommand\PrintCode[2]{\textcolor{DefinedColorGreen}{\textbackslash #1\{#2\}}}
\newcommand\PrintSymbol[1]{\textcolor{red}{$#1$}}
\newcommand\SpecCharPair[3]{\PrintSymbol{#1} & \PrintCode{#2}{#3}}

\paragraph{math related}~{}

\begin{longtable}{|c|m{10em}|c|m{10em}|c|m{10em}|}
    \hline
    \SpecCharPair{\sin\theta}{sin}{\textbackslash theta} &
    \SpecCharPair{\cos{\theta}}{cos}{\textbackslash theta} &
    \SpecCharPair{\cosh{h}}{cosh}{h} \\
    \hline
    \SpecCharPair{\sinh g}{sinh}{g} &
    \SpecCharPair{\min{L}}{min}{L} &
    \SpecCharPair{\max{H}}{max}{H} \\
    \hline
    \SpecCharPair{\sup{t}}{sup}{t} &
    \SpecCharPair{\ker{x}}{ker}{x} &
    \SpecCharPair{\Pr{x}}{Pr}{x} \\
    \hline
    \SpecCharPair{\arctan{\frac{L}{T}}}{arctan}{\textbackslash frac\{L\}\{T\}} &
    \SpecCharPair{\arcsin\frac{L}{r}}{arcsin}{\textbackslash frac\{L\}\{r\}} &
    \SpecCharPair{\arccos{\frac{T}{r}}}{arccos}{\textbackslash frac\{T\}\{r\}} \\
    \hline
    \SpecCharPair{\operatorname{argch}l}{operatorname}{\{argch\}l} &
    \SpecCharPair{\operatorname{argsh}k}{operatorname}{\{argsh\}k} &
    \SpecCharPair{\operatorname{th}i}{operatorname}{\{th\}i} \\
    \hline
    \SpecCharPair{\operatorname{ch}h}{operatorname}{\{ch\}h} &
    \SpecCharPair{\operatorname{argth}m}{operatorname}{\{argth\}m} &
    \SpecCharPair{\arg{x}}{arg}{x} \\
    \hline
    \SpecCharPair{\tan{theta}}{tan}{theta} &
    \SpecCharPair{\tanh{i}}{tanh}{i} &
    \SpecCharPair{\limsup{S}}{limsup}{S} \\
    \hline
    \SpecCharPair{\liminf{I}}{liminf}{I} &
    \SpecCharPair{\lg{X}}{lg}{X} &
    \SpecCharPair{\inf{s}}{inf}{s} \\
    \hline
    \SpecCharPair{\exp{\!t}}{exp}{\textbackslash !t} &
    \SpecCharPair{\ln{X}}{ln}{X} &
    \SpecCharPair{\log{X}}{log}{X} \\
    \hline
    \SpecCharPair{\deg{x}}{deg}{x} &
    \SpecCharPair{\det{x}}{det}{x} &
    \SpecCharPair{\dim{x}}{dim}{x} \\
    \hline
    \SpecCharPair{\log{_\alpha{x}}}{log}{\_\textbackslash alpha\{x\}} &
    \SpecCharPair{\hom{x}}{hom}{x} &
    \SpecCharPair{\lim{_{t\to n}T}}{lim}{\_\{t \textbackslash to n\}T} \\
    \hline
\end{longtable}

\begin{longtable}{|c|m{9em}|c|m{9em}|c|m{9em}|c|m{9em}|}
    \hline
    \SpecCharPair{\nabla}{nabla}{} &
    \SpecCharPair{\partial{x}}{partial}{x} &
    \SpecCharPair{\mathrm{d}{x}}{mathrm\{d\}}{x} \\
    \hline
\end{longtable}

\paragraph{relation symbols}~{}
\paragraph{binary operations}~{}
\paragraph{set and/or logic notation}~{}
\paragraph{delimiters}~{}
\paragraph{greek letters}~{}


\begin{longtable}{|c|m{9em}|c|m{9em}|c|m{9em}|c|m{9em}|}
    \hline
    \SpecCharPair{\alpha}{}{alpha} &
    \SpecCharPair{\beta}{}{beta} &
    \SpecCharPair{\gamma}{}{gamma} &
    \SpecCharPair{\delta}{}{delta} \\
    \hline
    \SpecCharPair{\epsilon}{}{epsilon} &
    \SpecCharPair{\varepsilon}{}{varepsilon} &
    \SpecCharPair{\zeta}{}{zeta} &
    \SpecCharPair{\eta}{}{eta} \\
    \hline
    \SpecCharPair{\theta}{}{theta} &
    \SpecCharPair{\vartheta}{}{vartheta} &
    \SpecCharPair{\iota}{}{iota} &
    \SpecCharPair{\kappa}{}{kappa} \\
    \hline
    \SpecCharPair{\lambda}{}{lambda} &
    \SpecCharPair{\mu}{}{mu} &
    \SpecCharPair{\nu}{}{nu} &
    \SpecCharPair{\xi}{}{xi} \\
    \hline
    \PrintSymbol{o} & \textcolor{DefinedColorGreen}{o} &
    \SpecCharPair{\pi}{}{pi} &
    \SpecCharPair{\varpi}{}{varpi} &
    \SpecCharPair{\rho}{}{rho} \\
    \hline
    \SpecCharPair{\varrho}{}{varrho} &
    \SpecCharPair{\sigma}{}{sigma} &
    \SpecCharPair{\varsigma}{}{varsigma} &
    \SpecCharPair{\tau}{}{tau} \\
    \hline
    \SpecCharPair{\upsilon}{}{upsilon} &
    \SpecCharPair{\phi}{}{phi} &
    \SpecCharPair{\varphi}{}{varphi} &
    \SpecCharPair{\chi}{}{chi} \\
    \hline
    \SpecCharPair{\psi}{}{psi} &
    \SpecCharPair{\omega}{}{omega} &
    \SpecCharPair{\Gamma}{}{Gamma} &
    \SpecCharPair{\Delta}{}{Delta} \\
    \hline
    \SpecCharPair{\Theta}{}{Theta} &
    \SpecCharPair{\Lambda}{}{Lambda} &
    \SpecCharPair{\Xi}{}{Xi} &
    \SpecCharPair{\Pi}{}{Pi} \\
    \hline
    \SpecCharPair{\Sigma}{}{Sigma} &
    \SpecCharPair{\Upsilon}{}{Upsilon} &
    \SpecCharPair{\Phi}{}{Phi} &
    \SpecCharPair{\Psi}{}{Psi} \\
    \hline
    \SpecCharPair{\Omega}{}{Omega} \\
    \cline{1-2}
\end{longtable}

\paragraph{other symbols}~{}

\begin{longtable}{|c|m{9em}|c|m{9em}|c|m{9em}|c|m{9em}|}
    \hline
    \SpecCharPair{\bar{x}}{bar}{x} &
    \SpecCharPair{\acute{\eta}}{acute}{\textbackslash eta} &
    \SpecCharPair{\grave{\eta}}{grave}{\textbackslash eta} &
    \SpecCharPair{\check{\alpha}}{check}{\textbackslash alpha} \\
    \hline
    \SpecCharPair{\breve{a}}{breve}{a} &
    \SpecCharPair{\dot{x}}{dot}{x} &
    \SpecCharPair{\hat{\alpha}}{hat}{\textbackslash alpha} &
    \SpecCharPair{\ddot{y}}{ddot}{y} \\
    \hline
    \SpecCharPair{\tilde{\iota}}{tilde}{\textbackslash iota} \\
    \cline{1-2}
\end{longtable}

\paragraph{trigonometric functions}~{}



\subsubsection{shaded 设置段落阴影}

使用 framed package 的 shaded 环境来进行设置,这个阴影环境支持跨页,颜色设置在导言区 \textcolor{red}{eg. \textbackslash 
colorlet\{shadecolor\}\{gray!20\}};并使用 \textcolor{red}{eg. \textbackslash vspace\{-0.8em\}} 来进行前后间距设置

\begin{lstlisting}[language={sh}]
\vspace{`-`0.8em}
\begin{shaded}
a b c d ... 等具体内容
\end{shaded}
\vspace{`-`0.8em}
\end{lstlisting}

\vspace{-0.8em}
\begin{shaded}
在实际的网络中存在如下图所示的情况,因此实际考虑有 “quantize-op” 与 relu 或者 conv 时,
是与 “quantize-op” 之前的 relu 融合还是与 “quantize-op” 之后的 conv 融合的问题
\end{shaded}
\vspace{-0.8em}

\subsubsection{footnote}

\begin{lstlisting}[language={python}]
AAA \footnote{AAA 的注脚内容}
\end{lstlisting}

\subsubsection{embeded image}

\begin{lstlisting}[language={sh}]
\begin{figure}[htbp!]
\setlength{\belowcaptionskip}{`-`1.3em}
\centering
\includegraphics[scale=0.68]{`{\color{red}images/aaa.jpg}`}
\caption{aaa}
\label{fig:aaa}
\end{figure}
\end{lstlisting}


\subsubsection{section}

章节标题的编码方式

\begin{lstlisting}[language={python}]
\chapter{name}
\section{name}
\subsection{name}
\subsubsection{name}

\paragraph{name}~{}
\subparagraph{name}~{}

\subparagraph*{name}~{} # 添加一个 '*' 不生成标题 
\end{lstlisting}






\subsection{other}

%%\subsubsection{chen shuo config}
%%这个是陈硕的 latex 常用设置,其中的反斜杠使用正斜杠替代,这样方便显示
%%
%%https://blog.csdn.net/solstice/article/details/638
%%
%%\paragraph{段首缩进}~{}
%%
%%中文习惯在段首缩进两格,在LaTeX中,/parindent 表示段首缩进的长度,我们将它设置为当前字号的两个
%%大写字母M的宽度,大约正好是两个汉字的宽度:{\color{DefinedColorRed}/setlength\{/parindent\}\{2em\}}
%%
%%LaTeX 默认每节的第一段的段首不缩进,这不符合中文排版习惯。我们希望正文的每一段都要缩进,使用
%%indentfirst宏包就可办到:{\color{DefinedColorRed}/usepackage\{indentfirst\}}
%%
%%\paragraph{段距行距}~{}
%%
%%LaTeX 用/baselineskip表示当前的行距,其默认值大约是当前字号的1.2倍,如果当前字号是10pt,
%%那么/baselineskip是12pt。这对英文排版是合适的,对中文就显得太拥挤了,因为英文正文多为小
%%写字母,字高与小写的x差不多(即1ex)。如果字号为10pt,那么1ex =4.3pt。我通常把行距设为字
%%号的1.8倍:
%%
%%{\color{DefinedColorRed}/setlength\{/baselineskip\}\{1.8em\}}
%%
%%这个值随时可以改,对更改点以后的文字有效
%%
%%LaTeX 用/parskip表示段距,我一般把它设为1ex:{\color{DefinedColorRed}/setlength\{/parskip\}\{1ex\}}
%%
%%注意这些修改长度的命令最好都放在正文区(即/begin\{document\}之后)
%%
%%\paragraph{页眉页脚}~{}
%%
%%/documentclass[10pt, a4paper]{book}
%%/usepackage{fancyhdr}
%%
%%在 LaTeX 中先把 page style 设为fancy,再设置这个style中的页眉和页脚。但是它默认每章的
%%第一页的page style是plain,需要单独处理。
%%
%%\begin{lstlisting}[language={matlab}]
%%% 设置 plain style 的属性
%%/fancypagestyle{plain}{%
%%/fancyhf{}                    % 清空当前设置
%%
%%% 设置页眉 (head)
%%/fancyhead[RE]{/leftmark}     % 在偶数页的右侧显示章名
%%/fancyhead[LO]{/rightmark}    % 在奇数页的左侧显示小节名
%%/fancyhead[LE,RO]{~/thepage~} % 在偶数页的左侧,奇数页的右侧显示页码
%%
%%% 设置页脚:在每页的右下脚以斜体显示书名
%%/fancyfoot[RO,RE]{/it Typesetting with /LaTeX}
%%
%%/renewcommand{/headrulewidth}{0.7pt} % 页眉与正文之间的水平线粗细
%%/renewcommand{/footrulewidth}{0pt}
%%}
%%
%%/pagestyle{fancy}             % 选用 fancy style
%%% 其余同 plain style
%%/fancyhf{}                   
%%/fancyhead[RE]{/leftmark}
%%/fancyhead[LO]{/rightmark}
%%/fancyhead[LE,RO]{~/thepage~}
%%/fancyfoot[RO,RE]{/it Typesetting with /LaTeX}
%%/renewcommand{/headrulewidth}{0.7pt}
%%/renewcommand{/footrulewidth}{0pt}
%%
%%% 设置章名和节名的显示方式
%%/renewcommand{/chaptermark}[1]{/markboth{~第~/thechapter~章~~~#1~}{}}
%%/renewcommand{/sectionmark}[1]{/markright{~/thesection~~#1~}{}}
%%\end{lstlisting}
%%
%%\paragraph{章节标题}~{}
%%
%%通常用titlesec宏包来设置正文中出现的章节标题的格式:
%%
%%/usepackage{titlesec}
%%
%%设置章名为右对齐,字号为/Huge,字型为黑体,章号用粗体,并设置间距:
%%
%%/titleformat{/chapter}{/flushright/Huge/hei}{{/bf /thechapter}}{0pt}{}
%%
%%/titlespacing{/chapter}{0pt}{-20pt}{25pt}
%%
%%设置节名的字号为/Large,字型为黑体,节号用粗体,并设置间距:
%%
%%/titleformat{/section}{/Large /hei}{{/bf /thesection/space}}{0pt}{}
%%
%%/titlespacing*{/section}{0pt}{1ex plus .3ex minus .2ex}{-.2ex plus .2ex}
%%
%%其中/hei的定义为:
%%
%%/newcommand{/hei}{/CJKfamily{hei}}
%%
%%\paragraph{纸张大小}~{}
%%
%%\begin{lstlisting}[language={matlab}]
%%/documentclass[10pt, b5paper]{report}
%%/usepackage[body={12.6cm, 20cm}, centering, dvipdfm]{geometry}
%%% 以上将版心宽度设为 12.6cm,高度 20cm,版心居中,且自动设置PDF文件的纸张大小
%%\end{lstlisting}
%%
%%\subsubsection{documentclass}
%%
%%latex 有三种文档类型(CTE 环境):ctexart、ctexrep、ctexbook
%%
%%优先选用 ctexrep 文档类型,因为这个支持到 chapter,且\textcolor{red}{目录}后面不会添加一个空白页
%%
%%而 book 类型是要考虑印刷时的奇偶页问题,所以会在\textcolor{red}{目录}后面在不满足 chapter 所在页是奇数页时会添加一个空白页,
%%这个是 book 规范设置
%%


\subsection{experience}

\begin{enumerate}[topsep=0pt,itemsep=0pt,parsep=0pt,leftmargin=3.6em,label=\arabic*>]
    \item enumerate 的 list 的 leftmargin 参考设置
    \begin{enumerate}[topsep=0pt,itemsep=0pt,parsep=0pt,leftmargin=1.8em]
        \item 最外层的\ {\color{DefinedColorRed}leftmargin = 3.6em}
        \item 嵌套第一层的\ {\color{DefinedColorRed}leftmargin = 1.8em},并且可以去掉 label 设置
    \end{enumerate}
    \item 在 lstlisting 环境中,如果需要对某行使用 latex 命令来进行特殊设置,比如高亮、斜体、粗体等
        需要在其逃逸符号\ {\color{DefinedColorRed}(反引号)}\ 中使用
\end{enumerate}





\newpage
\subsection{errors}

\begin{lstlisting}[language={c}]
Runaway argument?
{\376\377\0001\000.\0
! File ended while scanning use of \@@BOOKMARK.
<inserted text>
    \par
l.232 \begin{document}}
\end{lstlisting}

遇到这个错误是上一次编译中断后的 aux/out 文件的部分内容缺失,因此只需要删除 *.aux / *.out 文件之后重新编译即可



 % ...



%
% body end ...
%


\end{sloppypar}
\end{document}

